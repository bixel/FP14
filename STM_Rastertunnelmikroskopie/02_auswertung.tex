\section{Messung} % (fold)
\label{sec:messung}
Im Folgenden Abschnitt werden die Messungen der Gitterkonstante von HOPG, sowie
der Plateauhöhen von Gold erläutert. Zum einlesen und einer einfachen
Rauschunterdrückung der Messdaten wird die freie Software \texttt{gwyddion}
\cite{gwyddion} benutzt. Für die Analyse wird \texttt{python} \cite{python3}
verwendet.

Um mit dem Mikroskop verwertbare Bilder zu erzeugen, muss zunächst eine Stück
Wolframdraht mit einer Zange abgetrennt werden. Dabei ist es essentiell, den
Draht abzureissen, statt ihn abzuschneiden. Dadurch wird gewährleistet, dass
ein Ende des Drates eine sehr feine -- im Idealfall einatomige -- Spitze
aufweist.

Die Spitze wird in den Mikroskopschlitten eingelegt und die Probe mit Hilfe von
Piezzoelementen an diese herangefahren.

\subsection{Gittervektoren von HOPG}
\label{subsec:gitter}
Zunächst werden einige Testaufnahmen erstellt, um Mikroskopeinstellungen zu
finden, die die zu untersuchende Gitterstruktur erkennen lassen.
Der Bildausschnitt deckt dabei eine Fläche von $\num{2}\times\num{2}\,
\si{\nano\meter\squared}$ ab. Die Aufnahme, nach Vorverarbeitung durch
\texttt{gwyddion}, ist in Abbildung \ref{fig:hopg1} dargestellt.
\begin{figure}
    \centering
    \includegraphics[width=0.5\linewidth]{build/plots/HOPG_downwards.pdf}
    \caption{Aufnahme von Graphit mit dem Rastertunnelmikroskop. Eine
             periodische Struktur ist klar zu erkennen.}
    \label{fig:hopg1}
\end{figure}
Offensichtlich taucht bei $x \approx \SI{1.05}{\nano\meter}$ eine Unstetigkeit
auf. Außerdem ist die Aufnahme im Bereich $y > \SI{1.5}{\nano\meter}$
anscheinend verzerrt. Diese Werte werden als grobe Grenzen der Daten einer
Bestimmung der Gittervektoren benutzt.

In einem nächsten Schritt werden alle lokalen Maxima in einer Umgebung von
\num{12} Pixeln gesucht und markiert. Im zuvor gewählten Selektionsbereich
werden zudem zwei Korridore mit möglichst vielen Maxima gewählt, die sich an
der grob zu erkennenden Struktur der Gitterstruktur orientieren.
Die grobe Vorselektion, die Maxima und die Selektionskorridore sind in
Abbildung \ref{fig:hopg1_selektion} dargestellt.
An die Punkte innerhalb dieser Korridore werden in einem Least-Squares-Fit
lineare Funktionen angepasst und damit Richtungsvektoren bestimmt. Die Länge
dieser Vektoren wird duch Mittelung über alle benutzten Punkte bestimmt.
Dabei werden zwei Gittervektoren $\vec{a}_1$ und $\vec{a}_2$ der Hexagonalen Struktur, sowie ein Vektor des Ebenenabstandes $\vec{b}$ bestimmt.
Die Vektoren sind in Abbildunge \ref{fig:hopg1_vektoren} eingezeichnet und haben die Werte
\begin{align*}
    \input{build/tex/hopg1_vec_a.tex} \qquad &\text{mit} \qquad
    \input{build/tex/hopg1_vec_a_len.tex}\,,\\
    \input{build/tex/hopg1_vec_b.tex} \qquad &\text{mit}\qquad
    \input{build/tex/hopg1_vec_b_len.tex}\,,\\
    \text{und}\qquad\input{build/tex/hopg1_vec_c.tex} \qquad &\text{mit}\qquad
    \input{build/tex/hopg1_vec_c_len.tex}\,.
\end{align*}

\begin{figure}
    \centering
    \subcaptionbox{
        Grobe Vorselektion der zum Fitten genutzten Datenpunkte.
        \label{fig:hopg1_selektion}
    }[0.6\linewidth]{\includegraphics[width=0.57\linewidth]{build/plots/hopg_down_selection.pdf}}
    \subcaptionbox{
        Gefittete Gittervektoren.
        \label{fig:hopg1_vektoren}
    }[0.39\linewidth]{\includegraphics[width=0.42\linewidth]{build/plots/hopg_down_arrows.pdf}}
    \caption{Vorselektion und Fitresultate der Bestimmung der Gittervektoren.}
    \label{fig:hopg_fit}
\end{figure}

\subsection{Plateauhöhen einer Goldoberfläche}
\label{subsec:gold}

\clearpage
\section{Diskussion}
\label{sec:diskussion}
