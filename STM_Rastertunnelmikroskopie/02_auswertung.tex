\section{Messung} % (fold)
\label{sec:messung}
Im Folgenden Abschnitt werden die Messungen der Gitterkonstante von HOPG, sowie
der Plateauhöhen von Gold erläutert. Zum einlesen der Messdaten und für eine
einfache Rauschunterdrückung wird die freie Software \texttt{gwyddion}
\cite{gwyddion} benutzt. Für die Analyse wird \texttt{python} \cite{python3}
verwendet.

Um mit dem Mikroskop verwertbare Bilder zu erzeugen, muss zunächst eine Stück
Wolframdraht mit einer Zange abgetrennt werden. Dabei ist es essentiell, den
Draht abzureissen, statt ihn abzuschneiden. Dadurch wird gewährleistet, dass
ein Ende des Drates eine sehr feine -- im Idealfall einatomige -- Spitze
aufweist.

Die Spitze wird in den Mikroskopschlitten eingelegt und die Probe mit Hilfe von
Piezzoelementen an diese herangefahren. Die Steuerung wird mit Hilfe der dem
Mikroskop beiliegenden Software \emph{Zitation einfügen} durchgeführt.
Sobald der Abstand so klein ist, dass ein Tunnelstrom gemessen wird, fährt
die Probe ein zuvor definiertes Raster in $x$- und $y$-Richtung ab. Das so
entstehende Bild wird gespeichert und ausgewertet.

\subsection{Gittervektoren von HOPG}
\label{subsec:gitter}
Zunächst werden einige Testaufnahmen erstellt, um Mikroskopeinstellungen zu
finden, die die zu untersuchende Gitterstruktur erkennen lassen.
Der Bildausschnitt deckt dabei eine Fläche von $\num{2}\times\num{2}\,
\si{\nano\meter\squared}$ ab. Die Aufnahme, nach Vorverarbeitung durch
\texttt{gwyddion}, ist in Abbildung \ref{fig:hopg1} dargestellt.
\begin{figure}
    \centering
    \includegraphics[width=0.5\linewidth]{build/plots/HOPG_downwards.pdf}
    \caption{Aufnahme von Graphit mit dem Rastertunnelmikroskop. Eine
             periodische Struktur ist klar zu erkennen.}
    \label{fig:hopg1}
\end{figure}
Offensichtlich taucht bei $x \approx \SI{1.05}{\nano\meter}$ eine Unstetigkeit
auf. Außerdem ist die Aufnahme im Bereich $y > \SI{1.5}{\nano\meter}$
anscheinend verzerrt. Diese Werte werden als grobe Selektion der Daten
benutzt, mit denen die Gittervektoren bestimmt werden.

In einem nächsten Schritt werden alle lokalen Maxima in einer Umgebung von
\num{12} Pixeln gesucht und markiert. Im zuvor gewählten Selektionsbereich
werden zudem zwei Korridore mit möglichst vielen Maxima gewählt, die sich an
der grob zu erkennenden Struktur der Gitterstruktur orientieren.
Die grobe Vorselektion, die Maxima und die Selektionskorridore sind in
Abbildung \ref{fig:hopg1_selektion} beispielhaft dargestellt.
An die Punkte innerhalb dieser Korridore werden in einem Least-Squares-Fit
lineare Funktionen angepasst und damit Richtungsvektoren bestimmt. Die Länge
dieser Vektoren wird duch Mittelung über alle benutzten Punkte bestimmt.
Dabei werden zwei Gittervektoren $\vec{a}_1$ und $\vec{a}_2$ bestimmt.
Die Vektoren sind in Abbildunge \ref{fig:hopg1_vektoren} eingezeichnet.
Zudem wird der Winkel zwischen beiden Vektoren bestimmt.
Diese Analyse wird für alle vier Messungen durchgeführt. Es ergeben sich die
in Tabelle \ref{tab:vektoren} aufgeführten Werte.
\begin{table}
    \centering
    \caption{
        Werte der durch Fit bestimmten Gittervektoren von HOPG. Es sind $x$-
        und $y$-Komponenten der Vektoren $\vec{a}_i$, deren Länge $a_i$, sowie
        der Winkel $\alpha$ zwischen beiden Vektoren aufgeführt.
    }
    \label{tab:vektoren}
    \input{build/tex/table_vec.tex}
\end{table}
Mittelung über alle Messungen ergibt
\begin{align*}
    \input{build/tex/avg_a1.tex}\,,\\
    \input{build/tex/avg_a2.tex}\,,\\
    \text{und}\qquad\input{build/tex/avg_a1_len.tex}\,,\\
    \input{build/tex/avg_a2_len.tex}\,,\\
    \text{sowie}\qquad\input{build/tex/avg_angle.tex}\,.
\end{align*}

Durch Vergleich mit den Literaturwerten \cite{HOPG_gittervektoren} $\hat{a}
= \SI{2.461}{\angstrom}$ und dem erwarteten Winkel $\alpha = \SI{60}{\degree}$
lässt sich
eine Skalierung $s_x$ bzw. $s_y$ der $x$- und $y$- Achse finden, sodass die
gemessenen Werte für $\left|\vec{a}_i\right|$ mit diesen übereinstimmen.
Diese Transformation lässt sich mit einer Diagonalmatrix $S$ ausdrücken als
$\vec{a}_i \to S\vec{a}_i$ und die zu erfüllenden Bedingungen lauten
\begin{align*}
    \frac{S\vec{a}_1\vec{a}_2}
         {\sqrt{\left(S\vec{a}_1\right)^2 \left(S\vec{a}_2\right)^2}}
         &= \cos{\alpha}\,\\
         \text{und}\qquad\left(S\vec{a}_1\right)^2
         &= \left(S\vec{a}_1\right)^2 = \hat{a}^2\,.
\end{align*}
Daraus folgt für ein gegebenes $s_y$
\begin{align*}
    s_x^2 &= \frac{\hat{a}^2\cos{\alpha} - s_y^2 a_{1,y} a_{2,y}}
                  {a_{1,x} a_{2,x}}\,.
\end{align*}
Die Skalierungsfaktoren sind schließlich $\input{build/tex/scale_x.tex}$ und
$\input{build/tex/scale_y.tex}$. Eine entsprechend Skalierte Darstellung der
Messdaten ist in Abbildung \ref{fig:scale} gezeigt.
\begin{figure}
    \centering
    \subcaptionbox{
        Grobe Vorselektion der zum Fitten genutzten Datenpunkte.
        \label{fig:hopg1_selektion}
    }[0.6\linewidth]{\includegraphics[width=0.57\linewidth]{build/plots/hopg_down_selection.pdf}}
    \subcaptionbox{
        Gefittete Gittervektoren.
        \label{fig:hopg1_vektoren}
    }[0.39\linewidth]{\includegraphics[width=0.42\linewidth]{build/plots/hopg_down_arrows.pdf}}
    \caption{Vorselektion und Fitresultate der Bestimmung der Gittervektoren.}
    \label{fig:hopg_fit}
\end{figure}

\subsection{Plateauhöhen einer Goldoberfläche}
\label{subsec:gold}

\clearpage
\section{Diskussion}
\label{sec:diskussion}
