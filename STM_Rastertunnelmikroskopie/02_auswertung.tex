\section{Messung} % (fold)
\label{sec:messung}
Im Folgenden Abschnitt werden die Messungen der Gitterkonstante von HOPG, sowie
der Plateauhöhen von Gold erläutert. Zum einlesen und einer einfachen
Rauschunterdrückung der Messdaten wird die freie Software \texttt{gwyddion}
benutzt. Für die Analyse wird \texttt{python} verwendet.

Um mit dem Mikroskop verwertbare Bilder zu erzeugen, muss zunächst eine Stück
Wolframdraht mit einer Zange abgetrennt werden. Dabei ist es essentiell, den
Draht abzureissen, statt ihn abzuschneiden. Dadurch wird gewährleistet, dass
ein Ende des Drates eine sehr feine -- im Idealfall einatomige -- Spitze
besitzt.

Die Spitze wird in den Mikroskopschlitten eingelegt und die Probe mit Hilfe von
Piezzoelementen an diese herangefahren.

\subsection{Gittervektoren von HOPG}
\label{subsec:gitter}


\subsection{Plateauhöhen einer Goldoberfläche}
\label{subsec:gold}

\section{Diskussion}
\label{sec:diskussion}
