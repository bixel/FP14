\section{Messung} % (fold)
\label{sec:messung}

Im Folgenden wird zunächst die Stärke des Magnetfeldes bestimmt. 
Anschließend wird die Faraday-Rotation bestimmt und daraus die Masse der freien Elektronen.

Aufgrund der Starken schwankungen der Messpunkte wird auf eine Fehlerrechnung verzichtet.

\subsection{Bestimmung der Magnetfeldstärke} % (fold)
\label{sub:bestimmung_der_magnetfeldstärke}

Die zur Bestimmung der Magnetfeldstärke verwendeten Daten sind in Tabelle \ref{tab_mag} dargestellt.
Zur Bestimmung des Mittelpunktes der symmetrischen Kurve wurde sowohl eine Ausgleichrechung mithilfe einer Gauß-, als auch einer Lorentz-Funktion durchgeführt (Abbildung \ref{fig_mag}).
Beide Funktionen haben ihren Mittelpunkt bei
\begin{equation}
	z_\text{max} \approx 29.74.
\end{equation}

Somit bestimmt sich die Magnetfeldstärke zu
\begin{equation}
	B(z_\text{max}) \approx \SI{455}{\milli\tesla},
\end{equation}
welche im weiteren Verlauf der Rechnung verwendet wird.

\begin{figure}
	\inncludegraphics[width = 7cm]{data/gauß.pdf}
	\caption{Messpunkte und Ausgleichsrechnungen zur Bestimmung der Magnetfeldstärke.}
	\label{fig_mag}
\end{figure}

\begin{table}
	\centering
	\caption[]{Zur Bestimmung der Magnetfeldstärke verwendete Messpunkte.}
	\begin{tabular}{r r r r}
		d\,[mm] & B\,[mT] & d\,[mm] & B\,[mT]\\
		\hline\hline
		 0	&	  0.80	&	30	&	455.00\\
		 2	&	  1.28	&	31	&	454.00\\
		 4	&	  2.19	&	32	&	451.00\\
		 6	&	  3.81	&	33	&	446.00\\
		 8	&	  6.86	&	35	&	427.00\\
		10	&	 12.24	&	36	&	412.00\\
		12	&	 22.40	&	38	&	360.00\\
		14	&	 42.10	&	40	&	266.00\\
		16	&	 83.00	&	42	&	153.00\\
		18	&	168.60	&	44	&	 78.00\\
		20	&	301.00	&	46	&	 40.60\\
		22	&	389.00	&	48	&	 21.20\\
		24	&	428.00	&	50	&	 11.90\\
		25	&	439.00	&	52	&	  6.60\\
		26	&	445.00	&	54	&	  3.76\\
		27	&	451.00	&	56	&	  2.19\\
		28	&	454.00	&	58	&	  1.31\\
		29	&	455.00	&	60	&	  0.82\\
		\hline
	\end{tabular}
	\label{tab_mag}
\end{table}
\FloatBarrier
\subsection{Bestimmung der Masse von freien Elektronen} % (fold)
\label{sub:bestimmung_der_masse_von_freien_elektronen}

Die zur berechnung der Masse freier Elektronen verwendeten Daten sind in Tabelle \ref{tab_1} bis \ref{tab_2} dargestellt.
Zunächst werden die Winkel in rad umgerechnet wobei
\begin{eqnarray}
	1 \text[Bs] = \frac{pi}{648000} \text[rad],\\
	1 \text[°] = \frac{pi}{180} \text[rad]
\end{eqnarray}
gilt.

Anschließend wird der Winkel noch auf die Dicke $D$ der Probe normiert und die Winkel der hochreinen Probe von den Winkeln der anderen beiden Proben abgezogen.
Somit ergeben sich die Winkel aus Tabelle \ref{wink1}

Somit sollte nur der Anteil der n-dotierung übrig bleiben, welcher aus den zu untersuchenden freien Elektronen besteht.
Die Daten der Proben sind in Tabelle \ref{data} aufgeführt.

Aus dem Winkel lässt sich mithilfe einer linearen Regression mit
\begin{equation}
	\Theta / D = m_\text{reg} \lambda^2 + b
\end{equation}
die Masse der Elektronen über
\begin{equation}
	m_\text{elek} = \sqrt{\frac{e^3 N B}{8 \pi^2 \epsilon_0 c^3 n m_\text{reg}}}
\end{equation}
bestimmen.




\begin{table}
	\centering
	\caption{Daten der verwendeten Proben.}
	\begin{tabular}{r r r}
		Probe & D \,[mm]& N\,[e18 1/cm³]\\ 
		\hline	\hline
		GeAs hochrein & 5.11 & 0\\
		GeAs n-dotiert dünn & 1.296 & 2.8\\
		GeAs n-dotiert dick & 1.460 & 1.2\\
	\end{tabular}
\end{table}


\begin{table}
	\centering
	\caption{Drehwinkel der n-dotierten Proben, nachdem der Winkel der hochreinen Probe abgezogen wurde.}
	\begin{tabular}
		GeAs n-dotiert dünn & GeAs n-dotiert dick\\
		\hline \hline
		102.42	&	 26.04\\
		 94.23	&	 50.89\\
		107.44	&	 56.92\\
		 75.55	&	 26.51\\
		 46.94	&	 43.68\\
		 53.59	&	  5.16\\
		 -7.11	&	 24.01\\
		-35.89	&	 85.58\\
		 14.30	&	 23.62\\

	\end{tabular}
\end{table}

\begin{table}
	\centering
	\caption[]{Drehwinkel der hochreinen Probe.}
	\begin{tabular}{r r r r}
		Filter\,[$\mu$m] & Polarisation & Winkel\,[°] & Winkel\,[Bs]\\
		\hline\hline
		2.650	&	+	&	194	&	51\\
				&	-	&	205	&	55\\
		2.510	&	+	&	197	&	33\\
				&	-	&	201	&	59\\
		2.340	&	+	&	194	&	 3\\
				&	-	&	202	&	 2\\
		2.156	&	+	&	194	&	52\\
				&	-	&	201	&	42\\
		1.960	&	+	&	192	&	40\\
				&	-	&	204	&	15\\
		1.720	&	+	&	192	&	58\\
				&	-	&	204	&	31\\
		1.450	&	+	&	191	&	 4\\
				&	-	&	207	&	42\\
		1.200	&	+	&	184	&	36\\
				&	-	&	209	&	35\\
		1.060	&	+	&	179	&	 5\\
				&	-	&	214	&	55\\
		\hline
	\end{tabular}
	\label{tab_1}
\end{table}

\begin{table}
	\centering
	\caption[]{Drehwinkel der dünnen Probe.}
	\begin{tabular}{r r r r}
		Filter\,[$\mu$m] & Polarisation & Winkel\,[°] & Winkel\,[Bs]\\
		\hline\hline
		2.650	&	+	&	187	&	 2
				&	-	&	205	&	 3
		2.510	&	+	&	192	&	 1
				&	-	&	207	&	39
		2.340	&	+	&	191	&	57
				&	-	&	209	&	 0
		2.156	&	+	&	192	&	20
				&	-	&	205	&	 3
		1.960	&	+	&	193	&	 0
				&	-	&	203	&	44
		1.720	&	+	&	194	&	 7
				&	-	&	205	&	10
		1.450	&	+	&	194	&	13
				&	-	&	197	&	30
		1.200	&	+	&	195	&	50
				&	-	&	194	&	11
		1.060	&	+	&	192	&	38
				&	-	&	203	&	54
		\hline
	\end{tabular}
	\label{tab_2}
\end{table}

\begin{table}
	\centering
	\caption[]{Drehwinkel der dicken Probe.}
	\begin{tabular}{r r r r}
		Filter\,[$\mu$m] & Polarisation & Winkel\,[°] & Winkel\,[Bs]\\
		\hline\hline
		2.650	&	+	&	194	&	59\\
				&	-	&	201	&	 7\\
		2.510	&	+	&	193	&	40\\
				&	-	&	202	&	32\\
		2.340	&	+	&	195	&	49\\
				&	-	&	206	&	46\\
		2.156	&	+	&	195	&	27\\
				&	-	&	201	&	 6\\
		1.960	&	+	&	193	&	 4\\
				&	-	&	203	&	 2\\
		1.720	&	+	&	196	&	54\\
				&	-	&	200	&	38\\
		1.450	&	+	&	195	&	14\\
				&	-	&	203	&	25\\
		1.200	&	+	&	184	&	46\\
				&	-	&	204	&	15\\
		1.060	&	+	&	191	&	32\\
				&	-	&	204	&	31\\
		\hline
	\end{tabular}
	\label{tab_2}
\end{table}