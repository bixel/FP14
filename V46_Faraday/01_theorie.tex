\section{Grundlagen} % (fold)
\label{sec:grundlagen}
In diesem Versuch soll die effektive Masse von Elektronen in Halbleitern
besitmmt werden.
Dabei wird die Faraday-Rotation zu Nutzen gemacht, die eine Drehung
der Polarisationsebene von Licht in, von einem Magnetfeld umgebenen,
Material beschreibt.
Zunächst werden einige physikalische Grundlagen erläutert, um anschließend
die Vorgehensweise und danach zur Messung zu kommen.

\subsection{Effektive Massen in Halbleitern} % (fold)
\label{sub:effektive_masse}
Halbleiter zeichnen sich durch eine schmale Bandlücke zwischen Leitungs-
und Valenzband aus.
Während die genaue Struktur der Energiebänder komplex ist, reicht meist
eine vereinfachte Betrachtung aus, um physikalische Verhaltensweisen zu
beschreiben.
Hierfür lässt sich die Funktion der Elektronenenergie in Abhängigkeit des
Wellenvektors $\vec{k}$ im Phasenraum
$\epsilon(\vec{k})$ als Taylorreihe bis zum Term zweiter Ordnung entwickeln:
\begin{equation}
    \label{eqn:energie}
    \epsilon\left(\vec{k}\right) = \epsilon(0) + \frac{1}{2}\sum_{i=1}^{3}\frac{\partial^2\epsilon}{\partial k_i^2}\cdot k_i^2 + \dots\,.
\end{equation}
Wobei die effektive Masse $m^*$ durch einen Vergleich mit
\begin{equation*}
    \epsilon = \frac{\hbar^2 k^2}{2m}
\end{equation*}
mit
\begin{equation}
    m_i^* = \frac{\hbar^2}{\left.\frac{\partial^2\epsilon}{\partial k_i^2}\right|_{k=0}}
\end{equation}
identifiziert werden kann. Der große Vorteil dieser Betrachtung ist die
Möglichkeit der Behandlung der Elektronen als Freie Quantenmechanische
Teilchen.
\begin{figure}[h]
    \centering
    \includegraphics[width=0.8\linewidth]{img/band.pdf}
    \caption{
        Einfache Näherung der Struktur der Energiebänder in einem Halbleiter
        \cite{V46}.
    }
    \label{fig:band}
\end{figure}
In Einem symmetrischen Kristall sind die Ausprägungen vom $m_i^*$ in allen
Raumrichtungen identisch und die Funktion \eqref{eqn:energie} lässt sich
vereinfachen zu
\begin{equation}
    \label{eqn:energie_einfach}
    \epsilon\left(\vec{k}\right) = \epsilon(0) + \frac{\hbar^2k^2}{2m^*}\,,
\end{equation}
was den Lösungen der Schrödingergleichung für freie Elektronen entspricht.
Die Notwendigkeit der Betrachtung des periodischen Potentials in einem Kristall
fällt damit weg und der Hamiltonoperator vereinfacht sich zu
\begin{equation*}
    \frac{\hbar^2}{2m_0}\mathrm{\Delta} + V(\vec{r})
    \quad\Rightarrow\quad
    \frac{\hbar^2}{2m^*}\mathrm{\Delta}\,.
\end{equation*}
Für kleine äußere Felder lässt sich das verhalten der Teilchen sogar
klassisch betrachten und kann durch die Newtonschen Gesetzte beschrieben
werden.
% subsection effektive_masse (end)

\subsection{Zirkulare Doppelbrechung} % (fold)
\label{sub:doppelbrechung}
Einige Kristalle haben die Fähigkeit, die Polarisationsebene von linear
polarisiertem Licht zu drehen. Eine Welle $E(z)$, die sich in $z$-Richtung
durch den Kristall ausbreitet hat an der Stelle $L$ im Kristall die Form
\begin{equation}
    \label{eqn:energie_final}
    E(L) = E_0\e^{\i\psi}\left(
        \cos\theta\vec{x_0} + \sin\theta\vec{y_0}
    \right)\,.
\end{equation}
Hierbei wurde die linear polarisiert Welle in einen links- und einen
rechtspolarisierten Anteil mit den jeweiligen Wellenzahlen $k_\text{L}$ und
$k_\text{R}$ zerlegt und die Größen
\begin{align*}
    \psi   &= \frac{L}{2}\left(k_\text{R}+k_\text{L}\right)\\
    \text{und}\qquad\theta &= \frac{L}{2}\left(k_\text{R}-k_\text{L}\right)\\
\end{align*}
eingeführt.
Die Größe $\theta$ kann zudem mit Hilfe der Phasengeschwindigkeit
$v_\text{Ph} = \omega/k$ oder dem Brechungsindex $n=c/v_\text{Ph}$ darstellen,
wobei die Größen wiederum bezüglich der zirkularen Polarisationsrichtungen
$R$ und $L$ betrachtet werden:
\begin{equation}
    \label{eqn:theta}
    \theta = \frac{L\omega}{2}
    \left(
        \frac{1}{v_\text{Ph,R}} - \frac{1}{v_\text{Ph,L}}
    \right)
    = \frac{L\omega}{2c}\left(n_\text{R}-n_\text{L}\right)\,.
\end{equation}
Im Kristall bilden die Elektronen mit den Atomrümpfen Dipolmomente aus,
die durch die einfallende Lichtwelle angeregt werden und diese weiter
übertragen können.
Insgesamt bildet sich so eine Polarisation $\vec{P}$ des Kristalles aus,
für die mit der Influenzkonstante $\epsilon_0$ und der dielektrischen
Suszeptibilität $\chi$ gilt
\begin{equation}
    \label{eqn:polarisation}
    \vec{P} = \epsilon_0\chi\vec{E}\,.
\end{equation}
Die Suszeptibilität ist dabei in isotropen Materialien eine skalare Größe,
wird allgemein jedoch als oft symmetrischer Tensor betrachtet:
\begin{equation}
    \label{eqn:suszeptibilitaet_inaktiv}
    \mathbf{\chi} =
    \left(\begin{array}{ccc}
        \chi_{xx} & 0 & 0 \\
        0 & \chi_{yy} & 0 \\
        0 & 0 & \chi_{xx} \\
    \end{array}\right)\,.
\end{equation}
Die oben genannte Drehung der Polarisationsebene tritt nun genau dann auf, wenn
die Nebendiagonalelemente $\chi_{xy}$ nicht alle verschwinden, sondern der
Tensor die Gestalt
\begin{equation}
    \label{eqn:suszeptibilitaet_aktiv}
    \mathbf{\chi} =
    \left(\begin{array}{ccc}
        \chi_{xx}    & \i\chi_{xy} & 0 \\
        -\i\chi_{xy} & \chi_{yy}   & 0 \\
        0            & 0           & \chi_{xx} \\
    \end{array}\right)
\end{equation}
annimmt. In diesem Fall treten, wie in der Anleitung \cite{V46} mit Hilfe
des Ansatzes einer ebenen Welle in $z$-Richtung gezeigt, verschiedene
Phasengeschwindigkeiten $v_\text{Ph,R}$, $v_\text{Ph,L}$ in für rechts- und
linkspolarisiertes Licht auf.
Für die Größe $\theta$ gilt dann näherungesweise
\begin{equation}
    \label{eqn:theta_cirkular}
    \theta \approx \frac{L\omega}{2c}\frac{1}{\sqrt{1+\chi_{xx}}}\chi_{xy}\,,
\end{equation}
was mit der Phasengeschwindigkeit $v_\text{Ph}$ oder dem Brechungsindex $n$
des Materials auch ausgedrückt werden kann mit
\begin{equation}
    \label{eqn:theta_n}
    \theta
    \approx \frac{L\omega}{2c^2}v_\text{Ph}\chi_{xy}
    = \frac{L\omega}{2cn}\chi_{xy}\,.
\end{equation}
% subsection doppelbrechung (end)

\subsection{Faraday-Effekt} % (fold)
\label{sub:faraday_effekt}
Der Faraday-Effekt beschreibt nun prinzipiell die Erzeugung doppelt brechender
Eigenschaften eines eigentlich optisch inaktiven Materials mit Hilfe eines
äußeren Magnetfeldes.
Ein Kristall, der von einem Suszeptibilitätstensor der Form
\eqref{eqn:suszeptibilitaet_inaktiv} beschrieben wird gelangt in einem
Magnetfeld somit zu einer Form wie \eqref{eqn:suszeptibilitaet_aktiv}.
Dabei werden effektiv die schwingenden Elektronen, die als Dipole vom
einfallenden Licht angeregt werden, vom äußeren Magenetfeld in ihrer
Bahn beeinflusst.
Der Polarisationswinkel $\theta$ wird nun abhängig vom Magnetfeld
$B$ und der Ladungsträgerdichte $N$.
Schließlich kann somit eine Näherung für quasifreie Ladungsträger,
also die Elektronen im Leitungsband aufgestellt werden:
\begin{equation}
    \label{eqn:theta_final}
    \theta_\text{frei}
    \approx \frac{e_0^3}{8\pi^2\epsilon_0 c^3}
    \frac{1}{m^2}\lambda^2 \frac{NBL}{n}\,.
\end{equation}
Hierbei tauchen die Elementarladung $e_0$ und die Wellelänge $\lambda$ des
einfallenden Lichtes auf.

% subsection faraday_effekt (end)
% section grundlagen (end)
\clearpage
\section{Messung des Polarisationswinkels $\theta$} % (fold)
\label{sec:messung}
In der hier durchgeführten Messung kann die Masse $m$ in der oben genannten
Gleichung \eqref{eqn:theta_final} mit der effektiven Masse $m^*$ der
Elektronen identifieziert werden.
Somit ermöglicht die Bestimmung des Polarisationswinkels $\theta$ eine Messung
der effektiven Masse $m^*$.

\begin{figure}[h]
    \centering
    \includegraphics[width=0.9\linewidth]{img/aufbau.pdf}
    \caption{
        Schematische Darstellung der Messapparatur \cite{V46}.
    }
    \label{fig:aufbau}
\end{figure}
Der Versuchsaufbau ist in Abbildung \ref{fig:aufbau} schematisch dargestellt.
Der zu untersuchende Halbleiterkristall kann in einen Elektromagneten
eingebracht werden,
dessen Magnetfeldrichtung umpolbar sind, um den Winkel $\theta$
effektiv verdoppeln zu können.
Als Lichtquelle dient eine Halogen-Lampe, deren Licht zunächst mit Hilfe eines
Interferenzfilters monochromatisiert und Anschließend mit einem
drehbaren Glan-Thompson-Primsa polarisiert wird.
Zudem ist ein Lichtzerhacker hinter der Lampe angebracht, der mit einem
Selektivverstärker gekoppelt wird, um Rauschen zu unterdrücken.
Das Licht wird hinter der Probe durch ein weiteres Glan-Thompson-Primsa in
zwei, senkrecht zueinander stehende Polarisationsebenen aufgeteilt, deren
Intensität jeweils mittels Photowiderständen in ein Spannungssignal
umgewandelt wird.
Ein Differenzverstärker gibt schließlich das Differenzsignal der
Photospannungen auf ein Oszilloskop, dessen Amplitude dann proportional zur
Drehung der Polarisationsebene des ausfallenden Lichtes ist.

Nachdem die Apparatur justiert ist und das Signal auf dem Oszilloskop
ohne Probe und ohne Magnetfeld tatsächlich keine Intensität anzeigt,
kann eine Probe in den Elektromagneten eingebracht und das Magnetfeld
hochgefahren werden.
Das polarisiert Licht aus dem ersten Glan-Thompson-Primsa wird nun in der
Probe gedreht, wobei das Signal wächst.
Durch Einstellen des maximalen Signals bei beiden Magnetfeldrichtungen
lässt sich nun der Rotationswinkel $\theta$ bestimmen:
\begin{equation}
    \label{eqn:theta_messung}
    \theta = \frac{1}{2}\left(\theta_1-\theta_2\right)\,.
\end{equation}
Hierbei bezeichnen $\theta_1$ und $\theta_2$ die beiden Stellungen des ersten
Glan-Thompson-Primsas.

% section messung (end)
