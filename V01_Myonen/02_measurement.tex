\section{Messung der Lebensdauer kosmischer Myonen}
\label{sec:messung}
In diesem Abschnitt wird die Bestimmung der Lebensdauer kosmischer Myonen
vorgestellt.
Zunächst muss hierfür die Messapparatur kalibriert werden.
Außerdem wird durch Messung der Myonen-Zählrate unter Variation der
Verzögerungsleitungen eine Zeitauflösung der Apparatur ermittelt.

\subsection{Zeitkalibration der Apparatur}
\label{subsec:kalibration}
Die auszuwertenden Daten werden in dieser Messung mit Hilfe eines
Vielkanalanalysators (VKA) gewonnen.
Dieser liefert Histogramm-Daten von Myonen-Kandidaten, die in \num{512} Kanäle
eingeteilt sind.
Die Kanalnummer $C_i$ eines Ereignisses hängt dabei linear mit der
Zeit $t_i$ des entsprechenden Myon-Kandidaten zwischen Start- und
Stopp-Impuls zusammen:
\begin{equation}
    t_i = a + b \cdot C_i\,,
\end{equation}
mit den Koeffizienten $a$ und $b$, die im Folgenden bestimmt werden.

Zur Kalibration der Apparatur werden in zeitlich konstanten Abständen
Signalimpulse auf den VKA gegeben.
Diese Impulse füllen jeweils einen spezifischen Kanal, wobei die Entsprechende
Zeit $t_i$ bekannt ist.
Durch Einstellung verschiedener Signalzeiten können mehreren Kanälen eine
Zeit zugeordnet werden.
Die Software \texttt{Maestro V6.06} des VKA führt daraufhin eine lineare Ausgleichsrechnung mit diesen
Kanälen durch und liefert so die Parameter
\begin{align*}
     a &= \SI{0.023}{\micro \second}\\
     \text{und} \quad b &= \SI[per-mode=fraction]{0.047}{\micro \second \per {Kanal}} \,.
 \end{align*}
 Im Folgenden wird lediglich der Parameter $b$ von Interesse sein.

\subsection{Zeitauflösung der Apparatur}
\label{subsec:zeitaufloesung}
Um verschiedene Signallaufzeiten der beiden SEV auszugleichen, werden vor der
Messung Verzögerungsleitungen angeschlossen, die eine zusätzliche
Verzögerungszeit $T_\text{VZ}$ einbringen.
Durch Messung der Myonen-Zählraten unter Variation der Verzögerungszeit
lässt sich so eine Auflösungszeit $\Delta t_\text{K}$ der Koinzidenzeinheit
als Breite einer Gaußkurve bestimmen.
Die Messwerte sind in Tabelle \ref{tab:resolution} aufgeführt, Abbildung
\ref{fig:resolution} zeigt den Fit einer Gauß-Funktion an die Datenpunkte.
\begin{figure}
    \centering
    \includegraphics[width=0.7\linewidth]{img/resolution.pdf}
\end{figure}
