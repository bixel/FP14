\section{Messung der Lebensdauer kosmischer Myonen}
\label{sec:messung}
In diesem Abschnitt wird die Bestimmung der Lebensdauer kosmischer Myonen
vorgestellt.
Zunächst muss hierfür die Messapparatur kalibriert werden.
Außerdem wird durch Messung der Myonen-Zählrate unter Variation der
Verzögerungsleitungen eine Zeitauflösung der Apparatur ermittelt.

\subsection{Zeitkalibration der Apparatur}
\label{subsec:kalibration}
Die auszuwertenden Daten werden in dieser Messung mit Hilfe eines
Vielkanalanalysators (VKA) gewonnen.
Dieser liefert Histogramm-Daten von Myonen-Kandidaten, die in \num{512} Kanäle
eingeteilt sind.
Die Kanalnummer $C_i$ eines Ereignisses hängt dabei linear mit der
Zeit $t_i$ des entsprechenden Myon-Kandidaten zwischen Start- und
Stopp-Impuls zusammen:
\begin{equation}
    t_i = a + b \cdot C_i\,,
\end{equation}
mit den Koeffizienten $a$ und $b$, die im Folgenden bestimmt werden.

Zur Kalibration der Apparatur werden in zeitlich konstanten Abständen
Signalimpulse auf den VKA gegeben.
Diese Impulse füllen jeweils einen spezifischen Kanal, wobei die Entsprechende
Zeit $t_i$ bekannt ist.
Durch Einstellung verschiedener Signalzeiten können mehreren Kanälen eine
Zeit zugeordnet werden.
Die Software \texttt{Maestro V6.06} des VKA führt daraufhin eine lineare Ausgleichsrechnung mit diesen
Kanälen durch und liefert so die Parameter
\begin{align*}
     a &= \SI{0.023}{\micro \second}\\
     \text{und} \quad b &= \SI[per-mode=fraction]{0.047}{\micro \second \per {Kanal}} \,.
 \end{align*}
 Im Folgenden wird lediglich der Parameter $b$ von Interesse sein.

\subsection{Zeitauflösung der Apparatur}
\label{subsec:zeitaufloesung}
Um verschiedene Signallaufzeiten der beiden SEV auszugleichen, werden vor der
Messung Verzögerungsleitungen angeschlossen, die eine zusätzliche
Verzögerungszeit $T_\text{VZ}$ einbringen.
Durch Messung der Myonen-Zählraten unter Variation der Verzögerungszeit
lässt sich so eine Auflösungszeit $\Delta t_\text{K}$ der Koinzidenzeinheit
als Breite einer Gaußkurve bestimmen.
Die Messwerte sind in Tabelle \ref{tab:resolution} aufgeführt, Abbildung
\ref{fig:resolution} zeigt den Fit einer Gauß-Funktion an die Datenpunkte,
wodurch sich die folgende Auflösungszeit ergibt:
\begin{equation*}
    \Delta t_\text{K} = \SI{4.50+-0.11}{\nano \second}\,.
\end{equation*}
\begin{table}
    \centering
    \caption{
        Messwerte zur Bestimmung der zu verwendenden Verzögerungsleitung
        und Auflösungszeit $\Delta t_\text{K}$ der Koinzidenzeinheit.
    }
    \label{tab:resolution}
    \begin{tabular}{SSSSS}
        \toprule
        {Leitung 1} & {Leitung 2} & {$T_\text{VZ}$} & {Ereignisse} & {Rate [\si{\hertz}]} \\
        \midrule
        16.0 & 0.0  &  -16.0 &     0 &    0.00 \\
        16.0 & 2.0  &  -14.0 &     1 &    0.03 \\
        16.0 & 4.0  &  -12.0 &     1 &    0.03 \\
        16.0 & 6.0  &  -10.0 &    14 &    0.47 \\
        16.0 & 8.0  &   -8.0 &    63 &    2.10 \\
        16.0 & 10.0 &   -6.0 &   179 &    5.97 \\
        16.0 & 12.0 &   -4.0 &   261 &    8.70 \\
        16.0 & 14.0 &   -2.0 &   408 &   13.60 \\
        16.0 & 14.5 &   -1.5 &   389 &   12.97 \\
        16.0 & 15.0 &   -1.0 &   385 &   12.83 \\
        16.0 & 15.5 &   -0.5 &   409 &   13.63 \\
        16.0 & 16.0 &    0.0 &   436 &   14.53 \\
        16.0 & 16.5 &    0.5 &   444 &   14.80 \\
        16.0 & 17.0 &    1.0 &   392 &   13.07 \\
        16.0 & 18.0 &    2.0 &   417 &   13.90 \\
        16.0 & 20.0 &    4.0 &   324 &   10.80 \\
        16.0 & 22.0 &    6.0 &   191 &    6.37 \\
        16.0 & 24.0 &    8.0 &   104 &    3.47 \\
        16.0 & 26.0 &   10.0 &    34 &    1.13 \\
        16.0 & 28.0 &   12.0 &     8 &    0.27 \\
        16.0 & 30.0 &   14.0 &     4 &    0.13 \\
        16.0 & 32.0 &   16.0 &     3 &    0.10 \\
        \bottomrule
    \end{tabular}
\end{table}
\begin{figure}[htb]
    \centering
    \includegraphics[width=0.7\linewidth]{img/resolution.pdf}
    \caption{
        Myonenrate in Abhängigkeit der effektiven Signalverzögerung
        $T_\text{VZ}$.
    }
    \label{fig:resolution}
\end{figure}
