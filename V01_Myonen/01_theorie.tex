\section{Einleitung} % (fold)
\label{sec:einleitung}

Myonen gehören wie Elektronen zur Familie Leptonen und entstehen durch Pionen-Zerfall in den oberen Schichten der Atmosphäre.
In diesem Versuch wird die Lebensdauer dieser Myonen untersucht.

\section{Theorie} % (fold)
\label{sec:theorie}

\subsection{Eigenschaften von Myonen} % (fold)
\label{sub:eigenschaften_von_myonen}

% subsection eigenschaften_von_myonen (end)

Myonen sind 206,77 mal schwerer als ein Elektronen und gehören zur zweiten Generation der Leptonenfamilie.
Zudem unterliegen diese nur der Schwachen und Elektromagnetischen Wechselwirkung und besitzen im gegensatz zu Elektronen eine endliche Lebensdauer.

\subsection{Definition der Lebensdauer} % (fold)
\label{sub:definition_der_lebensdauer}

Der Zerfall von instabilen Teilchen ist ein statistischer Prozess, weshalb jedes Teilchen eine indivduelle Lebensdauer besitzt.
Eine allgemeine Definition der Lebensdauer ist über 

\begin{equation}
	\text{d}W = \lambda \text{d}t
\end{equation}

gegeben.

Bei dieser Definition ist die Wahrscheinlichkeit nicht explizit von $t$ abhängig, weshalb das Alter des Teilchens keine Rolle spielt.
Für $N$ zerfallende Teilchen ergiebt sich somit 

\begin{equation}
	\text{d}N = -N \text{d}W = -\lambda \text{d}t.
\end{equation}

Für große $N$ ergiebt sich näherungsweise für das Integral

\begin{equation}
	\frac{N(t)}{N_\text{0}} = \exp{-\lambda t},
\end{equation}

und somit für die Verteilung
 
\begin{equation}
	\frac{\text{d}N(t)}{N_\text{0}} = \lambda \exp{-\lambda t} \text{d}t .
\end{equation}
\\
Die charakteristische Lebensdauer $\tau$ wird über die Mittelwerte aller möglichen Lebensdauern gebildet, wobei diese mit der Häufigkeit ihres Vorkommens gewichtet sind.
Dies entspricht dem Ersten Moment der Verteilungsfunktion und es ergibt sich damit

\begin{equation}
	\tau = \int_\text{0}^\infty \lambda \exp{-\lambda t} \text{d}t = \frac{1}{\lambda} .
\end{equation}

\subsection{Abschätzung der Lebensdauer mit Hilfe einer Stichprobe} % (fold)
\label{sub:abschaetzung_der_lebensdauer_mit_hilfe_einer_stichprobe}
