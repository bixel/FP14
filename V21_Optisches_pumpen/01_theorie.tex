\section{Theoretische Grundlagen}
\label{sec:theoretische_grundlagen}
Atome können in verschiedene Energieniveaus angeregt werden. Dabei werden
Elektronen aus ihrem Grundzustand in einen Zustand höherer Energie gebracht.
Die niedrigsten niveaus werden dabei als S- und P-Schalten bezeichnet.  Das
Verhältnis der Besetzungszahlen $N_1$ und $N_2$ dieser Zustände ohne äußeres
Magnetfeld wird dabei durch die Boltzmannverteilung beschrieben:
\begin{equation}
\label{eq:boltzmann}
    \frac{N_2}{N_1} = \frac{g_2}{g_1}\cdot
        \mathrm{e}^\frac{W_1 - W_2}{k_\text{B}T}\,.
\end{equation}
Dabei bezeichnet $g_i$ und $W_i$ ein jeweiliges statistisches Gewicht,
beziehungsweise die jeweilige Energie $T$ die Temeperatur und $k_\text{B}$ die
bekannte Boltzmannkonstante.

Diese Verteilung kann durch optisches Pumpen beeinflusst werden.
Durch Einstrahlung von Licht der Wellenlänge
\begin{equation}
\label{eq:wellenlaenge}
    h\nu = W_2 - W_1
\end{equation}
können Übergänge vom niedrigeren Energieniveau $W_1$ ind das höhere niveau
$W_2$ angeregt werden. Außerdem können Elektronen spontan oder induziert in ein
niedrigeres Niveau fallen, wobei ein neues Quant der Energie
\ref{eq:wellenlaenge} abgestrahlt wird.

\subsection{Zeemanneffekt und Hyperfeinstruktur}
\label{subsec:zeemanneffekt_und_hyperfeinstruktur}
Die Energiestruktur von Atomen ist unter anderem durch den Gesamtdrehimpuls
$\vec{L}$ und Spin $\vec{S}$ der Elektronenhülle, sowie den Kernspin $\vec{I}$
bestimmt. Dabei sind an diese Drehimpulse die magnetischen Momente
\begin{align}
    \vec{\mu}_L &= -\mu_\text{B}\vec{L}\,,
        \label{eq:magn_moment_elektronen}\\
    \vec{\mu}_S &= -g_S\mu_\text{B}\vec{S}\,,
        \label{eq:magn_moment_espin}\\
    \text{und}\qquad\vec{\mu}_I &= -g_I\mu_\text{B}\vec{I}
        \label{eq:magn_moment_kernspin}
\end{align}
gekoppelt, wobei $g_I$ und $g_S$ die Landé-Faktoren bezeichnen.
