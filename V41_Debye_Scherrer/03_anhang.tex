\newpage
\section{Anhang} % (fold)
\label{sec:anhang}
\FloatBarrier

\begin{table}[!htbp]
    \centering
    \caption{Vergleich der Werte der gemessenen Probe 6 mit den Werten der BCC-Struktur, wenn von rechts nach links gemessen wird.}
    \input{build/tex/arrProbeDiffN_second_one.tex}
    \label{tab:probe_vergleich_2}
\end{table}
\begin{table}[!htbp]
    \centering
    \caption{Vergleich der Werte der gemessenen Probe 6 mit den Werten der SC-Struktur, wenn von rechts nach links gemessen wird.}
    \input{build/tex/arrProbeDiffN_third_one.tex}
    \label{tab:probe_vergleich_3}
\end{table}
\begin{table}[!htbp]
    \centering
    \caption{Vergleich der Werte der gemessenen Probe 6 mit den Werten der Diamant-Struktur, wenn von rechts nach links gemessen wird.}
    \input{build/tex/arrProbeDiffN_fourth_one.tex}
    \label{tab:probe_vergleich_4}
\end{table}
\begin{table}[!htbp]
    \centering
    \caption{Vergleich der Werte der gemessenen Probe 6 mit den Werten der SC-Struktur, wenn von links nach rechts gemessen wird.}
    \input{build/tex/arrProbeDiffInvN_second_one.tex}
    \label{tab:probe_vergleich_inv_2}
\end{table}
\begin{table}[!htbp]
    \centering
    \caption{Vergleich der Werte der gemessenen Probe 6 mit den Werten der BCC-Struktur, wenn von links nach rechts gemessen wird.}
    \input{build/tex/arrProbeDiffInvN_third_one.tex}
    \label{tab:probe_vergleich_inv_3}
\end{table}
\begin{table}[!htbp]
    \centering
    \caption{Vergleich der Werte der gemessenen Probe 6 mit den Werten der Diamant-Struktur, wenn von links nach rechts gemessen wird.}
    \input{build/tex/arrProbeDiffInvN_fourth_one.tex}
    \label{tab:probe_vergleich_inv_4}
\end{table}
\begin{table}[!htbp]
    \centering
    \caption{Werte der Probe 6 der BCC-Struktur, welche für die lineare Regression genutzt werden, wenn von rechts nach links gemessen wird.}
    \input{build/tex/arrProbeDiffN_second_two.tex}
    \label{tab:data_probe_fit_2}
\end{table}
\begin{table}[!htbp]
    \centering
    \caption{Werte der Probe 6 der SC-Struktur, welche für die lineare Regression genutzt werden, wenn von rechts nach links gemessen wird.}
    \input{build/tex/arrProbeDiffN_third_two.tex}
    \label{tab:data_probe_fit_3}
\end{table}
\begin{table}[!htbp]
    \centering
    \caption{Werte der Probe 6 der Diamant-Struktur, welche für die lineare Regression genutzt werden, wenn von rechts nach links gemessen wird.}
    \input{build/tex/arrProbeDiffN_fourth_two.tex}
    \label{tab:data_probe_fit_4}
\end{table}
\begin{table}[!htbp]
    \centering
    \caption{Werte der Probe 6 der SC-Struktur, welche für die lineare Regression genutzt werden, wenn von links nach rechts gemessen wird.}
    \input{build/tex/arrProbeDiffInvN_second_two.tex}
    \label{tab:data_probe_fit_inv_2}
\end{table}
\begin{table}[!htbp]
    \centering
    \caption{Werte der Probe 6 der BCC-Struktur, welche für die lineare Regression genutzt werden, wenn von links nach rechts gemessen wird.}
    \input{build/tex/arrProbeDiffInvN_third_two.tex}
    \label{tab:data_probe_fit_inv_3}
\end{table}
\begin{table}[!htbp]
    \centering
    \caption{Werte der Probe 6 der Diamant-Struktur, welche für die lineare Regression genutzt werden, wenn von links nach rechts gemessen wird.}
    \input{build/tex/arrProbeDiffInvN_fourth_two.tex}
    \label{tab:data_probe_fit_inv_4}
\end{table}
\begin{figure}
    \centering
    \includegraphics[width=1.0\linewidth]{build/pic/arrProbeDiffN_second.pdf}
    \caption{Errechnete Werte der Gitterkonstanten $a$ der Probe 6 mit einer BCC-Struktur und die dazugehörige Regressionsgerade, wenn von rechts nach links gemessen wird.}
    \label{fig:data_probe_fit_2}
\end{figure}
\begin{figure}
    \centering
    \includegraphics[width=1.0\linewidth]{build/pic/arrProbeDiffN_third.pdf}
    \caption{Errechnete Werte der Gitterkonstanten $a$ der Probe 6 mit einer SC-Struktur und die dazugehörige Regressionsgerade, wenn von rechts nach links gemessen wird.}
    \label{fig:data_probe_fit_3}
\end{figure}
\begin{figure}
    \centering
    \includegraphics[width=1.0\linewidth]{build/pic/arrProbeDiffN_fourth.pdf}
    \caption{Errechnete Werte der Gitterkonstanten $a$ der Probe 6 mit einer Diamant-Struktur und die dazugehörige Regressionsgerade, wenn von rechts nach links gemessen wird.}
    \label{fig:data_probe_fit_4}
\end{figure}
\clearpage
\begin{figure}
    \centering
    \includegraphics[width=1.0\linewidth]{build/pic/arrProbeDiffInvN_second.pdf}
    \caption{Errechnete Werte der Gitterkonstanten $a$ der Probe 6 mit einer SC-Struktur und die dazugehörige Regressionsgerade, wenn von links nach rechts gemessen wird.}
    \label{fig:data_probe_fit_inv_2}
\end{figure}
\begin{figure}
    \centering
    \includegraphics[width=1.0\linewidth]{build/pic/arrProbeDiffInvN_third.pdf}
    \caption{Errechnete Werte der Gitterkonstanten $a$ der Probe 6 mit einer BCC-Struktur und die dazugehörige Regressionsgerade, wenn von links nach rechts gemessen wird.}
    \label{fig:data_probe_fit_inv_3}
\end{figure}
\begin{figure}
    \centering
    \includegraphics[width=1.0\linewidth]{build/pic/arrProbeDiffInvN_fourth.pdf}
    \caption{Errechnete Werte der Gitterkonstanten $a$ der Probe 6 mit einer Diamant-Struktur und die dazugehörige Regressionsgerade, wenn von links nach rechts gemessen wird.}
    \label{fig:data_probe_fit_inv_4}
\end{figure}
\begin{table}[!htbp]
    \centering
    \caption{Vergleich der Werte des gemessenen Salzes 1 mit den Werten der Steinsalz-Struktur, wenn von rechts nach links gemessen wird.}
    \input{build/tex/arrSaltDiffN_second_one.tex}
    \label{tab:salt_vergleich_2}
\end{table}
\begin{table}[!htbp]
    \centering
    \caption{Vergleich der Werte des gemessenen Salzes 1 mit den Werten der Fluorid-Struktur, wenn von rechts nach links gemessen wird.}
    \input{build/tex/arrSaltDiffN_third_one.tex}
    \label{tab:salt_vergleich_3}
\end{table}
\begin{table}[!htbp]
    \centering
    \caption{Vergleich der Werte des gemessenen Salzes 1 mit den Werten der Zinkblende-Struktur, wenn von rechts nach links gemessen wird.}
    \input{build/tex/arrSaltDiffN_fourth_one.tex}
    \label{tab:salt_vergleich_4}
\end{table}
\begin{table}[!htbp]
    \centering
    \caption{Vergleich der Werte des gemessenen Salzes 1 mit den Werten der Steinsalz-Struktur, wenn von links nach rechts gemessen wird.}
    \input{build/tex/arrSaltDiffInvN_second_one.tex}
    \label{tab:salt_vergleich_inv_2}
\end{table}
\begin{table}[!htbp]
    \centering
    \caption{Vergleich der Werte des gemessenen Salzes 1 mit den Werten der Fluorid-Struktur, wenn von links nach rechts gemessen wird.}
    \input{build/tex/arrSaltDiffInvN_third_one.tex}
    \label{tab:salt_vergleich_inv_3}
\end{table}
\begin{table}[!htbp]
    \centering
    \caption{Vergleich der Werte des gemessenen Salzes 1 mit den Werten der Zinkblende-Struktur, wenn von links nach rechts gemessen wird.}
    \input{build/tex/arrSaltDiffInvN_fourth_one.tex}
    \label{tab:salt_vergleich_inv_4}
\end{table}
\begin{table}[!htbp]
    \centering
    \caption{Werte des Salzes 1 der Steinsalz-Struktur, welche für die lineare Regression genutzt werden, wenn von rechts nach links gemessen wird.}
    \input{build/tex/arrSaltDiffN_second_two.tex}
    \label{tab:data_salt_fit_2}
\end{table}
\begin{table}[!htbp]
    \centering
    \caption{Werte des Salzes 1 der Fluorid-Struktur, welche für die lineare Regression genutzt werden, wenn von rechts nach links gemessen wird.}
    \input{build/tex/arrSaltDiffN_third_two.tex}
    \label{tab:data_salt_fit_3}
\end{table}
\begin{table}[!htbp]
    \centering
    \caption{Werte des Salzes 1 der Zinkblende-Struktur, welche für die lineare Regression genutzt werden, wenn von rechts nach links gemessen wird.}
    \input{build/tex/arrSaltDiffN_fourth_two.tex}
    \label{tab:data_salt_fit_4}
\end{table}
\begin{table}[!htbp]
    \centering
    \caption{Werte des Salzes 1 der Steinsalz-Struktur, welche für die lineare Regression genutzt werden, wenn von links nach rechts gemessen wird.}
    \input{build/tex/arrSaltDiffInvN_second_two.tex}
    \label{tab:data_salt_fit_inv_2}
\end{table}
\begin{table}[!htbp]
    \centering
    \caption{Werte des Salzes 1 der Fluorid-Struktur, welche für die lineare Regression genutzt werden, wenn von links nach rechts gemessen wird.}
    \input{build/tex/arrSaltDiffInvN_third_two.tex}
    \label{tab:data_salt_fit_inv_3}
\end{table}
\begin{table}[!htbp]
    \centering
    \caption{Werte des Salzes 1 der Zinkblende-Struktur, welche für die lineare Regression genutzt werden, wenn von links nach rechts gemessen wird.}
    \input{build/tex/arrSaltDiffInvN_fourth_two.tex}
    \label{tab:data_salt_fit_inv_4}
\end{table}
\begin{figure}
    \centering
    \includegraphics[width=1.0\linewidth]{build/pic/arrSaltDiffN_second.pdf}
    \caption{Errechnete Werte der Gitterkonstanten $a$ des Salzes 1 mit einer Steinsalz-Struktur und die dazugehörige Regressionsgerade, wenn von rechts nach links gemessen wird.}
    \label{fig:data_salt_fit_2}
\end{figure}
\clearpage
\begin{figure}
    \centering
    \includegraphics[width=1.0\linewidth]{build/pic/arrSaltDiffN_third.pdf}
    \caption{Errechnete Werte der Gitterkonstanten $a$ des Salzes 1 mit einer Fluorid-Struktur und die dazugehörige Regressionsgerade, wenn von rechts nach links gemessen wird.}
    \label{fig:data_salt_fit_3}
\end{figure}
\begin{figure}
    \centering
    \includegraphics[width=1.0\linewidth]{build/pic/arrSaltDiffN_fourth.pdf}
    \caption{Errechnete Werte der Gitterkonstanten $a$ des Salzes 1 mit einer Zinkblende-Struktur und die dazugehörige Regressionsgerade, wenn von rechts nach links gemessen wird.}
    \label{fig:data_salt_fit_4}
\end{figure}
\begin{figure}
    \centering
    \includegraphics[width=1.0\linewidth]{build/pic/arrSaltDiffInvN_second.pdf}
    \caption{Errechnete Werte der Gitterkonstanten $a$ des Salzes 1 mit einer Steinsalz-Struktur und die dazugehörige Regressionsgerade, wenn von links nach rechts gemessen wird.}
    \label{fig:data_salt_fit_inv_2}
\end{figure}
\begin{figure}
    \centering
    \includegraphics[width=1.0\linewidth]{build/pic/arrSaltDiffInvN_third.pdf}
    \caption{Errechnete Werte der Gitterkonstanten $a$ des Salzes 1 mit einer Fluorid-Struktur und die dazugehörige Regressionsgerade, wenn von links nach rechts gemessen wird.}
    \label{fig:data_salt_fit_inv_3}
\end{figure}
\begin{figure}
    \centering
    \includegraphics[width=1.0\linewidth]{build/pic/arrSaltDiffInvN_fourth.pdf}
    \caption{Errechnete Werte der Gitterkonstanten $a$ des Salzes 1 mit einer Zinkblende-Struktur und die dazugehörige Regressionsgerade, wenn von links nach rechts gemessen wird.}
    \label{fig:data_salt_fit_inv_4}
\end{figure}
\begin{table}[!htbp]
    \centering
    \caption{Kombinationen der Werte $(h k l)$ für nicht verschwindende Reflexe der Steinsalz-Struktur bei Betrachtung verschiedener Formfaktoren.}
    \input{build/tex/Steinsalz_first.tex}
    \label{tab:struktur_hkl_stein}
\end{table}
\begin{table}[!htbp]
    \centering
    \caption{Kombinationen der Werte $(h k l)$ für nicht verschwindende Reflexe der Steinsalz-Struktur bei Betrachtung verschiedener Formfaktoren.}
    \input{build/tex/Steinsalz_second.tex}
    \label{tab:struktur_hkl_stein1}
\end{table}
\begin{table}[!htbp]
    \centering
    \caption{Kombinationen der Werte $(h k l)$ für nicht verschwindende Reflexe der Steinsalz-Struktur bei Betrachtung verschiedener Formfaktoren.}
    \input{build/tex/Steinsalz_third.tex}
    \label{tab:struktur_hkl_stein2}
\end{table}
\begin{table}[!htbp]
    \centering
    \caption{Kombinationen der Werte $(h k l)$ für nicht verschwindende Reflexe der Steinsalz-Struktur bei Betrachtung verschiedener Formfaktoren.}
    \input{build/tex/Steinsalz_fourth.tex}
    \label{tab:struktur_hkl_stein3}
\end{table}
\begin{table}[!htbp]
    \centering
    \caption{Kombinationen der Werte $(h k l)$ für nicht verschwindende Reflexe der Steinsalz-Struktur bei Betrachtung verschiedener Formfaktoren.}
    \input{build/tex/Steinsalz_fifth.tex}
    \label{tab:struktur_hkl_stein4}
\end{table}
\begin{table}[!htbp]
    \centering
    \caption{Kombinationen der Werte $(h k l)$ für nicht verschwindende Reflexe der Zinkblende-Struktur bei Betrachtung verschiedener Formfaktoren.}
    \input{build/tex/Zinkblende_first.tex}
    \label{tab:struktur_hkl_zink}
\end{table}
\begin{table}[!htbp]
    \centering
    \caption{Kombinationen der Werte $(h k l)$ für nicht verschwindende Reflexe der Zinkblende-Struktur bei Betrachtung verschiedener Formfaktoren.}
    \input{build/tex/Zinkblende_second.tex}
    \label{tab:struktur_hkl_zink1}
\end{table}
\begin{table}[!htbp]
    \centering
    \caption{Kombinationen der Werte $(h k l)$ für nicht verschwindende Reflexe der Zinkblende-Struktur bei Betrachtung verschiedener Formfaktoren.}
    \input{build/tex/Zinkblende_third.tex}
    \label{tab:struktur_hkl_zink2}
\end{table}
\begin{table}[!htbp]
    \centering
    \caption{Kombinationen der Werte $(h k l)$ für nicht verschwindende Reflexe der Zinkblende-Struktur bei Betrachtung verschiedener Formfaktoren.}
    \input{build/tex/Zinkblende_fourth.tex}
    \label{tab:struktur_hkl_zink3}
\end{table}
\begin{table}[!htbp]
    \centering
    \caption{Kombinationen der Werte $(h k l)$ für nicht verschwindende Reflexe der Zinkblende-Struktur bei Betrachtung verschiedener Formfaktoren.}
    \input{build/tex/Zinkblende_fifth.tex}
    \label{tab:struktur_hkl_zink4}
\end{table}
\clearpage
\begin{table}[!htbp]
    \centering
    \caption{Kombinationen der Werte $(h k l)$ für nicht verschwindende Reflexe der Fluorid-Struktur bei Betrachtung verschiedener Formfaktoren.}
    \input{build/tex/Fluorid_first.tex}
    \label{tab:struktur_hkl_fluor}
\end{table}
\begin{table}[!htbp]
    \centering
    \caption{Kombinationen der Werte $(h k l)$ für nicht verschwindende Reflexe der Fluorid-Struktur bei Betrachtung verschiedener Formfaktoren.}
    \input{build/tex/Fluorid_second.tex}
    \label{tab:struktur_hkl_fluor1}
\end{table}
\begin{table}[!htbp]
    \centering
    \caption{Kombinationen der Werte $(h k l)$ für nicht verschwindende Reflexe der Fluorid-Struktur bei Betrachtung verschiedener Formfaktoren.}
    \input{build/tex/Fluorid_third.tex}
    \label{tab:struktur_hkl_fluor2}
\end{table}
\begin{table}[!htbp]
    \centering
    \caption{Kombinationen der Werte $(h k l)$ für nicht verschwindende Reflexe der Fluorid-Struktur bei Betrachtung verschiedener Formfaktoren.}
    \input{build/tex/Fluorid_fourth.tex}
    \label{tab:struktur_hkl_fluor3}
\end{table}
\begin{table}[!htbp]
    \centering
    \caption{Kombinationen der Werte $(h k l)$ für nicht verschwindende Reflexe der Fluorid-Struktur bei Betrachtung verschiedener Formfaktoren.}
    \input{build/tex/Fluorid_fifth.tex}
    \label{tab:struktur_hkl_fluor4}
\end{table}
\begin{table}[!htbp]
    \centering
    \caption{Kombinationen der Werte $(h k l)$ für nicht verschwindende Reflexe der Caesium-Struktur bei Betrachtung verschiedener Formfaktoren.}
    \input{build/tex/Caesiumstruktur_first.tex}
    \label{tab:struktur_hkl_caesium}
\end{table}
\begin{table}[!htbp]
    \centering
    \caption{Kombinationen der Werte $(h k l)$ für nicht verschwindende Reflexe der Caesium-Struktur bei Betrachtung verschiedener Formfaktoren.}
    \input{build/tex/Caesiumstruktur_second.tex}
    \label{tab:struktur_hkl_caesium1}
\end{table}
\begin{table}[!htbp]
    \centering
    \caption{Kombinationen der Werte $(h k l)$ für nicht verschwindende Reflexe der Caesium-Struktur bei Betrachtung verschiedener Formfaktoren.}
    \input{build/tex/Caesiumstruktur_third.tex}
    \label{tab:struktur_hkl_caesium2}
\end{table}
\begin{table}[!htbp]
    \centering
    \caption{Kombinationen der Werte $(h k l)$ für nicht verschwindende Reflexe der Caesium-Struktur bei Betrachtung verschiedener Formfaktoren.}
    \input{build/tex/Caesiumstruktur_fourth.tex}
    \label{tab:struktur_hkl_caesium3}
\end{table}
\begin{table}[!htbp]
    \centering
    \caption{Kombinationen der Werte $(h k l)$ für nicht verschwindende Reflexe der Caesium-Struktur bei Betrachtung verschiedener Formfaktoren.}
    \input{build/tex/Caesiumstruktur_fifth.tex}
    \label{tab:struktur_hkl_caesium4}
\end{table}
