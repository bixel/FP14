\section{Messung} % (fold)
\label{sec:messung}

Zunächst wird berechnet, welche Kombinationen der Millerschen Indizes $h$, $k$ und $l$ bei den betrachteten Kristallstrukturen möglich sind.
Für diese wird anschließend der Wert $m_i = h_i^2 + k_i^2 + l_i^2$ bestimmt und die Wurzel aus dem Quotienten $m_i / m_1$ berechnet.
Anschließend werden diese Werte mit den Gemessenen verglichen und so die Wahrscheinlichsten Kristallstrukturen bestimmt.

Abschließend können die Gitterkonstanten der Strukturen berechnet werden.

Zur Berechnung der Fehler wurde die uncertainties Bibliothek in python genutzt, wodurch der Fehler nach
\begin{equation*}
    \Delta f = \sqrt{\sum_{i=1}^N \left( \left(\frac{\partial f}{\partial x_i}\right)^2  \Delta f_{x_i}^2 \right)}
\end{equation*}
berechnet wird \cite{py-uncertainties}.
Dabei wurde ein Fehler auf die gemessene Länge $s$ von
\begin{equation*}
    \Delta s = \SI{0.01}{\milli\meter}
\end{equation*}
angenommen.
Sind bei Werten keine Fehler angegeben, so haben diese keine Aussagekraft, da der Wert mithilfe der durchgeführten Messung nicht so genau bestimmt werden kann.

\subsection{Mögliche Werte für $h$, $k$ und $l$ verschiedener Kristallstrukturen} % (fold)
\label{sub:mögliche_werte_für_h_k_und_l_verschiedener_kristallstrukturen}

Zur Berechnung der nicht verschwindenden Netzebenen wird der Strukturfaktor nach
\begin{equation*}
    S = \sum_j{f_j \exp{(-2 \pi i(x_j \cdot h + y_j \cdot k + z_j \cdot l))}}\,,
\end{equation*}
und den fundamentalen Translationen aus Kapitel \ref{sub:kubische_kristallstrukturen} berechnet.
Es wurden die Kristallgitter SC-Strukur, FCC-Struktur, BCC-Struktur, Diamant-Struktur, Zinkblende-Struktur, Steinsalz-Struktur, Fluorid-Struktur und Caesiumchlorid-Struktur betrachtet.
Die nicht verschwindenden Kombinationen der Werte $h$, $k$ und $l$ sind in Tabelle \ref{tab:struktur_hkl} aufgelistet.
\begin{table}[!ht]
    \centering
    \caption{Kombinationen der Werte $(h k l)$ für nicht verschwindende Reflexe verschiedener Kristallgitter. Für die Steinsalz-Struktur wurde eine Basis aus Chlor auf der fundamentalen Translation A und Calcium auf der fundamentalen Translation B und für die Caesiumchlorid-Struktur eine Basis aus Caesium auf der fundamentalen Translation A und Chlot auf der fundamentalen Translation B.}
    \input{build/tex/hkl.tex}
    \label{tab:struktur_hkl}
\end{table}
\begin{table}[!ht]
    \centering
    \caption{Verwendete Formfaktoren zur Berechnung der nicht verschwindenden Reflexe der Werte $(h, k, l)$ \cite{V41}.}
    \input{build/tex/formfaktoren.tex}
    \label{tab:formfaktoren}
\end{table}
Bei einer weiteren Betrachtung der Formfaktoren aus Tabelle \ref{tab:formfaktoren} ergeben sich zusätzlich die in den Tabellen \ref{tab:struktur_hkl_stein} bis \ref{tab:struktur_hkl_caesium4} dargestellten Reflexe.


\subsection{Messdaten} % (fold)
\label{sub:messdaten}

Um die Gitterstrukturen der gemessenen Probe 6 und dem Salz 1 zu bestimmen, werden zunächst die Streuwinkel aus den Ringen der Filmstreifen nach
\begin{equation*}
    \Theta' = \frac{s}{R}\,,
\end{equation*}
mit $R = \SI{57.3}{\milli\meter}$, bestimmt.

Da die statistische Verteilung der Kristallorientierungen auf einen Kegelmantel mit dem Öffnungswinkel $2\Theta$ abbildet ergibt sich für $\Theta$
\begin{equation*}
    \Theta = \frac{\Theta'}{2}\,,
\end{equation*}
woraus sich der Netzebenenabstand nach der Bragg'schen Bedingung
\begin{equation*}
    d = \frac{\lambda}{2\sin{(\Theta)}}\,,
\end{equation*}
mit $\lambda = \SI{1.5417}{\angstrom}$, ergibt.
Diese ist ein Mittelwert aus der $\lambda_{K_{\alpha,1}} = \SI{1.54093}{\angstrom}$ und $\lambda_{K_{\alpha,2}} = \SI{1.54478}{\angstrom}$ Wellenlänge der Röntgenstrahlung.
Für die letzten vier Werte ist diese Näherung jedoch nicht gültig, da es zu einer Aufspaltung der Ringe gekommen ist.
Somit wurde für diese Werte der Mittelwert nach
\begin{equation*}
    s_\text{Mittel} = \frac{s_{i} + s_{i+1}}{2}
\end{equation*}
gebildet.
Zudem ist es zu einer Aufspaltung der ersten vier Werte gekommen, aufgrund des nicht perfekten herausfilterns der $K_{\beta}$ Linie.
Auch für diese ist der Mittelwert errechnet worden.

Die errechneten Ergebnisse sind in den Tabellen \ref{tab:data_1} bis \ref{tab:data_4} dargestellt.
Da im Nachhinein keine Zuordnung von vorne und hinten auf dem Filmstreifen möglich ist,
werden diese sowohl von rechts nach links ausgemessen,
als auch von links nach rechts ausgemessen ausgewertet.

\begin{table}[!ht]
    \centering
    \caption{Gemessene und errechnete Werte der gemessenen Probe 6, wenn von rechts nach links gemessen wird.}
    \input{build/tex/data_probe.tex}
    \label{tab:data_1}
\end{table}
\begin{table}[!ht]
    \centering
    \caption{Gemessene und errechnete Werte der gemessenen Probe 6, wenn von links nach rechts gemessen wird.}
    \input{build/tex/data_probe_inv.tex}
    \label{tab:data_2}
\end{table}
\begin{table}[!ht]
    \centering
    \caption{Gemessene und errechnete Werte des gemessenen Salz 1, wenn von rechts nach links gemessen wird.}
    \input{build/tex/data_salt.tex}
    \label{tab:data_3}
\end{table}
\begin{table}[!ht]
    \centering
    \caption{Gemessene und errechnete Werte des gemessenen Salz 1, wenn von links nach rechts gemessen wird.}
    \input{build/tex/data_salt_inv.tex}
    \label{tab:data_4}
\end{table}
\clearpage


\subsection{Analyse der Probe 6} % (fold)
\label{sub:analyse_der_probe_6}

\subsubsection{Bestimmung der Kristallstruktur} % (fold)
\label{ssub:bestimmung_der_kristallstruktur_probe}

Da es sich bei der Probe 6 um ein Metall handeln soll,
kommen für die Kristallgitter SC-Struktur, BCC-Struktur, FCC-Struktur und Diamant-Struktur infrage.

Da sich für die Gitterkonstante die Beziehung
\begin{equation*}
    a = d \cdot \sqrt{h^2 + k^2 + l^2} = d \cdot \sqrt{m} = d_i \cdot \sqrt{m_i}\,,
\end{equation*}
ergibt,
lassen sich die gemessenen Netzebenenabstände $d_i$ mit den Kombinationen von $(h k l)$
der möglichen Kristallgitter über die Beziehung
\begin{equation*}
    \frac{d_1}{d_i} = \sqrt{\frac{m_i}{m_1}}\,,
\end{equation*}
vergleichen.
Dabei ergeben sich die in den Tabellen \ref{tab:probe_vergleich} und \ref{tab:probe_vergleich_inv}
dargestellten Abweichungen.
Eine genauere Betrachtung der wahrscheinlichsten Kristallgitter ist in den Tabellen \ref{tab:probe_vergleich_1} und \ref{tab:probe_vergleich_inv_1} dargestellt.

Zum Vergleich sind die unwahrscheinlicheren Kristallgitter in den Tabellen \ref{tab:probe_vergleich_2} bis \ref{tab:probe_vergleich_inv_4} abgebildet.

\begin{table}[!ht]
    \centering
    \caption{Mittlere Abweichung der verglichenen Werte der gemessenen Probe 6, wenn von rechts nach links gemessen wird.}
    \input{build/tex/arrProbeDiffN.tex}
    \label{tab:probe_vergleich}
\end{table}
\begin{table}[!ht]
    \centering
    \caption{Vergleich der Werte der gemessenen Probe 6 mit den Werten der FCC-Struktur, wenn von rechts nach links gemessen wird.}
    \input{build/tex/arrProbeDiffN_first_one.tex}
    \label{tab:probe_vergleich_1}
\end{table}
\begin{table}[!ht]
    \centering
    \caption{Mittlere Abweichung der verglichenen Werte der gemessenen Probe 6, wenn von links nach rechts gemessen wird.}
    \input{build/tex/arrProbeDiffInvN.tex}
    \label{tab:probe_vergleich_inv}
\end{table}
\begin{table}[!ht]
    \centering
    \caption{Vergleich der Werte der gemessenen Probe 6 mit den Werten der FCC-Struktur, wenn von links nach rechts gemessen wird.}
    \input{build/tex/arrProbeDiffInvN_first_one.tex}
    \label{tab:probe_vergleich_inv_1}
\end{table}

\clearpage
\subsubsection{Bestimmung der Gitterkonstanten $a$} % (fold)
\label{sub:bestimmung_der_gitterkonstanten_probe}

Zur Bestimmung der Gitterkonstanten wird der Zusammenhang
\begin{equation*}
    a = d_i \cdot \sqrt{m_i}\,,
\end{equation*}
verwendet.
Um die scheinbare Abhängigkeit zwischen der Gitterkonstanten $a$ und dem Winkel $\Theta$ auszugleichen,
werden die errechnete Konstanten gegen $\cos^2{(\Theta)}$ aufgetragen und mithilfe einer linearen Regression der Form
$f(x) = c_\text{Probe} \cdot x + b$ die Konstante $a$ über die Beziehung
\begin{equation*}
    a = b\,,
\end{equation*}
bestimmt.

Die verwendeten Werte für die Regression der wahrscheinlichsten Kristallstruktur sind in den Tabellen \ref{tab:data_probe_fit_1} und \ref{tab:data_probe_fit_inv_1} dargestellt.
Die zugehörigen Graphen sind in den Abbildungen \ref{fig:data_probe_fit_1} und \ref{fig:data_probe_fit_inv_1} abgebildet.
Die Ergebnisse der Regression zeigen die Tabellen \ref{tab:data_probe_fit_erg_1} und \ref{tab:data_probe_fit_erg_inv_1}.
Die Werte der unwahrscheinlicheren Kristallstrukturen zeigen die Tabellen \ref{tab:data_probe_fit_2} bis \ref{tab:data_probe_fit_inv_4}.

\begin{table}[!ht]
    \centering
    \caption{Werte der Probe 6, welche für die lineare Regression genutzt werden, wenn von rechts nach links gemessen wird.}
    \input{build/tex/arrProbeDiffN_first_two.tex}
    \label{tab:data_probe_fit_1}
\end{table}
\begin{figure}[!ht]
    \centering
    \includegraphics[width=1\linewidth]{build/pic/arrProbeDiffN_first.pdf}
    \caption{Errechnete Werte der Gitterkonstanten $a$ der Probe 6 mit einer FCC-Struktur und die dazugehörige Regressionsgerade, wenn von rechts nach links gemessen wird.}
    \label{fig:data_probe_fit_1}
\end{figure}
\begin{table}[!ht]
    \centering
    \caption{Ergebnisse der linearen Regression der Probe 6, wenn von rechts nach links gemessen wird.}
    \input{build/tex/arrProbeDiffN_fit.tex}
    \label{tab:data_probe_fit_erg_1}
\end{table}
\begin{table}[!ht]
    \centering
    \caption{Werte der Probe 6, welche für die lineare Regression genutzt werden, wenn von links nach rechts gemessen wird.}
    \input{build/tex/arrProbeDiffInvN_first_two.tex}
    \label{tab:data_probe_fit_inv_1}
\end{table}
\begin{figure}[!ht]
    \centering
    \includegraphics[width=1\linewidth]{build/pic/arrProbeDiffInvN_first.pdf}
    \caption{Errechnete Werte der Gitterkonstanten $a$ der Probe 6 mit einer FCC-Struktur und die dazugehörige Regressionsgerade, wenn von links nach rechts gemessen wird.}
    \label{fig:data_probe_fit_inv_1}
\end{figure}
\begin{table}[!ht]
    \centering
    \caption{Ergebnisse der linearen Regression der Probe 6, wenn von links nach rechts gemessen wird.}
    \input{build/tex/arrProbeDiffInvN_fit.tex}
    \label{tab:data_probe_fit_erg_inv_1}
\end{table}

\clearpage
\subsection{Analyse des Salzes 1} % (fold)
\label{sub:analyse_des_salzes_1}

\subsubsection{Bestimmung der Kristallstruktur} % (fold)
\label{ssub:bestimmung_der_kristallstruktur_salz}


Da es sich bei dem Salz 1 um ein Salz handeln soll,
kommen für die Kristallgitter Zinkblende-Struktur, Steinsalz-Struktur, Fluorid-Struktur und Caesiumchlorid-Struktur infrage.

Wie bei in Abschnitt \ref{sub:analyse_der_probe_6} bereits beschrieben werden die Kristallgitter über die Beziehung
\begin{equation*}
    \frac{d_1}{d_i} = \sqrt{\frac{m_i}{m_1}}\,,
\end{equation*}
verglichen.
Dabei ergaben sich die in den Tabellen \ref{tab:salt_vergleich} und \ref{tab:salt_vergleich_inv}
dargestellten Abweichungen der wahrscheinlichsten Kristallstrukturen.
Eine genauere Betrachtung der wahrscheinlichsten Kristallgitter ist in den Tabellen \ref{tab:salt_vergleich_1} bis \ref{tab:salt_vergleich_inv_1} dargestellt.

Zum Vergleich sind die unwahrscheinlicheren Kristallgitter in den Tabellen \ref{tab:salt_vergleich_2} bis \ref{tab:salt_vergleich_inv_4} abgebildet.

\begin{table}[!ht]
    \centering
    \caption{Mittlere Abweichung der verglichenen Werte des gemessenenen Salzes 1, wenn von rechts nach links gemessen wird.}
    \input{build/tex/arrSaltDiffN.tex}
    \label{tab:salt_vergleich}
\end{table}
\begin{table}[!ht]
    \centering
    \caption{Vergleich der Werte des gemessenen Salzes 1 mit den Werten der Caesiumchlorid-Struktur, wenn von rechts nach links gemessen wird. Dabei sitzen auf den Positionen A der fundamentalen Translation Caesium Atome und auf den Positionen B der fundamentalen Translation Chlor Atome.}
    \input{build/tex/arrSaltDiffN_first_one.tex}
    \label{tab:salt_vergleich_1}
\end{table}
\begin{table}[!ht]
    \centering
    \caption{Mittlere Abweichung der verglichenen Werte des gemessenenen Salzes 1, wenn von links nach rechts gemessen wird.}
    \input{build/tex/arrSaltDiffInvN.tex}
    \label{tab:salt_vergleich_inv}
\end{table}
\begin{table}[!ht]
    \centering
    \caption{Vergleich der Werte des gemessenen Salzes 1 mit den Werten der Caesiumchlorid-Struktur, wenn von links nach rechts gemessen wird. Dabei sitzen auf den Positionen A der fundamentalen Translation Caesium Atome und auf den Positionen B der fundamentalen Translation Chlor Atome.}
    \input{build/tex/arrSaltDiffInvN_first_one.tex}
    \label{tab:salt_vergleich_inv_1}
\end{table}

\clearpage
\subsubsection{Bestimmung der Gitterkonstanten $a$} % (fold)
\label{sub:bestimmung_der_gitterkonstanten_salt}

Um die scheinbare Abhängigkeit zwischen der Gitterkonstanten $a$ und dem Winkel $\Theta$ auszugleichen, werden,
wie in Abschnitt \ref{sub:bestimmung_der_gitterkonstanten_probe} beschrieben,
die errechneten Konstanten gegen $\cos^2{(\Theta)}$ aufgetragen und mithilfe einer linearen Regression der Form
$f(x) = c_\text{Salz} \cdot x + b$ die Konstante $a$ über die Beziehung
\begin{equation*}
    a = b\,,
\end{equation*}
bestimmt.

Die verwendeten Werte für die Regression der wahrscheinlichsten Kristallstruktur sind in den Tabellen \ref{tab:data_salt_fit_1} bis \ref{tab:data_salt_fit_inv_1} dargestellt.
Die zugehörigen Graphen sind in den Abbildungen \ref{fig:data_salt_fit_1} bis \ref{fig:data_salt_fit_inv_1} abgebildet.
Die Ergebnisse der Regression zeigen die Tabellen \ref{tab:data_salt_fit_erg_1} bis \ref{tab:data_salt_fit_erg_inv_1}.
Die Werte der unwahrscheinlicheren Kristallstrukturen zeigen die Tabellen \ref{tab:data_salt_fit_2} bis \ref{tab:data_salt_fit_inv_4}.

\begin{table}[!ht]
    \centering
    \caption{Werte des Salzes 1 der Caesiumchlorid-Struktur, welche für die lineare Regression genutzt werden, wenn von rechts nach links gemessen wird. Dabei sitzen auf den Positionen A der fundamentalen Translation Caesium Atome und auf den Positionen B der fundamentalen Translation Chlor Atome.}
    \input{build/tex/arrSaltDiffN_first_two.tex}
    \label{tab:data_salt_fit_1}
\end{table}
\begin{figure}[!ht]
    \centering
    \includegraphics[width=1\linewidth]{build/pic/arrSaltDiffN_first.pdf}
    \caption{Errechnete Werte der Gitterkonstanten $a$ des Salzes 1 mit einer Caesiumchlorid-Struktur und die dazugehörige Regressionsgerade, wenn von rechts nach links gemessen wird. Dabei sitzen auf den Positionen A der fundamentalen Translation Caesium Atome und auf den Positionen B der fundamentalen Translation Chlor Atome.}
    \label{fig:data_salt_fit_1}
\end{figure}
\begin{table}[!ht]
    \centering
    \caption{Ergebnisse der linearen Regression des Salzes 1 der Caesiumchlorid-Struktur, wenn von rechts nach links gemessen wird. Dabei sitzen auf den Positionen A der fundamentalen Translation Caesium Atome und auf den Positionen B der fundamentalen Translation Chlor Atome.}
    \input{build/tex/arrSaltDiffN_fit.tex}
    \label{tab:data_salt_fit_erg_1}
\end{table}
\begin{table}[!ht]
    \centering
    \caption{Werte des Salzes 1 der Caesiumchlorid-Struktur, welche für die lineare Regression genutzt werden, wenn von links nach rechts gemessen wird. Dabei sitzen auf den Positionen A der fundamentalen Translation Caesium Atome und auf den Positionen B der fundamentalen Translation Chlor Atome.}
    \input{build/tex/arrSaltDiffInvN_first_two.tex}
    \label{tab:data_salt_fit_inv_2}
\end{table}
\begin{figure}[!ht]
    \centering
    \includegraphics[width=1\linewidth]{build/pic/arrSaltDiffInvN_first.pdf}
    \caption{Errechnete Werte der Gitterkonstanten $a$ des Salzes 1 mit einer Caesiumchlorid-Struktur und die dazugehörige Regressionsgerade, wenn von links nach rechts gemessen wird. Dabei sitzen auf den Positionen A der fundamentalen Translation Caesium Atome und auf den Positionen B der fundamentalen Translation Chlor Atome.}
    \label{fig:data_salt_fit_inv_1}
\end{figure}
\begin{table}[!ht]
    \centering
    \caption{Ergebnisse der linearen Regression des Salzes 1 der Caesiumchlorid-Struktur, wenn von links nach rechts gemessen wird. Dabei sitzen auf den Positionen A der fundamentalen Translation Caesium Atome und auf den Positionen B der fundamentalen Translation Chlor Atome.}
    \input{build/tex/arrSaltDiffInvN_fit.tex}
    \label{tab:data_salt_fit_erg_inv_1}
\end{table}

\clearpage
\section{Diskussion} % (fold)
\label{sec:diskussion}

Mithilfe des Versuchs ließ sich die Kristallstruktur der gemessenen Probe 6 und dem Salz 1 auf eine wahrscheinlichste Kristallstruktur einschränken.
Aufgrund der geringeren Abweichung der Werte ist es wahrscheinlicher, dass bei der Probe 6 die von rechts nach links und bei dem Salz 1 die von rechts nach links gemessenen Streifen die richtige Ableserichtung darstellen.

Für die Probe 6 ergibt sich für das wahrscheinlichste Kristallgitter eine FCC-Struktur mit einer mittleren Abweichung von $\SI{0.92 \pm 0.02}{\percent}$.
Obwohl die sichtbaren Ringe sehr deutlich sind, fehlen einige Reflexe der Kombinationen von $h$, $k$ und $l$ für große $m$, welche sich durch eine längere Belichtung auf dem Filmstreifen zeigen könnten.
Weiterhin kam es zu einer Aufspaltung in den letzten beiden und den ersten beiden Ringen.
Dies ist auf die verschiedenen Wellenlängen der $K_{\alpha,1}$, $K_{\alpha,2}$ und $K_{\beta}$ Linie der Kupfer Anode zurückzuführen.
Dies musste durch Mittelung der jeweils zusammengehörigen Werte ausgeglichen werden.
Für die Gitterkonstante der FCC-Struktur ergibt sich mithilfe einer linearen Regression der Wert $\SI{3.05 \pm 0.02}{\angstrom}$.

Beim gemessenen Salz 1 ist es hingegen am wahrscheinlichsten, dass es sich um eine Caesiumchlorid-Struktur handelt bei einer mittleren Abweichung von $\SI{3.18 \pm 0.04}{\percent}$.
Die Ringe waren auf dem Streifen schwach bis deutlich zu erkennen, doch fehlten hierbei die Reflexe für größere $m$.
Um die Reflexe zu verdeutlichen und die fehlenden Reflexe sichtbar zu machen, könnte eine längere Belichtungszeit gewählt werden.
Die Gitterkonstante der Caesiumchlorid-Struktur ergibt sich somit mithilfe einer linearen Regression zu $\SI{4.13}{\angstrom}$.
Der Fehler ist nicht signifikant.
Zudem erbringt eine Betrachtung verschiedener Formfaktoren keine Änderung im Ergebnis

Abweichungen der Quotienten von den theoretischen Werten könnten sich neben den Ablesefehlern durch Inhomogenitäten oder Fehler innerhalb der Elementarzellen ergeben haben.
Aufgrund der gleichen Reflexe bei der Betrachtung verschiedener Formfaktoren innerhalb der Basis, ist es nicht möglich, auf die Atome, aus denen das Gitter aufgebaut ist, zurückzuschließen.
Da jedoch Aluminium, Kupfer, Nickel, Blei, Gold, Platin, Silber, Rhodium und Palladium in natürlicher Weise in einer FCC-Struktur vorkommen, ist anzunehmen, dass sich eines dieser Stoffe hinter der Probe 6 verbirgt.
Bei dem Salz 1 hingegen ist es wahrscheinlich dass die Caesiumchlroid-Strukur aus Caesium und Chlor aufgebaut ist, da dieses für diese Elemente charakteristisch ist.
