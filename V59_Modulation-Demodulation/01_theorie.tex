\section{Theoretische Grundlagen}
\label{sec:theorie}

Dieser Versuch behandelt die Modulation und Demodlation von elektromagnetischen Wellen, um mit Ihnen Informationen von einem Ort
zum anderen übertragen zu können.
Es wird dabei mehr auf die Amplituden- und Frequenzmodulation als Modulationstechnik eingegangen als auf die Phasenmodulation.
Da aus dem modulierten Signal die Informationen wieder rekonstruiert werden müssen, werden auch Demodulationsverfahren angewendet.

\subsection{Amplitudenmodulation}
\label{subsec:klassisch}

Wird die einfachste Form der Amplitudenmodulation betrachtet, so führt die Amplitude einer Hochfrequenten Trägerschwingung $U_\text{T}(t)$ Schwankungen im Rhythmus eines niederfrequenten Modulationssignals $U_\text{M}(t)$ aus.
Mit den dazugehörigen Kreisfrequenzen $\omega_\text{T}$ und $\omega_\text{M}$ und Amplituden $U_\text{0,T}$ und $U_\text{0,M}$ ergibt sich für die Schwingungen

\begin{eqnarray*}
    U_\text{T}(t) &=& U_\text{0,T} \cos{\omega_\text{T} t} \\
    U_\text{M}(t) &=& U_\text{0,M} \cos{\omega_\text{M} t}
\end{eqnarray*}

\subsection{Frequenzmodulation}
\label{subsec:einstein}


\subsection{Modulationsschaltungen}
\label{subsec:debye}

\subsection{Demodulator - Schaltungen}
\label{subsec:debye}

\section{Durchführung}
\label{sec:durchführung}
