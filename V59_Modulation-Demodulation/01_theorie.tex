\section{Theoretische Grundlagen}
\label{sec:theorie}

Dieser Versuch behandelt die Modulation und Demodlation von elektromagnetischen Wellen, um mit Ihnen Informationen von einem Ort
zum anderen übertragen zu können.
Es wird dabei mehr auf die Amplituden- und Frequenzmodulation als Modulationstechnik eingegangen als auf die Phasenmodulation.
Da aus dem modulierten Signal die Informationen wieder rekonstruiert werden müssen, werden auch Demodulationsverfahren angewendet.

\subsection{Amplitudenmodulation}
\label{subsec:klassisch}

Wird die einfachste Form der Amplitudenmodulation betrachtet, so führt die Amplitude einer Hochfrequenten Trägerschwingung $U_\text{T}(t)$ Schwankungen im Rhythmus eines niederfrequenten Modulationssignals $U_\text{M}(t)$ aus.
Mit den dazugehörigen Kreisfrequenzen $\omega_\text{T}$ und $\omega_\text{M}$ und Amplituden $U_\text{0,T}$ und $U_\text{0,M}$ ergibt sich für die Schwingungen
\begin{eqnarray*}
    U_\text{T}(t) &=& U_\text{0,T} \cos{\omega_\text{T} t}\, , \\
    U_\text{M}(t) &=& U_\text{0,M} \cos{\omega_\text{M} t}\, .
\end{eqnarray*}
Damit ergibt sich die amplitudenmodulierte Schwingung 
\begin{equation}
    U_\text{mod}(t) = U_\text{0,T} ( 1 + m \cos{\omega_\text{M} t} ) \cos{\omega_\text{T} t} \label{am:formel1}
\end{equation}
mit dem Modulationsgrad
\begin{equation}
    m = \gamma U_\text{0,M}
\end{equation}
welche einen Wert zwischen 0 und 1 annehmen kann.
Zudem liegt die Amplitude der Schwingung somit zwischen den Extremwerten
\begin{equation}
    U_\text{0,T} (1 \pm m)\,.
\end{equation}
Eine schematische Darstellung der Schwingung ist in Abbildung \ref{am:formel2} dargestellt.

\begin{figure}[!h]
    \centering
    \includegraphics[width=14cm]{images/am-amplitude.png}
    \caption{Zeitabhängigkeit der Momentanspannung $U_\text{mod}(t)$.}
    \label{am:amplitude}
\end{figure}

Mithilfe trigonometrischer Umformulierung ergibt sich aus Gleichung \eqref{am:formel1} 
\begin{equation}
     U_\text{mod}(t) = U_\text{0,T} \left( \cos{\omega_\text{T} t} + \frac{m}{2}\cos{(\omega_\text{T} + \omega_\text{M}) t} + \frac{m}{2}\cos{(\omega_\text{T} - \omega_\text{M}) t}\right) \,. \label{am:formel2}
\end{equation}

Daraus ist zu erkennen, dass das Frequenzspektrum einer in dieser Art und Weise modelierten Schwingung aus 3 Linien mit den Kreisfrequenzen $\omega_\text{T} + \omega_\text{M}$, $\omega_\text{T} - \omega_\text{M}$ und $\omega_\text{T}$ besteht.
$\omega_\text{T}$ überträgt dabei keinerlei Informationen, da sie nicht von der Modulationsamplitude $U_\text{0,M}$ abhängt.
In der Praxis wird daher versucht diese zu unterdrücken, da diese mit einem unnötigen Energieverbrauch verbunden ist.
Da die gesamte Information bereits in einem Seitenband steckt - bei mehreren Frequenzen in der Modulationsspannung verbreitern sich die äußeren Linien zu Frequenzbändern - kann mithilfe eines geeigneten Bandpassfilters bereits im Modulationsprozess eine Seite unterdrückt werden.

Nachteile der Amplitudenmodulation sind die geringe Störsicherheit und die geringe Verzerrungsfreiheit.

\subsection{Frequenzmodulation}
\label{subsec:einstein}

Eine frequenzmodulierte Schwingung kann durch Gleichung
\begin{equation*}
    U(t) = U_\text{0} \sin{\left( \omega_\text{T} t + m \frac{\omega_\text{T}}{\omega_\text{M}} \cos{\omega_\text{M} t} \right)}
\end{equation*}
dargestellt werden.
Die Momentanfrequenz $f$ ergibt sich durch Ableiten des Arguments der Sinusfunktion zu
\begin{equation*}
    f(t) = \frac{\omega_\text{T}}{2 \pi} ( 1 - m \sin{\omega_\text{M} t}) \,.
\end{equation*}
Der Vorfaktor der Sinusfunktion $\frac{m \omega_\text{T}}{2 \pi}$ wird als Frequenzhub bezeichnet und gibt die Variationsbreite der Schwingungsfrequenz an.
Im folgenden soll nur der Fall der Schmalband-Frequenzmodulation (niedriger Frequenzhub) betrachtet werden, also 
\begin{equation*}
    m \frac{\omega_\text{T}}{\omega_\text{M}} \ll 1\,.
\end{equation*}
Ein Beispiel für eine frequenzmodulierte Schwingung ist in Abbildung \ref{fm:modulation} dargestellt.

\begin{figure}[!h]
    \centering
    \includegraphics[width = 14cm]{images/fm-modulation.png}
    \caption{Zeitlicher Verlauf einer sinusförmig frequenzmodulierten Schwingung}
    \label{fm:modulation}
\end{figure}

Auch hier ist das Frequenzspektrum recht einfach was an der umgeformten Gleichung 
\begin{equation*}
    U(t) = U_\text{0} \left( \sin{\omega_\text{T} t}\cos{\left(m\frac{\omega_\text{T}}{\omega_\text{M}} \cos{\omega_\text{M} t}\right)} + \cos{\omega_\text{T} t}\sin{\left(m\frac{\omega_\text{T}}{\omega_\text{M}} \cos{\omega_\text{M} t}\right)} \right)\,.
\end{equation*}

Für eine schwach frequenzmodulierte Schwingung ergibt sich somit näherungsweise
\begin{equation*}
    U(t) \approx U_\text{0} \left( \sin{\omega_\text{T} t} + \frac{m \omega_\text{T}}{2 \omega_\text{M}} \cos{(\omega_\text{T} + \omega_\text{M}) t} + \frac{m \omega_\text{T}}{2 \omega_\text{M}} \cos{(\omega_\text{T} - \omega_\text{M}) t}\right) \,,
\end{equation*}
woraus sich wieder drei Teilschwingungen wie bei der amplitudenmodulierten Schwingung ergeben.
Der Unterschied liegt darin, dass die Trägerschwingung gegen die beiden Seitenlinien um $\frac{\pi}{2}$ verschoben ist.

\subsection{Modulationsschaltungen}
\label{subsec:debye}

\subsection{Demodulator - Schaltungen}
\label{subsec:debye}

\section{Durchführung}
\label{sec:durchführung}
