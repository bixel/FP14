\section{H"amocyanin}

(Abbildung)

Mit einer Größe von \unit[3-8]{MDa} gehören Hämocyanine zu den größten bekannten Proteinen der Natur (Habe im Evolution of Hämocyanin structure gesucht, aber nichts über die Größe direkt gefunden).
Diese transportieren sowohl den Sauerstoff von marin lebenden wirbellosen Tieren als auch einiger Spinnenarten.
Sie liegen ohne an Blutzellen gebunden zu sein direkt in der Hämolymphe dieser Tierarten vor.
In den aktiven Zentren bindet der Sauerstoff an Kupfer und oxidiert diesen von Cu(I) zu Cu(II).
Dieses emittiert, durch Kupfer-Peroxo-Komplexbildung mit dem Sauerstoff, Lichtquanten im sichtbaren Wellenlängenbereich von \unit[480-420]{nm}. (http://www.chemieunterricht.de/dc2/komplexe/farbe.html)(http://archimed.uni-mainz.de/pub/2001/0109/diss.pdf)
Damit sind Lebewesen mit Hämocyaninen Blaublüter.

Hämocyanine besitzen mit einem Radius von ca. \unit[35]{nm} und einer Höhe von ca. \unit[15]{nm} eine hohlzylindrische Struktur (Quelle Paper von Julian).
Sie bestehen aus 10 strukturell ähnlichen Untereinheiten, und sind somit Decamere (Oligomere bestehend aus 10 Untereinheiten) (Quelle Paper Julian). 
Jede Untereinheit besteht weiterhin aus sieben bis acht funktionellen Einheiten (FE), die artspezifisch sind.
Das Hämocyanin des \textit{Sepia Officinalis} (\textit{gemeiner Tintenfisch}), das in dieser Arbeit untersucht wird, besteht beispielsweise aus acht dieser FEs (benannt N' term-a-b-c-d-d'-e-f-g-C' term) (Quelle Paper von Julian).
Jede FE ist ca. \unit[50]{kDa} schwer, wodurch das gesamte Protein ein Gewicht von ca. \unit[4]{MDa} besitzt.

Die FEs FE-a,FE-b,FE-c,FE-d,FE-e und FE-f sind bei allen bekannten Hämocyaninen vorhanden.
Dabei ähnelt sich deren Architektur und es liegt nahe, dass sich diese FEs vor der Aufspaltung der einzelnen Arten entwickelt haben.
Daher ist anzunehmen, dass diese in der Struktur dieselben Positionen besetzen und somit dieselbe Rolle beim Bilden der Quartärstruktur ihres Proteins spielen, obwohl dies noch nicht direkt gezeigt werden konnte (Habe die Paper gesucht aber nicht gefunden, waren das die, die du mir noch schicken wolltest?).
FE-d' hingegen entwickelte sich vor etwa 740 Millionen Jahren durch eine Duplizierung der FE-d.

Geometrisch betrachtet besteht der Hohlzylinder aus zwei kovalent miteinander verbundenen Bereichen, der äußeren Wand und dem inneren Kragen.
Die Wand ist charakteristisch für alle Hämocyanine und setzt sich aus den oben beschriebenen FEs a,b,c,d,e und f zusammen.
Der Kragen hingegen ist bei jeder Hämocyaninart anders aufgebaut und besteht beim \textit{Sepia Officinalis} aus FE-d' und FE-g.

Da die Funktionsweise von Proteinen durch ihre Quartärstruktur festgelegt ist, ist es notwendig eine Strukturanalyse durchzuführen.
Eine Möglichkeit wäre die hochauflösende Kristallographie, doch lassen sich besonders große Proteine nur sehr schwer kristallisieren.
Zudem liegen diese nicht mehr in ihrer nativen Konformation vor, wie es in Lösung der Fall ist.
Daher bietet sich die Kryotransmissionselektronenmikroskopie an, welche eine Hochauflösung der Struktur ermöglicht.

\FloatBarrier