\section{Kryopr"aparation}
\label{sec:probenpr_aparation_mit_kryopr_aparation}

Hämocyanine gehören zu den größten Proteinen der Natur und sind daher besonders anfällig für Rekonstruktionsfehler bei der negativen Kontrastierung (\cite{christos_diss}).
Bei der Kryopräparation wird die Probe in amorphem, vitrifiziertem Eis schockgefroren.
Damit erfüllt diese Methode die Anforderungen aus Kapitel \ref{sec:probenpr_aparation_mit_negativer_kontrastierung}.
Zudem wird die Probe in Lösung gefroren und nicht getrocknet und somit verbleibt diese in ihrer nativen Konformation.
Für den Vitrifizierungsvorgang ist das Gerät \textit{Cryo Plunge 3} verwendet worden.
Ein Gefrieren ohne Bildung von Eiskristallen ist mit flüssigem Stickstoff nicht möglich, da der Leidenfrost-Effekt ein langsames Gefrieren bedingt.
Daher findet das eigentliche Vitrifizieren in einem mit Ethan gefüllten Behälter statt, der mit Stickstoff gekühlt wird (\cite{bio_TEM}).

Wie in Kapitel \ref{sec:elektron_spektroskopie} beschrieben, sind keine Strukturen bei einem Amplitudenkontrast etwa gleich großer Elemente zu erkennen.
Da Wasser aus leichten Elementen besteht ist es besonders wichtig auf den Phasenkontrast nach Gleichung \eqref{KTF} zurückzugreifen.)

Der Probenträger ist ein Metallgitter (z.B. Quantifoil R2/1 Cu-300.).
Dieses ist bereits mit einem Kohlenstofffilm überzogen, in dem in regelmäßigen Abständen industriell Löcher geätzt worden sind (\cite{grid}).

Die Oberfläche des Gitters wird, wie in Kapitel \ref{sec:probenpr_aparation_mit_negativer_kontrastierung} beschrieben, durch ein Plasma polarisiert.
Zum Auftragen der Probe wird das Gitter in eine Pinzette eingespannt und in das Gerät gehängt.
Durch eine sich seitlich befindende Öffnung wird nun die Probe auf das Gitter aufgetragen.
Das Gerät übernimmt den Ablösch- und Gefriervorgang vollautomatisch.
Vom Probenträger wird die Probenlösung mit Filterpapier fast vollkommen entfernt, sodass nur noch ein dünner Film übrig bleibt.
Anschließend wird das Gitter in den mit flüssigem Ethan gefüllten Behälter fallen gelassen.
Dabei ist es wichtig, dass die Luftfeuchtigkeit erhöht ist.
Dadurch wird ein Verdunsten während des Fallens verringert.
Abschließend kommt das Gitter direkt in die sich in flüssigem Stickstoff befindliche Probenaufbewahrungsbox und diese wird wiederum bis zur Verwendung in flüssigem Stickstoff gelagert.

\FloatBarrier