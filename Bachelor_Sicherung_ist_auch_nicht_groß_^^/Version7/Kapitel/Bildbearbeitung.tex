\section{Digitale Bildbearbeitung} % (fold)
\label{sec:bildbearbeitung}

Bei der Kryo-TEM sind von derselben Position 15 Bilder in \unit[200]{ms} Abschnitten gemacht worden.
Um das Signal-zu-Rausch Verhältnis zu verbessern, werden diese miteinander verglichen, gleich ausgerichtet und aufsummiert.

In der Kopfzeile werden die Bilder zunächst nummeriert (\textit{micrograph ID}), sodass später eine Zuordnung möglich ist.
Anschließend wird die KTF der einzelnen Bilder analysiert und in der Kopfzeile als KTF-Objekt gespeichert.
Dafür wird ein KTF-Fit in das 1D Powerspektrum des Bildes gelegt, wobei z.B. der Defokus durch die Fitparameter bestimmt werden konnte (Vergleich \eqref{KTF}).
Dafür ist das Programm \textit{CTF-GUI} verwendet worden (Tabelle \ref{EDV}:GUI).
Dieses gibt neben den Fitparametern auch die dazugehörigen Bilder aus, welche bei der Qualitätskontrolle eine Rolle spielen.
Stimmt die Qualität nicht, kann das Bild sofort aussortiert werden.

Für die Extrahierung der Partikel im nächsten Schritt werden sowohl ein gaußförmiger Hochpass-, als auch ein gaußförmiger Tiefpassfilter auf eine Kopie der Bilder angewendet.
Dieser bewirkt, dass die besonders kleinen und großen Frequenzen (große und kleine Objekte) unterdrückt werden und die Strukturen der Proteine besser zu erkennen sind.

\FloatBarrier