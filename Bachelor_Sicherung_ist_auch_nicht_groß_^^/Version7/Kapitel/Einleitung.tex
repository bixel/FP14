
\chapter{Einleitung}
\label{einleitung}
Der Sauerstofftransport ist für alle Lebewesen von großer Bedeutung.
Bei marin lebenden wirbellosen Lebewesen und einigen Spinnenarten binden Hämocyanine den Sauerstoff. Um zu verstehen wie dieser Prozess funktioniert, muss die Quartärstruktur der Proteine bekannt sein.\\

In dieser Arbeit wird das Hämocyanin des \textit{Gemeinen Tintenfisches} (\textit{Sepia Officinalis}, Abbildung \ref{tinti}) näher untersucht. Mithilfe der Kryotransmissionselektronenmikroskopie wird versucht die Struktur aufzulösen. Von einer hochaufgelösten Struktur kann die Funktionsweise abgeleitet werden. 

\begin{wrapfigure}{r}{0.35\textwidth}
	\centering
	\includegraphics[width = 5cm]{Abbildungen/sepi.jpg}
	\caption[Bild eines \textit{Sepia Officinalis}]{\textit{Sepia Officinalis} (\cite{sepi_offi}).}
	\label{tinti}
\end{wrapfigure}
Dieses Protein ist bereits in mehreren Arbeiten behandelt worden (\cite{sepia_3D}), Quelle aus dem Paper von Julian noch suchen) es konnten jedoch bisher keine hinreichenden Auflösungen erzielt werden.
In der Arbeit von (Name hier) (\cite{old_struc}) ist eine $C5$ Symmetrie des Proteins postuliert worden.  (Name) (\cite{new_struc}) konnte durch eine Hochauflösung des Hämocyanins des (Tier hinzufügen) feststellen, dass die Wand aus sich überschneidenden Untereinheiten besteht.
Da angenommen wird, dass der Aufbau der äußeren Wand der Hämocyanine bei allen Tieren gleich ist, wird diese Aussage für das Hämocyanin des \textit{Gemeinen Tintenfisches} überprüft. \\

Die Arbeit ist wie folgt strukturiert.
Zunächst wird in Kapitel \ref{theo} auf Hämocyanine im Allgemeinen eingegangen.
Zusätzlich wird die Analyse des Hämocyanins mittels der Kryotransmissionselektronenmikroskopie thematisiert.
An diese schließt sich in Kapitel \ref{EPA} eine Erklärung der Einzelpartikelanalyse an, welche zu einer 3D Rekonstruktion des Proteins führt.
In Kapitel \ref{ergebi} werden die Ergebnisse diskutiert und insbesondere auf die Probleme bei der Aufschlüsselung von Strukturen mit nicht bekannter Symmetrie eingegangen.
Abschließend wird anhand der Rekonstruktion die Topologie der Untereinheiten bestimmt.

\FloatBarrier