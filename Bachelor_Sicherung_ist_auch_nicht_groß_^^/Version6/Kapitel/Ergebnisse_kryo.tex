\section{Kryotransmissionselektronenmikroskopie} % (fold)
\label{sec:kryopraeparation}

Die Gitter mit Probe sind bereits Präpariert gewesen, wobei der Puffer wie folgt zusammengesetzt worden ist:
(Puffer Rezeptur)

Nachdem einige Stellen mit der richtigen Eisdicke (Eisdicke jetzt hier) gefunden worden sind, war es möglich an ca. 800 Positionen Bilder aufzunehmen.
Während der Aufnahme sollte darauf geachtet werden, dass das Power Spektrum nicht astigmatisch oder verschwommen ist, um Fehler in der KTF zu minimieren.
Bei der Auswertung der Kryo-TEM Daten wurde nach der in Kapitel \ref{EPA} beschriebenen Einzelpartikelanalyse vorgegangen.
\\
\\
\textbf{Digitale Bildbearbeitung:}
Für jede der ca. 800 Position sind zunächst alle 15 Bilder miteinander verglichen, zueinander ausgerichtet und aufsummiert worden (Abbildung \ref{kryo_schema}: a).
Zudem werden die KTF Parameter mithilfe der \textit{CTF-GUI} bestimmt und in die Kopfzeilen der Bilder geschrieben.
Weiterhin werden keine Bilder aussortiert, da 800 Positionen nicht viel sind und auch schlechte Bilder etwas Information enthalten können.
Abschließend werden die Bilder kopiert, um den Faktor zwei, zum schnelleren Verarbeiten, verkleinert und sowohl Hoch-, als auch Tiefpass gefiltert, damit die Proteinstrukturen besser zu erkennen sind.
\\
\\
\begin{wrapfigure}{r}{0.2\textwidth}
	\centering
	\includegraphics[width =  0.2\textwidth]{Abbildungen/pdb2.png}
	\caption[Darstellung der Oktopus Röntgenkristallstruktur]{Verwendeten PDB Struktur 1JS8 eines \textit{Oktopus}-Hämocyanins ($http://www.rcsb.org/pdb/images/1js8_bio_r_500.jpg$).}
	\label{octo}
\end{wrapfigure}
\textbf{Extraktion der Partikel:}
Die zweifach verkleinerten Bilder sind mit einer Boxgröße von \unit[196]{px} ausgewählt worden, da diese nur geringfügig größer ist als das Protein selbst und der Mittelpunkt der Partikel leichter gefunden werden kann.
Für den Extraktionsprozess wird hingegen eine größere Boxgröße von \unit[256]{px} benutzt, da dies eine von EMAN empfohlene Größe ist.
Zudem werden Partikel, welche nicht ganz zentriert sind nicht abgeschnitten.
Die Koordinaten sind auf dem verkleinerten Datensatz ausgewählt worden, daher ist es nötig beim Wechsel auf den originalen Datensatz alle Koordinaten zu verdoppeln.
Dies führt dazu, dass sich auch die Boxgröße auf \unit[512]{px} vergrößert.
Anschließend werden die etwa 8000 Partikel invertiert, normalisiert und die Kopfzeile mit den aus den Bildern errechneten KTF-Informationen versehen.
Drei Partikel sind in Abbildung \ref{kryo_schema}:b dargestellt.
\\
\\
\textbf{Ausrichtung und Klassifizierung:}
In die \textit{GUI} ist der Partikeldatensatz eingelesen worden.
Dem K-Means Algorithmus werden ersten Durchlauf zehn und im zweiten Durchlauf 40 Klassen übergeben.
Die Entwicklung einiger Klassen ist in Abbildung \ref{kryo_schema}:c dargestellt.
\\
\\
\textbf{3D Rekonstruktion:} 
Mithilfe der Klassensummen ist es gelungen, eine der Geometrie entsprechende Startreferenz zu erzeugen (Abbildung \ref{kryo_schema}:d).
Aus dieser wurden erste 2D Projektionen erzeugt und mit den extrahierten Partikeln verglichen.
Nach verschiedenen Herangehensweisen hat sich eine Rekonstruktion ergeben, welche insgesamt auf etwa \unit[9.3]{\AA} aufgelöst ist (Vergleich Kapitel \ref{sec:symmetrieproblem}).
\begin{figure}
	\includegraphics[width = 14cm, height = 5cm]{Abbildungen/kryo.png}
	\caption[Kryo: Weg zur ersten Rekonstruktion]{\textbf{a):} Aufsummieren der 15 Bilder zu einem. \textbf{b):} Ausrichtung der Partikel. \textbf{c):} Klassifizierte Partikel. \textbf{d):} Erste Rekonstruktion.}
	\label{kryo_schema}
\end{figure}
\\
\\
\textbf{PDB Modell:}
Zum Vergleichen der Dichte mit der Röntgenkristallstruktur ist die Struktur des homologe \textit{Oktopus}-Hämocyanins benutzt worden (Abbildung \ref{octo}, PDB 1JS8 Quelle http://www.rcsb.org/pdb/explore/explore.do?structureId=1js8).
Zu beginn werden die Strukturen mit der Dichte verglichen, um die am wenigsten aufgelösten Bereiche zu finden. Hat sich die Auflösung soweit verbessert, dass feinere Strukturen in der Dichte sichtbar sind, ist es möglich, Anhand der N'- und C'-Terme der einzelnen FEs einen Strukturellen zusammenhang und damit eine Struktur zu erkennen (Vergleich Kapitel \ref{sec:aufschluesselung_der_struktur}).

\FloatBarrier