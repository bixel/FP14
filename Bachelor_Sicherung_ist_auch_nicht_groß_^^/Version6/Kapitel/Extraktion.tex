\section{Extraktion der Partikel} % (fold)
\label{sec:extraktion_der_partikel}

Aus den bearbeiteten Bildern müssen nun die einzelnen Partikel extrahiert werden.
Dafür ist das EMAN2 Programm \textit{e2boxer.py} verwendet worden.
Es wird dafür jeder Partikel auf jedem Bild einzeln ausgewählt und die Koordinaten der Partikelmittelpunkte in einer .box Datei gespeichert.
Anschließend werden die Koordianten auf den ungefilterten Datensatz angewendet und von dort extrahiert.
Dem EMAN-Programm \textit{batchboxer} wird die Boxgröße in Pixeln übergeben, wobei sich die gewählten Koordinaten im Mittelpunkt der quadratischen Box befinden.
Dabei sollte darauf geachtet werden, dass die Größe von EMAN empfohlen wird, da sich ansonsten die Rechenzeit erheblich verlängert (http://blake.bcm.edu/emanwiki/EMAN2/BoxSize).
Weiterhin werden die extrahierten Partikel in einem HDF Datensatz (\textit{stack}) gespeichert.
Abschließend wird der Datensatz normalisiert, invertiert und KTF-Korrigiert.

\FloatBarrier