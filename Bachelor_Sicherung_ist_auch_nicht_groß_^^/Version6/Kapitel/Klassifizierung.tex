\section{Ausrichtung und Klassifizierung} % (fold)
\label{sec:klassifizierung}

Klassifizieren bedeutet, dass Gruppen gleicher Projektionsrichtungen gefunden werden.
Durch Aufsummieren der Bilder zu einer Klassensumme wird das Signal-zu-Rausch Verhältnis verbessert, wodurch die geringe Elektronendosis ausgeglichen werden kann.
Eine Klassensumme entspricht dabei einer Projektionsrichtung.
Können manche Partikel keinen Klassen zugeordnet werden, werden diese aus dem Datensatz entfernt, da sie zum Rauschen beitragen.
Der Prozess ist mithilfe des Programms \textit{GUI} automatisiert worden.
\\
\\
\textbf{1. Ausrichtung (Referenzfrei):}
Zunächst werden alle Bilder der Partikel zu einem Bild aufsummiert und als Referenz benutzt.
Das Sparxmodul \textit{sxali2d.py} vergleicht zunächst alle Partikel mit dieser und richtet diese durch Rotation an diesem aus.
Anschließend werden die Partikel noch zueinander verschoben.
Mathematisch wird dazu eine Kreuzkorrelation angewandt (Quelle).
Nach jedem Iterationsschritt wird eine Überlagerung der neu gefundenen Partikel als Referenz genutzt.
\\
\\
\textbf{2. K-Means Klassifizierung:}
Eine mögliche Klassifizierungsmethode ist der K-Means Algorithmus (SPARX:\textit{sxk\_means.py}).
Dieser erstellt eine festgelegt Anzahl $K$ an Klassen, indem er die Bilder in einem Hyperraum in Clustern zusammenfasst.
Dazu werden zunächst $K$ zufällige Schwerpunkte in die Hyperebene gesetzt und durch Verschieben ebendieser die Cluster gefunden.
Dieser Algorithmus ist zwar einfach und schnell, jedoch nicht reproduzierbar, da das Ergebnis stark von den Anfangsbedingungen und der Qualität der Partikel abhängig ist (Quelle Paper).
\\ 
\\
\textbf{3. Ausrichtung (\textit{multi reference alignment}):}
Im Gegensatz zur referenzfreien Ausrichtung stehen nun die erzeugten Klassen als Referenz zur Verfügung.
Die Partikel werden der am besten passenden Klassensumme zugeordnet und nach dieser Ausgerichtet.
Anschließend werden die zugeordneten Partikel zu neuen Klassensummen aufsummiert.
\\ 
\\
\textbf{4. ISAC:}
Der ISAC (\textit{Iterative Stable Alignment and Clustering}) Algorithmus ist eine gute Möglichkeit zur Überprüfung der Klassenergebnisse (SPARX:\textit{sxisac.py}).
In diesem Fall wird nicht die Anzahl der Klassen, sondern die der Partikel pro Klasse vorgegeben.
Nach jedem Iterationsschritt werden die Klassen auf Stabilität und Reproduzierbarkeit im Hinblick auf Alignierung und Klassifizierung geprüft (Quelle Paper).

\FloatBarrier