\section{Aufschlüsselung der Struktur} % (fold)
\label{sec:aufschluesselung_der_struktur}
\begin{wrapfigure}{r}{0.38\textwidth}
	\centering
	\includegraphics[width = 5cm]{Abbildungen/aussen.png}
	\caption[Architektur der äußeren Wand]{\textbf{Oben:} Dichtevolumen des Hämocyanins. Farblich hervorgehoben ist einer der fünf strukturell gleich aufgebauten Bereiche der äußeren Wand, bestehend aus zwei Untereinheiten. Rot - FE-a; Hellgrün - FE-b; Orange - FE-c; Hellblau - FE-d; Dunkelblau - FE-e; Dunkelgrün - FE-f. \textbf{Unten:} Dichte des Hämocyanins inklusive der homologen Kristallstruktur.}
	\label{aussen}
	\centering
	\includegraphics[width = 5cm]{Abbildungen/innen2.png}
	\caption[Architektur des inneren Kragens]{Lila - FE-g, welche sich zu Dimeren zusammenfindet. Rot - Fe-d', welche die Lücken zwischen den FE-g Dimeren ausfüllt. \textbf{Oben:} Aufgeklappte Seitenansicht der inneren Struktur. \textbf{Unten:} Draufsicht von der oberen und der unteren Seite des Proteins.}
	\label{innen}
\end{wrapfigure}
\textbf{Äußere Wand:}\\
Eine Analyse der äußeren Wand mithilfe der homologen Kristallstrukturen lässt auf die in Abbildung \ref{aussen} dargestellte Architektur schließen.\cite{a:einstein}
Die Strukturen sind, wie in der Abbildung \ref{aussen}:Unten dargestellt, vollständig von der Dichte umschlossen.
Wären die FEs anders als angenommen angeordnet, würde dies zu größeren Lücken oder herausstehenden Strukturabschnitten führen.
Dies bestätigt, dass die Wand des \textit{Sepia}-Hämocyanins aus den in Kapitel \ref{haemo} beschriebenen Sequenzen aufgebaut ist.
\\
\\
\textbf{Innerer Kranz:}\\
Mithilfe der in Abbildung \ref{seq}:Unten dargestellten Sequenz des \textit{Sepia}-Hämocyanins und der bereits herausgefunden Architektur der Wand, konnten die FEs des inneren Kranzes identifiziert werden.
Damit dies erreicht werden konnte sind die N'- und C'-Terme der einzelnen FEs untersucht worden.
Der N'-Term stellt den Startpunkt einer Aminosäuresequenz dar.
Dieser muss mit dem C'-Term des vorherigen FE verbindbar sein.
\FloatBarrier
Diese Verbindungen (\textit{linker}) haben eine definierte Länge, weshalb nicht alle Kombinationen von FE Sequenzen in jedem Protein möglich sind.
Die Anordnung der FE-g und FE-d' sind in Abbildung \ref{innen} dargestellt.
Es ist zu erkennen, dass sich jeweils zwei FE-g innerhalb der Struktur zu einem Dimer zusammenschließen, während die FE-d' in den Lücken dazwischen liegen.
\\
\\
\textbf{Konformation der Untereinheiten}\\
\begin{figure}[h!]
\includegraphics[width = 14cm]{Abbildungen/Konformer2.png}
\caption[Architektur des \textit{Sepia}-Hämocyanins]{Architektur des \textit{Sepia}-Hämocyanins. \textbf{Oben links}: Schmematische Anordnung der funktionellen Einheiten. \textbf{Oben rechts}: Darstellung der fünf verschiedenen Konformationen der Untereinheiten inklusive der homologen Kristallstrukturen. \textbf{Unten:} Drauf- und Seitenansichten des Hämocyanins bei farblicher Markierung der verschiedenen Konformationen. Rot - Untereinheit b; Blau - Untereinheit c; Grün - Untereinheit e; Gelb - Untereinheit d; Lila - Untereinheit a.}
\label{done}
\end{figure}
Bei der Untersuchung des inneren Kranzes ist zunächst versucht worden auf Basis der postullierten Struktur des \textit{Oktopus}-Hämocyanins eine Architektur zu bestimmen (Abbildung \ref{seq}: Oben).
Eine Übereinstimmung der Kristallstrukturen in der rekonstruierten Dichte konnte jedoch nicht eindeutig festgestellt werden (Siehe \textbf{Innerer Kranz}.
Daher wurde der in Kapitel \ref{einleitung} beschriebene Ansatz der Überkreuzung zweier Untereinheiten überprüft (Abbildung \ref{done}): Oben links).
Die FEs passen bei dieser Möglichkeit des Strukturaufbaus offensichtlich besser zueinander und werden auch besser von der Dichte umschlossen.
Mit dieser neuen Struktur wurde durch das Vergleichen der zehn Untereinheiten herausgefunden, dass fünf verschiedene Konformationen existieren (Abbildung \ref{done}: Oben rechts).
Dabei kommt Typ a viermal, Typ b einmal, Typ c dreimal, Typ d einmal und Typ e einmal vor.
Die Verteilung der verschiedenen Konformationen ist in Abbildung \ref{done} (Unten) farblich dargestellt, wobei jede Farbe einem Typ entspricht.
Es ist zu erkennen, dass sich das Protein in zwei Bereiche unterteilt, einen geordneten und einen ungeordneten.
Dies ist durch den unsymmetrischen Bereich im inneren Kranz zu erklären, welcher sich genau im Zentrum der ungeordneten Konformationen befindet.

