\section{Trickreiche Symmetrieprobleme} % (fold)
\label{sec:symmetrieproblem}

\begin{figure}
	\includegraphics[width = 14cm, height = 6cm]{Abbildungen/sym_schem.png}
	\caption[Ablaufschema zur Symmetriebestimmung]{Ablaufschema zur Symmetriebestimmung. Mult: Verstärkung der Informationen, bin: grad der Verkleinerung des Datensatzes, C5: C5 Symmetrie.}
	\label{symmi}
\end{figure}

Um herauszufinden, welche Architektur das Hämocyanin des \textit{Sepia Officinalis} besitzt, ist nach dem in Abbildung \ref{symmi} dargestellten Schema vorgegangen worden.
Nach dem erzeugen der Startreferenz kann das Verfahren in vier Schritte unterteilt werden.
Dabei ist das Ziel der ersten drei Schritte eine gute Auflösung in allen Bereichen des Proteins zu erreichen.
Um dies zu realisieren werden die Partikel zunächst um den Faktor vier verkleinert um Rechenzeit zu sparen.
Dies führt zwar zu einer verschlechterten Auflösung, doch geht es zunächst um die Ausrichtung und Position der einzelnen FEs, wobei es sich um große Srukturen handelt.

Die originale reale Pixelgröße der Bilder beträgt \unit[0.83]{\AA/px}.
Wie in Kapitel \ref{sec:elektron_spektroskopie} kann damit eine maximale Auflösung von \unit[1.66]{\AA} erreicht werden.
Da die Größe des Bildes um den Faktor vier verkleinert wurde, vervierfacht sich damit auch die reale Pixelgöße auf \unit[3.32]{\AA/px} und somit ist die maximale Auflösung auf \unit[6.64]{\AA} beschränkt.

Der vierte Schritt dient dem Erkennen von Details und dazu wird auf einen zweifach verkleinerten Datensatz gewechselt.
Um die vorherigen Rekonstruktionen und Masken aus dem vierfach verkleinerten Datensatz nutzen zu können, muss deren Größe verdoppelt werden.
In diesem Schritt beträgt die reale Pixelgröße \unit[1.66]{\AA/px} und die Auflösung kann maximal \unit[3.32]{\AA} betragen.

Nach der Rekonstruktion kann das Volumen zudem noch mit einer sogenannten \textit{Benutzerfunktion} (engl.: \textit{Userfunction}) bearbeitet werden. In dieser werden vor allem Volumen ausgeschnitten, symmetrisiert und zusätzlich gefiltert um Rauschen zu unterdrücken.
\\
\\
\begin{wrapfigure}{r}{0.2\textwidth}
	\centering
	\includegraphics[width = 0.2\textwidth]{Abbildungen/inner_outer.png}
	\caption[Trennung des Proteins in zwei Bereiche]{Orange: Innerer Kragen; Blau: Äußere Wand}
	\label{i_o}
\end{wrapfigure}
\textbf{1. Schritt: Auflösung der äußeren Wand}\\
Zur Bestimmung der Architektur ist das Protein zunächst in 2 Bereiche unterteilt worden.
Einerseits in die äußere Wand und andererseits in den inneren Kragen (Abbildung \ref{i_o}).
Wie in Kapitel \ref{haemo} beschrieben, wird vermutet, dass die äußere Wand bei allen Hämocyaninen gleich aufgebaut ist.
Es ist davon auszugehen, dass es dadurch zur Bildung derselben räumlichen Anordnung kommen sollte und es wurde bereits bestätigt, dass in diesem Fall eine $C5$ Symmetrie vorliegt (Quelle).
Zur Überprüfung der $C5$ Symmetrie in der Wand wird dem Programm zunächst eine $C5$ Symmetrie für die Rekonstruktion übergeben.
Wenn keine $C5$ Symmetrie vorhanden wäre, würde die Symmetriesierung zu einer Zerstörung der Struktur führen, wie es im inneren Kragen der Fall gewesen ist.
Dies bestätigt die Vermutung aus Kapitel \ref{einleitung}, dass das Protein nicht $C5$ symmetrisch ist.
Im Vergleich dazu hat sich die Auflösung der Wand sichtbar verbessert, wodurch eine $C5$ Symmetrie in diesem Bereich bestätigt werden konnte.

Zudem wird eine noch bessere Auflösung durch eine KTF Korrektur und einer Anpassung des B-Faktors erreicht.
Während der gesamten Rekonstruktion ist eine alles umfassende Maske benutzt worden um nur das Rauschen außerhalb der Dichte zu unterdrücken, jedoch keine inneren Strukturen abzuschneiden.
\\
\\
\textbf{2. Schritt: Auflösung des inneren Kragens}\\
Nach der relativ guten Auflösung der äußeren Wand ist es nun von besonderem interesse, wie der innere Kragen zusammengesetzt ist.
Falls eine Symmetrie im inneren Kragen existiert, ist diese bisher nicht bekannt gewesen.
Daher wird während der nächsten Rekonstruktion dem Programm keine Symmetrieeigenschaft ($C1$) übergeben, sondern das erzeugte Volumen wird erst im Anschluss bearbeitet.
So werden für den nächsten Iterationsschritt bessere 2D Referenzprojektionen erhalten ohne auf den Rekonstruktionsprozess direkten Einfluss zu nehmen.

In den ersten Iterationen werden der Partikeln erste Projektionsrichtungen zugeordnet.
Sind diese vorhanden wird dazu übergegangen den äußeren Bereich nachträglich $C5$ zu symmetrisieren und die Inforamtion des innere Kragens durch eine Multiplikation zu besser auslesbaren niedrigeren Frequenzen zu verschieben.
Somit achtet das Programm beim nächsten Iterationsschritt bei der Rekonstruktion vermehrt auf diese Bereiche.
Dazu wird in der \textit{Benutzerfunktion} das Volumen mithilfe von Masken in die äußere Wand und den inneren Kragen aufgeteilt.

Zudem kann noch einmal überprüft werden, ob tasächlich eine $C5$ Symmetrie in der Wand vorliegt.
Da die Rekonstruktion ohne Symmetrie abläuft, wäre zu erwarten, dass sich bei falscher Symmetrisierung in der nächsten Iteration die Auflösung verschlechtert.
Diese hat sich jedoch verbessert, wodurch die $C5$ Symmetrie abermals bestätigt wurde.

Beim Teilen der Dichte in einen inneren und einen äußeren Bereich ist es zudem wichtig, welche Dichte ausgeschnitten wird.
Wird der äußere symmetrische Bereich ausgeschnitten, so verbleibt der unsymmetrische innere Kranz unverändert.
Da die Information zudem noch multiplikativ vergößert wird, wird auch das Rauschen im Rest des Bildes verstärkt.
Schneidet man hingegen den inneren Kragen aus und symmetrisiert den Rest, wird nicht nur die Auflösung der Wand verbessert, sondern auch das unsymmetrische Rauschen um die Dichte herum abgeschwächt.
Abschließend wird wie bei Schritt 1 die Auflösung durch KTF Korrektur und Anpassung des B-Faktors verbessert.
\\
\\
\textbf{3. Schritt: Auflösung eines unsymmetrischen Bereiches}\\
Im Laufe von Schritt 2 hat sich die Auflösung des inneren Bereiches soweit verbessert, dass die grobe Architektur sichtbar geworden ist.
Bei Betrachtung des Kragens fällt auf, dass vier Dichteeinheiten nicht so gut aufgelöst sind wie der Rest.
Zudem ist durch Rotation der Dichte zu erkennen, dass, bis auf den weniger gut aufgelösten Bereich, der innere Kragen eine $D1$ Symmetrie besitzt (Abbildung \ref{sym}).
Um auch im unsymmetrischen Bereich auf eine bessere Auflösung zu bekommen, wird eine ähnliche Methode wie in Schritt 2 angewendet.
Zunächst wird der Rekonstruktion keine Symmetrie übergeben, wobei auch die äußere Wand im Anschluss nicht symmetrisiert wird.
Dies könnte nämlich Auswirkungen auf die Referenzprojektionen des inneren Bereiches im nächsten Schritt haben.

In der \textit{Benutzerfunktion} wird nun anstatt des inneren, der unsymmetrische Bereich ausgeschnitten und dessen Information verstärkt.
Dabei gab es jedoch das Problem, dass sich der Bereich schnell verbessert hat. 
Die benutzte Maske passte daher nach dem ersten Iterationsschritt nicht mehr und um das Problem zu lösen ist nach jedem Iterationsschritt die Maske neu angepasst worden.
Nachdem nach einigen Iterationsschritten keine Anpassung der Maske mehr nötig war, konnte mit dem Rest der Iterationsschritte fortgefahren werden.

Nachdem sich die Auflösung des unsymmetrischen Bereiches verbessert hat wird eine erneute Iteration mit kleinen Winkelschritten durchgeführt.
Dabei wird wiederum das Volumen nicht nachträglich verändert, um die Projektionsrichtungen nicht zu beeinflussen.
Bevor das Volumen KTF korrigiert und eine Anpassung des B-Faktors vorgenommen wird, wird eine Iteration mit symmetriesierung der äußeren Wand vorgenommen.
\\
\begin{wrapfigure}{r}{0.2\textwidth}
	\centering
	\includegraphics[width = 0.2\textwidth]{Abbildungen/symmetrie.png}
	\caption[Trennenung des Proteins in Bereiche mit verschiedenen Symmetrien]{Blau: Äußere Wand mit $C5$ Symmetrie; Rot: Innerer Kragen mit $D1$ Symmetrie; Hellblau: Innerer Bereich ohne Symmetrie.}
	\label{sym}
\end{wrapfigure}
\textbf{4. Schritt: Höhere Auflösung}\\
Mit einem vierfach verkleinerten Datensatz ist der Auflösung eine relativ niedrige Grenze gesetzt worden.
Nachdem die grobe Struktur gelöst ist, können nun die Details der Dichte verbessert werden.
Dazu wird ein zweifach verkleinerter Datensatz erzeugt, da der originale Datensatz die Rechenzeiten erheblich verlängern würde, ohne das das Ergebnis anfangs deutlich besser wird.

Um die zuvor erreichten Ergebnisse verwenden zu können, muss die Größe der Referenzrekonstruktioen der letzten Iteration verdoppelt werden.
Da die Architektur bereits gelöst ist, können die Frequenzen den PDB Modellen entnommen werden.
Damit ist gleich sofort in der ersten Iteration eine KTF Korrektur möglich.
Abschließend ist noch der B-Faktor angepasst worden, sodass eine verbesserte Auflösung von insgesamt \unit[9.5]{\AA} erreicht werden konnte.
\FloatBarrier 