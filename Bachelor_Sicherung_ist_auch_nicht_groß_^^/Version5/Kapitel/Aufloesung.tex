\section{Verbesserung der Auflösung} % (fold)
\label{sec:verbesserung_der_aufloesung}

Um nach einem erfolgreichen Iterationsschritt eine verbesserte Referenz zu erhalten ist es von Vorteil, die zuletzt erzeugte Rekonstruktion zu bearbeiten.
Dies kann durch Maskierung, Filterung, Symmetriesierung oder KTF-Korrektur möglich gemacht werden.
\\
\\
\textbf{Maskierung:} Bei einer 3D Rekonstruktion entstehen Fragmente (Rauschen) außerhalb des Dichtevolumens des Partikels. 
Diese können beim nächsten Iterationsschritt zu fehlerhaften Zuordnungen der Projektionsparamter führen.
Ein Entfernen des Rauschens ist durch das Maskieren des Partikels mit einer binären Maske möglich.
Wird das Volumen mit der Maske multipliziert bleibt nur der Teil übrig, der von der Maske umschlossen ist.
\\
\\
\textbf{Filterung:} Stimmen die Partikelprojektionsrichtungen nicht genau mit der Realität überein, kommt es zu Überlappungen der Dichten.
Besonders bei den großen Frequenzen (kleinen Strukturen) entsteht so ein Rauschen im Volumen.
Daher ist es von Vorteil die diese aus der Dichte mithilfe eines Lowpassfilters herauszufiltern, um das Rauschen zu unterdrücken.
\\
\\
\textbf{Symmetrisierung:} Viele Strukturen von Proteinen weisen Symmetrien auf.
Dabei treten folglich genau dieselben strukturellen Anordnungen auf, welche auch in der Dichte übereinstimmen sollten.
Es ist in solchen Fällen nicht nur möglich während des Rekonstruktionsprozesses die Partikel mit der Referenz zu vergleiche, sondern auch innerhalb der Rekonstruktion die symmetrischen Bereiche. 
Besitzt ein Partikel beispielsweise eine $C5$ Symmetrie und werden die symmetrischen Bereiche miteinander verglichen ist das etwa so, als wären 5 
mal so viele Partikel zur Rekonstruktion verwendet worden.
Sind symmetriebrechende Abschnitte der Struktur vorhanden, ist es nötig diese zuvor mit einer extra angepassten Maske auszuschneiden.
Eine Symmetrisierung von nicht symmetrische Bereiche führt nämlich zu einer Zerstörung der Struktur.
\\
\\
\textbf{KTF-Korrektur:} Die KTF beinhaltet alle Strukturinformationen des Partikels, doch sind besonders die hohen Frequenzen nicht einfach auszulesen.
Wird nicht viel Wert auf eine Hochauflösung gelegt ist es zum Beispiel möglich die KTF-Informationen ab der ersten Nullstelle abzuschneiden, sodass nur die große Strukturen sichtbar werden.
Da jedoch für die Hochaufösung die großen Frequenzen relevant sind, ist es unter anderem möglich das Vorzeichen der negativen Amplituden zu ändern (\textit{phase flip}).
Dadurch werden die KTFs, welche durch verschiedene Defokus Werte erzeugt worden sind einheitlich und eine Ausrichtung der Partikel wird erleichtert (http://www.ncbi.nlm.nih.gov/pmc/articles/PMC3166661/).
Zudem können gesondert Struktureigene Frequenzen verstärkt werden (\textit{pw adjustment}).
Dafür werden Röntgenstrukturen in die Dichte gefittet und ist man sich sicher, dass diese an der richtigen Stelle sind werden deren Frequenzen extrahiert und in der KTF verstärkt.
Zudem können die großen Frequenzen mithilfe der Modifikation des B-Faktors besser betrachtet werden (Vergleich Kapitel \ref{sec:elektron_spektroskopie}). 
Jedoch verschlechtert sich das Signal-zu-Rausch Verhältnis, da bei der Verstärkung nicht zwischen Information und Rauschen unterschieden werden kann.

\FloatBarrier