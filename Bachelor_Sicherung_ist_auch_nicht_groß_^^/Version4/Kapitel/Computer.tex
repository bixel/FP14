Die Anzahl der verwendeten Partikel umfasst mehrere Tausend.
Um 3D Rekonstruktionen mithilfe der Einzelpartikelanalyse (Siehe \ref{einzelpartikelanalyse}) erstellen zu können, muss auf die Rechenleistung von Computern zurückgegriffen werden.
Eine Liste der verwendeten Systeme und Programme ist in Tabelle \ref{EDV} gegeben.

Um eine effizientere Methode der Verarbeitung zu gewährleisten ist es nötig das Bildformat der Bilddateien von TIFF (\textit{tagged image file format}) auf HDF (\textit{hierarchical data format}) zu ändern.
Diese Art von Dateien besitzen eine Kopfzeile (\textit{header}), in der Bildinformationen (Projektionsparamter, Verschiebungen, Bildzugehörigkeit etc.) gespeichert werden können.
Zudem gibt es die Möglichkeit die Speicherung in einer Datenbank (BDB, \textit{Berkeley-Datenbak} vorzunehmen.
In dieser sind Bild und Kopfzeile getrennt voneinander gespeichert, sodass diese getrennt geladen werden können um die Zugriffszeit zu minimieren.

In diesem Kapitel wird besonders auf die Datenauswertung der bei der Kryo-TEM entstandenen Daten eingegangen.
Bei der negativen Kontrastierung sind die Schritte analog. 
Der schematische Rekonstruktionsprozess kann wie folgt unterteilt werden:

\begin{enumerate}
	\item \textbf{Digitale Bildbearbeitung:} Bestimmung des Defokus aus der KTF-Information; Aussortieren von Bildern (\textit{micrograph}); Filterung der Bilder
	\item \textbf{Extraktion der Partikel:} Auswahl der Partikel aus den Bildern (\textit{particle picking}); Speichern der Einzelnen Partikel in einem Datensatz (\textit{stack})
	\item \textbf{Klassifizierung:} Zentrierung und Drehung der Partikel (\textit{alignment}); Zuordnung der Partikel in verschiedene Klassen; Aussortierung schlechter Einzelbilder
	\item \textbf{3D Rekonstruktion:} Iterative Bestimmung der Projektionsparameter; Rekonstruktion; Vergleich mit PDB Modellen
\end{enumerate}

\begin{table}
	\begin{tabular}[!h]{l l l}
		Name & Verwendung & Quelle \\
		\hline
		Eman/Eman2 & Prozessierung & http://blake.bcm.edu \\
		Sparx & Prozessierung & http://sparx-em.org \\
		CTF-GUI & KTF \& Defokus bestimmung & MPI-Dortmund (R.Efremov, C.Gatsogiannis)\\
		GUI & Klassifizierung (K-Means) & MPI-Dortmund (C.Gatogiannis)\\ 
		USFC Chimera & 3D Modellierung & http://www.cgl.ucsf.edu/chimera \\
		Python & Programmierung & https://www.python.org \\
		CLB cluster & Datenverarbeitung & MPI Dortmund \\
		\hline
		\hline
	\end{tabular}
	\caption[Verwendete EDV]{Verwendete Software, Programmiersprachen und Systeme.}
	\label{EDV}
\end{table}



\FloatBarrier