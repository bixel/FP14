\section{Diskussion}
\label{sec:diskussion}
Die Arbeit mit dem HeNe-Laser bietet tiefe Einblicke in die Funktionsweise
eines Gas-Lasers und macht dessen Empfindlichkeit deutlich.
Die Ergebnisse der Messungen stimmen großteils mit den Erwarteten Ergebnissen
überein.
Die Große Abweichung der mit Hilfe eines Spaltes bestimmten Wellenlängen
in Abschnitt \ref{subsec:wellenlänge} kann jedoch nicht ohne Weiteres
erklärt werden. Sie weist auf einen schwerwiegenden Systematischen Fehler hin.
Die Wellenlängenbestimmung mit Hilfe des Gitters stimmt mit
\input{build/lambda_grid.tex} dagegen innerhalb der Fehler mit
dem Erwarteten Wert $\lambda = \SI{638.2}{\nano\meter}$ überein.
Die Überprüfung der Stabilitätsbedingungen liefert beim konvexen Spiegel
eine gute Übereinstimmung mit den zu erwartenden Werten, während der maximale
Abstand bei Nutzung des planen Spiegels stark unterhalb des theoretischen
Wertes liegt. Dies verdeutlicht, dass der Laseraufbau mit einem planen Spiegel
eine genauere Justage erfordert, als der entsprechende Aufbau
mit Hilfe eines konvexen Spiegels.
