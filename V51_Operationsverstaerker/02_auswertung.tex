\section{Auswertung}
\label{sec:auswertung}
Im Folgenden werden alle Fehler mit Hilfe der \texttt{python}-Bibliothek
\texttt{uncertainties}\cite{py-uncertainties} berechnet, die eine Gaußsche
Fehlerfortpflanzung implementiert.

\subsection{Frequenzgang eines gegenkekoppelten Verstärkers} % (fold)

Im folgenden wird der Frequenzgang eines gegengekoppelten Verstärkers bei vier verschiedenen Verstärkungsgraden $V'$.
Bei allen Messungen wurde eine Eingangsspannung von $\SI{15}{\milli\ohm}$ verwendet.
Zudem ist auf den Widerstand $R_1$ noch der Innenwiderstand der Spannungsquelle mit $R_i = \SI{50}{\ohm}$ aufaddiert worden.

\paragraph{Verstärungsgrad 1 $\frac{\SI{1}{\mega\ohm}}{\SI{100}{\ohm}}$}

\begin{table}
\centering
\caption{Messwerte zum Verstärkungsgrad 1.}
    \label{tab:a_messwerte_1}
    \input{build/a_data_1.tex}
\end{table}

In Grafik \ref{fig:a_plot_1} sind die gemessenen Messdaten aus Tabelle \ref{tab:a_messwerte_1} halblogarithmisch dargestellt.
Mithilfe einer linearen Regression der Form $f_1(x)=m \cdot x + b$ wird die Steigung der fallenden Geraden bestimmt.
Es ergibt sich
\begin{equation*}
	f_1(x) = (-0.9480\pm0.0001) \cdot x + (13.0129\pm0.0111)\,.
\end{equation*}
Die Grenzfrequenz $\nu'_{g,1}$ - die Frequenz, bei der die Verstärkung auf $\frac{V'_1}{\sqrt{2}}$ abgefallen ist - kann bei dieser Verstärkung nicht bestimmt werden, da kein Plateau bei geringen Frequenzen erkennbar ist.
Somit kann auch keine Leerlaufverstärkung $V_1$ abgeschätzt werden.

\begin{figure}[h!]
    \centering
    \includegraphics[width=0.8\linewidth]{build/a_plot_1.pdf}
    \caption{Frequenzgang eine gegengekoppelten Verstärkers mit einer Verstärkung von $10\,\mathrm{k}$.}
    \label{fig:a_plot_1}
\end{figure}

\paragraph{Verstärungsgrad 2 $\frac{\SI{120}{\kilo\ohm}}{\SI{100}{\ohm}}$}

\begin{table}
\centering
\caption{Messwerte zum Verstärkungsgrad 2.}
    \label{tab:a_messwerte_2}
    \input{build/a_data_2.tex}
\end{table}

In Grafik \ref{fig:a_plot_2} sind die gemessenen Messdaten aus Tabelle \ref{a_tab_2} halblogarithmisch dargestellt.
Mithilfe einer linearen Regression der Form $f_2(x)=m \cdot x + b$ wird die Steigung der fallenden Geraden bestimmt.
Es ergibt sich
\begin{equation*}
	f_2(x) = (-0.9652\pm0.0002) \cdot x + (13.0827\pm0.0191)\,.
\end{equation*}
Die Grenzfrequenz $\nu'_{g,2}$ - die Frequenz, bei der die Verstärkung auf $\frac{V'_2}{\sqrt{2}}$ abgefallen ist - ist zu $\nu'_{g,2} = \SI{1567.4(193)}{\hertz}$ bestimmt worden.
Der Wert für die Verstärkung $V'_2$ lässt sich aus dem Wert des Plateaus zu $V'=560.0\pm6.7$ bestimmen.
Die ideale Verstärkung 
\begin{equation*}
    V'_{2,\text{ideal}} = \frac{R_N}{R_1} \approx 800.0
\end{equation*}
weicht somit um etwa $\SI{30}{\percent}$ von der gemessenen ab.
Mithilfe der Gleichung \eqref{calc_V} wird die Leerlaufverstärkung $V_2$ zu
\begin{equation*}
	V_2 = \frac{1}{V'_2} - \frac{R_1}{R_N} = \frac{1}{560.0\pm6.7} - \frac{\SI{150}{\ohm}}{\SI{120}{\kilo\ohm}} \approx 
    \num{1.87(7)e3}
\end{equation*}

\begin{figure}[h!]
    \centering
    \includegraphics[width=0.8\linewidth]{build/a_plot_2.pdf}
    \caption{Frequenzgang eine gegengekoppelten Verstärkers mit einer Verstärkung von $1200$.}
    \label{fig:a_plot_2}
\end{figure}

\paragraph{Verstärungsgrad 3 $\frac{\SI{4.7}{\kilo\ohm}}{\SI{100}{\ohm}}$}

\begin{table}
\centering
\caption{Messwerte zum Verstärkungsgrad 3.}
    \label{tab:a_messwerte_3}
    \input{build/a_data_3.tex}
\end{table}

In Grafik \ref{fig:a_plot_3} sind die gemessenen Messdaten aus Tabelle \ref{a_tab_3} halblogarithmisch dargestellt.
Mithilfe einer linearen Regression der Form $f_3(x)=m \cdot x + b$ wird die Steigung der fallenden Geraden bestimmt.
Es ergibt sich
\begin{equation*}
    f_3(x) = (-0.7991\pm0.0001) \cdot x + (11.0282\pm0.0187)\,.
\end{equation*}
Die Grenzfrequenz $\nu'_{g,3}$ - die Frequenz, bei der die Verstärkung auf $\frac{V'_3}{\sqrt{3}}$ abgefallen ist - ist zu $\nu'_{g,3} = \SI{3.060(6)e4}{\hertz}$ bestimmt worden.
Der Wert für die Verstärkung $V'_3$ lässt sich aus dem Wert des Plateaus zu $V'=22.667\pm0.033$ bestimmen.
Die ideale Verstärkung 
\begin{equation*}
    V'_{3,\text{ideal}} = \frac{R_N}{R_1} \approx 31.33
\end{equation*}
weicht somit um etwa $\SI{27}{\percent}$ von der gemessenen ab.
Mithilfe der Gleichung \eqref{calc_V} wird die Leerlaufverstärkung $V_3$ zu
\begin{equation*}
    V_3 = \frac{1}{V'_3} - \frac{R_1}{R_N} = \frac{1}{22.667\pm0.033} - \frac{\SI{150}{\ohm}}{\SI{4.7}{\kilo\ohm}} \approx \num{81.9\pm0.4}
\end{equation*}

\begin{figure}[h!]
    \centering
    \includegraphics[width=0.8\linewidth]{build/a_plot_3.pdf}
    \caption{Frequenzgang eine gegengekoppelten Verstärkers mit einer Verstärkung von $47$.}
    \label{fig:a_plot_3}
\end{figure}

\paragraph{Verstärungsgrad 4 $\frac{\SI{1}{\mega\ohm}}{\SI{4.7}{\kilo\ohm}}$}

\begin{table}
\centering
\caption{Messwerte zum Verstärkungsgrad 4.}
    \label{tab:a_messwerte_4}
    \input{build/a_data_4.tex}
\end{table}

In Grafik \ref{fig:a_plot_4} sind die gemessenen Messdaten aus Tabelle \ref{a_tab_4} halblogarithmisch dargestellt.
Mithilfe einer linearen Regression der Form $f_4(x)=m \cdot x + b$ wird die Steigung der fallenden Geraden bestimmt.
Es ergibt sich
\begin{equation*}
    f_4(x) = (-0.8821\pm0.0002) \cdot x + (12.1148\pm0.0217)\,.
\end{equation*}
Die Grenzfrequenz $\nu'_{g,4}$ - die Frequenz, bei der die Verstärkung auf $\frac{V'_4}{\sqrt{4}}$ abgefallen ist - ist zu $\nu'_{g,4} = \SI{4.57(11)e3}{\hertz}$ bestimmt worden.
Der Wert für die Verstärkung $V'$ lässt sich aus dem Wert des Plateaus zu $V'_4=152.7\pm3.3$ bestimmen.
Die ideale Verstärkung 
\begin{equation*}
    V'_{4,\text{ideal}} = \frac{R_N}{R_1} \approx 210.53
\end{equation*}
weicht somit um etwa $\SI{27}{\percent}$ von der gemessenen ab.
Mithilfe der Gleichung \eqref{calc_V} wird die Leerlaufverstärkung $V_4$ zu
\begin{equation*}
    V_4 = \frac{1}{V'_4} - \frac{R_1}{R_N} = \frac{1}{152.7\pm3.3} - \frac{\SI{4.75}{\kilo\ohm}}{\SI{1}{\mega\ohm}} \approx \num{560\pm40}
\end{equation*}

\begin{figure}[h!]
    \centering
    \includegraphics[width=0.8\linewidth]{build/a_plot_4.pdf}
    \caption{Frequenzgang eine gegengekoppelten Verstärkers mit einer Verstärkungsgrad von $212.77$.}
    \label{fig:a_plot_4}
\end{figure}

\paragraph{Grenzfrequenz - Verstärkugnsgrad Produkt}

Die Ergebnisse für das Produkt aus Grenzfrequenz $\nu'_{g,i}$ und Verstärkung $V'_i$ sind in Tabelle \ref{tab:a_nu_v} aufgerführt.
\begin{table}[h!]
\centering
\caption{Produkt aus Grenzfrequenz $\nu'_{g,i}$ und Verstärkung $V'_i$ der Messaufbauten 2 bis 4.}
    \label{tab:a_nu_v}
    \begin{tabular}{S[table-format=3.3(3)] S[table-format=3.3(2)] S[table-format=1.4(2)]}
            \toprule
            {Grenzfrequenz $\nu'_{g,i}/\si{\hertz}$} & {Verstärkung $V'_i$} & {$V'_i\cdot\nu'_{g,i}/\SI{e5}{\hertz}$}\\
            \midrule
            1567.4(193) & 560.0(67)  & 8.8032(20) \\
            30600(60)  & 22.667(33) & 6.9354(26)    \\
            4570(110)  & 152.7(33)  & 6.973(20)    \\
            \bottomrule
    \end{tabular}    
\end{table}
Es ist zu erkennen, dass die Größenordnung des Produktes dieselbe ist und die Werte maximal um etwa $\SI{20}{\percent}$ voneinander abweichen.

\subsection{Klemmenspannung eines speziellen NF-Generators}
\label{sub:klemmenspannung_eines_speziellen_nf_generators}

In den Schaltungen wurden die Widerstände $R_N = \SI{12}{\kilo\ohm}$ und $R_1 = \SI{100}{\ohm}$ gewählt.
Die Ergebnisse der Ausgangsspannungen sind in Tabelle \ref{tab:b_volt}
Der Unterschied in der Ausgangsspannung könnte sich durch die verschiedenen Eingangswiderstände der Schaltungen erklären lassen.

\begin{table}[h!]
    \centering
    \caption{Vergleich der Klemmenspannung eines speziellen NF-Generators mithilfe einer gegengekoppelten und einer Elektrometer-Verstärkerschaltung mit etwa der gleichen Verstärkung $V'$.}
    \label{tab:b_volt}
    \begin{tabular}{S[table-format=3.0] S[table-format=1.1(1)] S[table-format=2.1(1)]}
        \toprule
        {$\nu/\si{\hertz}$} & {$U_\mathrm{Gg}/\si{\volt}$} & {$U_\mathrm{Elek}/\si{\volt}$}\\
        \midrule
         50 & 8.4(1) & 12.5(1) \\
        100 & 8.4(1) & 12.5(1) \\
        150 & 8.4(1) & 12.3(1) \\
        200 & 8.3(1) & 12.1(1) \\
        250 & 8.3(1) & 11.9(1) \\
        \bottomrule
    \end{tabular}
\end{table}

\subsection{Messung des Eingangswiderstandes $R_e$ und der Leerlaufverstärkung $V$} % (fold)
\label{sub:}

Für die Berechnung des Eingangswiderstandes wurde die Generatorspannung $U_g = \SI{1.0(1)}{\volt}$, der Generatorwiderstand $R_v = \SI{10}{\kilo\ohm}$ und der Widerstand $R_N = \SI{100}{\kilo\ohm}$ gewählt.
Die Ergebnisse sind in Tabelle \ref{tab:c_ergebnisse} aufgeführt. Das Ergebnis in dem Frequenzbereich von $\SIrange{100}{1600}{\hertz}$ ist in der doppellogarithmischen Grafik \ref{fig:c_plot} dargestellt. Zudem ist in der doppellogarithmischen Grafik \ref{fig:c_plot_full} der Frequenzbereiches von $\SIrange{100}{100000}{\hertz}$ abgebildet.

Mithilfe der Gleichung
\begin{equation*}
    I = \frac{U_g}{R_v}
\end{equation*}
berechnet sich $I = \input{build/c_I.tex}$.
Mithilfe von $I$ ergibt sich nun der Eingangswiderstand $R_e$ aus dem Zusammenhang
\begin{equation*}
    R_e = \frac{U_e}{I}\,.
\end{equation*}
$U_a$ berechnet sich somit zu
\begin{equation*}
    U_a = R_N \cdot I = \input{c_U_a.tex} \,.
\end{equation*}
Die Ergebnis der Messung bis $\SI{20}{\kilo\hertz}$ liegen somit innerhalb des Fehlers des berechneten Wertes.
Die Leerlaufverstärkung $V$ lässt sich mithilfe von 
\begin{equation*}
    V = \frac{R_N}{R_e}
\end{equation*}
berechnen.

Mithilfe einer linearen Regression der Form $f_{c,R}(x) = m \cdot x + b$ ergeben sich für die linearen Bereiche die Geradengleichungen 
\begin{eqnarray*}
    f_{c,R,1}(x) &=& (\num{1.04525(7)}) \cdot x + (\num{-3.4633(24)}) \,, \\
    f_{c,V,1}(x) &=& (\num{-1.04525(7)}) \cdot x + (\num{14.9763(24)}) \,, \\
    f_{c,R,2}(x) &=& (\num{1.0493(1)}) \cdot x + (\num{-3.495(5)}) \,, \\
    f_{c,V,2}(x) &=& (\num{-1.0493(1)}) \cdot x + ( ) \,.
\end{eqnarray*}
Da die Steigungen im Bereich kleiner Freqeunzen bis auf die dritte Nachkommastelle mit denen größerer Frequenzen übereinstimmt, bestimmt sich der Frequenzbereich, in dem der Operationsverstärker einwandfrei arbeitet, zu $\SIrange{100}{20000}{\hertz}$.
Somit ist die Grenzfrequenz $\nu_g \approx \SI{20}{\kilo\hertz}$.
In den Messwerten ist dies gerade durch den Wert für die Frequenz $\nu$ erkennbar, an dem die Ausgangsspannung $U_a$ aufhört konstant zu bleiben.
Dies ist somit auch der Bereich, indem nicht mehr $R_v \gg R_e$ gilt und somit auch der Strom $I$ nicht mehr als konstant angenommen werden darf.

\begin{figure}[h!]
    \centering
    \includegraphics[width=0.8\linewidth]{build/c_plot.pdf}
    \caption{Doppellogarithmische Darstellung des Widerstands $R_e$ und der Leerlaufverstärkung $V$ in Abhängigkeit der Frequenz $\nu$ im Bereich $\SIrange{100}{1600}{\hertz}$.}
    \label{fig:c_plot}
\end{figure}

\begin{figure}[h!]
    \centering
    \includegraphics[width=0.8\linewidth]{build/c_plot_full.pdf}
    \caption{Doppellogarithmische Darstellung des Widerstands $R_e$ und der Leerlaufverstärkung $V$ in Abhängigkeit der Frequenz $\nu$ im Bereich $\SIrange{100}{100000}{\hertz}$.}
    \label{fig:c_plot_full}
\end{figure}

\begin{table}[h!]
    \centering
    \caption{Ergebnisse für den Eingangswiderstand $R_e$, die Leerlaufverstärkung $V$, die Eingangsspannung $U_e$ und die Ausgangsspannung $U_a$ bei verschiedenen Frequenzen $\nu$.}
    \label{tab:c_ergebnisse}
    \input{build/c_table.tex}
\end{table}