\section{Diskussion}
\label{sec:diskussion}

Alles in allem ist die Funktionsweise eines Operationsverstärkers verständlich gemacht worden.

Das Grenzfrequenz-Verstärkungsgrad Produkt verschiedener Verstärkungsgrade weicht bei den genutzten Operationsverstärkern um etwa $\SI{20}{\percent}$ voneinander ab.
Daher ist anzunehmen, dass dieses Gesetz gültig ist.

Zudem ist der Unterschied verschiedener Operationsverstärkertypen gezeigt worden.
So unterscheidet sich die Klemmenspannung eines Gegengekoppelten- und eines Elektrometer-Verstärkers.
Dies könnte jedoch am Unterschied der verschiedenen Eingangswiderstände $R_e$ erklären lassen.

Bei der Bestimmung des Eingangswiderstandes und der Leerlaufverstärkerung ergab sich, dass dieser mit steigender Frequenz ansteigt und somit die Leerlaufverstärkung abnimmt.
Zudem konnte der Frequenzbereich, in dem der Operationsverstärker einwandfrei arbeitet, zu $\SIrange{100}{20000}{\hertz}$ bestimmt werden.

Es ist möglich gewesen, mithilfe eines Operationsverstärkers Spannungen sowohl zu integrieren, als auch zu differenzieren.
Auch die Verwendung als Schmitt-Trigger und als Dreieckspannungsgenerator war möglich.
Die Kippspannung beim Schmitt-Trigger weicht dabei um etwa $\SI{4}{\percent}$ vom errechneten Wert ab.

Generiert man eine gedämpfte Schwingung, so konnte daraus die die Abklingdauer zu $\tau \approx \SI{-0.0046}{\second}$ bestimmt.

Die Verwendung als Exponentierer konnte nicht bestätigt werden, doch war die Verwendung als Logarithmierer sehr gut möglich. 
Die Änderung der Ausgangsspannung, wenn sich die Eingangsspannung um eine Zehnerpotenz ändert, konnte zu $\Delta U_a \approx \SI{108.3}{\milli\volt}$ bestimmt werden und die Betreibstemperatur zu $T \approx \SI{557}{\kelvin}$.

Die Phase zwischen Eingangs- und Ausgangsspannung lag bei dem verwendeten Operationsverstärker bei $\SI{180}{\degree}$