\section{Diskussion} % (fold)
\label{sec:diskussion}

Bei der Berechnung der Molwärme $C_\mathrm{V}$ von Kupfer kam es teilweise zu großen Abweichungen von der Theoriekurve (Siehe Abbildung \ref{Cv_gra}). 
Dies könnte sich dadurch erklären lassen, dass der Einfluss der abgegebenen Wärmestrahlung die Wärme $Q$ der Formel beeinflusst.
Die großen Fehler im Bereich von $\SIrange{210}{270}{\kelvin}$ sind durch einen Fehler des Messgerätes entstanden.
Dieses hat trotz Erhöhung der Leistung einen konstanten Wert für die Spannung angezeigt, obwohl dieser gestiegen sein müsste.
Festgestellt wurde dies jedoch erst beim herabsenken der Leistung, da sich auch dabei der Wert nicht weiter gesenkt hat.\\

Zu Beginn der Messung wurde die Probe schneller Warm als der äußere Kupferzylinder, weshalb sich die Wärme $Q$ durch eine vermehrte Abgabe von Wärmestrahlung verringert haben könnte.
Dies wäre eine Erklärung für den erhöhten $C_\mathrm{V}$ Wert im Bereich von $\SIrange{100}{140}{\kelvin}$.
Ab etwa $\SI{240}{\kelvin}$ kam es zum gegenteiligen Effekt.
Der äußere Kupferzylinder wurde schneller Warm als die Probe, weshalb sich die Wärme $Q$ durch eine vermehrte Aufnahme von Wärmestrahlung erhöht haben könnte.
Die würde den geringeren $C_\mathrm{V}$ Wert im Bereich von $\SIrange{240}{300}{\kelvin}$ erklären.\\

Bei der Bestimmung der Debye-Temperatur wurde ein Wert von \input{build/theta_D.tex} errechnet.
Dieser weicht um $\approx \SI{11.21}{\percent}$ vom Litearturwert $\Theta_\mathrm{D} = \SI{345}{\kelvin}$ \cite{kupfer3} ab.
Die Abweichung lässt sich durch den erhöhten $C_\mathrm{V}$ Wert im Bereich von $\SIrange{100}{140}{\kelvin}$ erklären.\\

Bei der theoretischen Betrachtung der Debye Temperatur mit den gegebenen Geschwindigkeiten für die longitudinale und die transversale Richtung wurde \input{build/theta_D_theo.tex} errechnet.
Dies entspricht einer Abweichung vom Literaturwert um $\approx \SI{3.92}{\percent}$. 
Dies zeigt, dass die Debye-Näherung hinreichend genaue Aussagen trifft.
Vom gemessenen Wert weicht dieser um $\approx \SI{7.89}{\percent}$ ab.
Dies kann durch die bereits erläuterten Einflüsse der Wärmestrahlung erklärt werden.\\

Abschließend lässt sich sagen, dass der Versuch zur Bestimmung der Molwärme von Kupfer geeignet ist.
Es sollte jedoch verstärkt darauf geachtet werden, dass das Gehäuse und die Probe etwa dieselbe Temperatur haben.