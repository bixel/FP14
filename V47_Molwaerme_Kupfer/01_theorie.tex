\section{Theoretische Grundlagen}
\label{sec:theorie}
In diesem Versuch soll die Molwärme $c_V$ bei konstantem Volumen in
Abhängigkeit von tiefen Temperaturen untersucht werden.
Während eine klassische Betrachtung eine konstanten Wert für $c_V$
liefert, erhält man mit Einbeziehung quantenmechanischer Effekte eine
Temperaturabhängigkeit, die im Folgenden beschrieben und schließlich 
gemessen wird.

\subsection{Klassiche Theorie der Molwärme}
\label{subsec:klassisch}
In der klassischen Wärmelehre verteilt sich die Energie in einem Körper
gleichmäßig auf alle Atome. Dabei entfallen auf jeden Freiheitsgrad im Mittel
der Anteil $\sfrac{1}{2}\kb T$, mit der Boltzmannkonstante $\kb$ und der
Temperatur $T$.
In einem Festkörper sitzen die Atome an festen Gitterplätzen, um die sie
Schwingungen in alle Raumrichtungen ausführen können. Für einen harmonischen
Oszillator entspricht die mittlere kinetische Energie der mittleren
potentiellen Energie, weshalb die gesamte Energie $U$ eines Mols klassisch
durch
\begin{equation}
    \label{eqn:innere_energie}
    U = 3 \kb \mathup{N}_\text{A} T = 3\mathup{R}T
\end{equation}
gegeben ist. Hierbei bezeichnet $\mathup{N}_\text{A}$ die Avogadrokonstante
und $\mathup{R}$ die allgemeine Gaskonstante.

Die Spezifische Molwärme ist durch die Ableitung der inneren Energie $U$ nach
der Temperatur bei konstantem Volumen definiert. Es folgt damit
die bekannte, temperatur- und materialunabhängige Größe
\begin{equation}
    \label{eqn:cv_klassisch}
    c_V = \left.\frac{\partial U}{\partial T}\right|_V = 3 \mathup{R}\,.
\end{equation}

\subsection{Quantenmechanisch: Das Einstein-Modell}
\label{subsec:einstein}

\subsection{Quantenmechanisch: Das Debye-Modell}
\label{subsec:debye}
