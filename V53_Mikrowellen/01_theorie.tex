\section{Einleitung}
\label{sec:einleitung}
Als allgegenwärtige Anwendung von physikalischem Know-How ist die
Mikrowellentechnik kaum noch aus dem Alltag wegzudenken.
Von dem Haushaltsgerät Mikrowelle bis zu Alarmanlagen wird diese Technik
in äußerst breit gefächerten Gebieten genutzt.

Ursprünglich schon zu Ende des zweiten Weltkrieges, als Frühwarnradar
entwickelt, wird Mikrowellenstrahlung schon einige Jahrzehnte eingesetzt.
Aus diesem Grund stellt sich die Untersuchung von Mikrowellen als interessante
Aufgabe für jeden Physiker dar.

Im Folgenden sollen verschiedene Welleneigenschaften mit Hilfe eines
Reflexklystrons untersucht und das Verhalten von Mikrowellen auf Hohlleitern
betrachtet werden.
