\section{Einleitung}
\label{sec:einleitung}
Als allgegenwärtige Anwendung von physikalischem Know-How ist die
Mikrowellentechnik kaum noch aus dem Alltag wegzudenken.
Von dem Haushaltsgerät Mikrowelle, das in wahrscheinlich jeder Küche der
westlichen Hemisphäre zu finden ist, bis zu Alarmanlagen wird diese Technik
in äußerst breit gefächerten Gebieten genutzt.
Ursprünglich schon zu Ende des zweiten Weltkrieges, als Frühwarnradar
entwickelt, wird Mikrowellenstrahlung schon einige Jahrzehnte eingesetzt.
Aus diesem Grund stellt sich die Untersuchung von Mikrowellen als interessante
Aufgabe für jeden Physiker dar.
Im Folgenden sollen verschiedene Welleneigenschaften mit Hilfe eines
Reflexklystrons untersucht und das Verhalten von Mikrowellen auf Hohlleitern
betrachtet werden.

\section{Grundlagen}
\label{sec:grundlagen}
\begin{wrapfigure}{r}{0.33\linewidth}
    \centering
    \includegraphics[width=0.9\linewidth]{img/hohlleiter.png}
    \caption{
        Der eransversal-elektrische $\text{TE}_{1,0}$-Modus in einem
        Rechteckhohlleiter, wie er in diesem Versuch untersucht wird
        \cite{V53}.
    }
    \label{fig:leiter}
\end{wrapfigure}
Als Mikrowellen wird elektromagnetische Strahlung im Frequenzbereich
von etwa \SI{300}{\mega \hertz} bis etwa \SI{300}{\giga \hertz} bezeichnet.
Niedrigere Frequenzen gehen über in Funkwellen während noch höhere Frequenzen
in den Bereich des infraroten Lichts reichen.

Elektromagnetische Wellen breiten sich im Vakuum kugelförmig aus, wobei ihre
Intensität mit $1 / r^2$ abnimmt.
Jede elektromagnetische Welle kann jedoch auch in verschiedenen metallischen
Wellenleitern transportiert werden, wobei -- je nach Leiter -- theoretisch
keine Verluste auftreten.
Der hier untersuchte Hohlleiter weist einen rechteckigen Querschnitt auf,
wie in Abbildung \ref{fig:leiter} dargestellt.
Leiter diese Art werden für Mikrowellen häufig verwendet.

\subsection{Moden}
\label{subse:moden}
Wie in einem Einfachen Hohlraumresonator können Wellen in einem Hohlleiter
konstruktiv oder destruktiv interferieren.
Ähnlich zu einer stehenden Welle in jenem Resonator kann sich dabei eine Welle
mit Bäuchen und Knoten zwischen den Leiterwänden ausbilden, die eine gewisse
Ausbreitungskomponente entlang der Leiterachse besitzt.
Für die Wellenlänge $\lambda$ der Mikrowellen muss dabei die Beziehung
\begin{equation*}
    \lambda \overset{!}{>} \lambda_\text{c} = 2a\,,
\end{equation*}
gelten, wobei $a$ den Abstand der Hohlleiterwänder bezeichnet und
$\lambda_\text{c}$ die Wellenlänge darstellt, unterhalb derer der Hohlleiter
keine Energie mehr transportiert -- die sogenannte Cut-Off-Wellenlänge.

Die fortlaufenden Wellen bilden unter dieser Voraussetzung verschiedene Moden,
die anhand der Ausprägung des elektrischen oder magnetischen Feldanteils
bezeichnet werden.
Bei transversal-elektrischen (TE-) Moden schwingt das elektrische Feld
senkrecht zur Ausbreitungsrichtung, während dies das magnetische Feld
bei transversal-magnetischen (TM-) Moden macht.
Die Anzahl der Schwingungsbäuche senkrecht zur Ausbreitungsrichtung $z$
entspricht dabei der Modenzahl $n$, $m$ in $x$- beziehungsweise $y$-Richtung
(siehe Abbildung \ref{fig:leiter}).

\subsection{Erzeugung von Mikrowellen mit einem Klystron}
\label{subsec:klystron}
Mikrowellen können mit Hilfe verschiedener Geräte erzeugt werden.
Das Klystron nutzt die von beschleunigten Elektronen abgegebene Strahlung.
Dafür werden die Elektronen resonant zwischen zwei Elektroden reflektiert
(Reflex-Klystron), wobei die abgegebene Bremsstrahlung im bereich der
Mikrowellen liegt.

Aus einer Kathode werden Elektronen emittiert und in Richtung eines positiv
geladenen Gitters beschleunigt.
Anschließend erreichen sie den Reflektor, bei dem sie auf Grund seiner
negativen Ladung abgebremst und in die entgegengesetzte Richtung beschleunigt
werden.
Wird nun eine periodische Spannung mit sich änderndem Vorzeichen zwischen
Reflektor und Gitter angelegt, beginnen die Elektronen zu schwingen und
passieren regelmäßig den Resonator.
Falls sie im Resonator abgebremst werden, geben sie Energie ab, die als
Mikrowellenstrahlung ausgekoppelt wird.
Dabei müssen tritt eine periodische Abhängigkeit der Leistung
von der Verweildauer der Elektronen vor dem Reflektor auf, die durch die
Reflektorspannung $V$ beeinflusst werden kann.
Zudem kann der Abstand der Reflektorplatte mechanisch verändert werden,
was einen größeren Einfluss auf die Frequenz hat, in der das Klystron schwingt.
Insgesamt ergeben sich somit zwei Möglichkeiten zur Beeinflussung der
Ausgangsleistung und -frequenz, die elektronische und mechanische Abstimmung.
Abbildung \ref{fig:aufbau} stellt den schematischen Aufbau des Klystrons dar.
Abbildung \ref{fig:output} zeigt die Ausgangsleistung in Abhängigkeit der
Reflektorspannung $V$.
\begin{figure}[p]
    \centering
    \includegraphics[width=0.9\linewidth]{img/aufbau.png}
    \caption{
        Schematische Darstellung des Aufbaus eines Reflex-Klystrons
        zur Erzeugung von Mikrowellenstrahlung.
    }
    \label{fig:aufbau}
\end{figure}
\begin{figure}[p]
    \centering
    \includegraphics[width=0.9\linewidth]{img/output.png}
    \caption{
        Ausgangsleistung des Klystrons in Abhängigkeit zur Reflektorspannung.
    }
    \label{fig:output}
\end{figure}

\clearpage
