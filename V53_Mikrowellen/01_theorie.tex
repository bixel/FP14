\section{Einleitung}
\label{sec:einleitung}
Als allgegenwärtige Anwendung von physikalischem Know-How ist die
Mikrowellentechnik kaum noch aus dem Alltag wegzudenken.
Von dem Haushaltsgerät Mikrowelle, das in wahrscheinlich jeder Küche der
westlichen Hemisphäre zu finden ist, bis zu Alarmanlagen wird diese Technik
in äußerst breit gefächerten Gebieten genutzt.

Ursprünglich schon zu Ende des zweiten Weltkrieges, als Frühwarnradar
entwickelt, wird Mikrowellenstrahlung schon einige Jahrzehnte eingesetzt.
Aus diesem Grund stellt sich die Untersuchung von Mikrowellen als interessante
Aufgabe für jeden Physiker dar.

Im Folgenden sollen verschiedene Welleneigenschaften mit Hilfe eines
Reflexklystrons untersucht und das Verhalten von Mikrowellen auf Hohlleitern
betrachtet werden.

\section{Grundlagen}
\label{sec:grundlagen}
Mikrowellenstrahlung bezeichnet elektromagnetische Strahlung im Frequenzbereich
von etwa \SI{300}{\mega \hertz} bis etwa \SI{300}{\giga \hertz}.
Niedrigere Frequenzen gehen über in Funkwellen während noch höhere Frequenzen
in den Bereich des infraroten Lichts reichen.
Elektromagnetische Wellen breiten sich im Vakuum kugelförmig
