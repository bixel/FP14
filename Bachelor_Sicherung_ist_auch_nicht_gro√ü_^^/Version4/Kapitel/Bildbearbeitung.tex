\section{Digitale Bildbearbeitung} % (fold)
\label{sec:bildbearbeitung}

Die Bilder des Mikroskops müssen zunächst in das HDF Format umformatiert werden, um effizienter bearbeitet werden zu können.
In der Kopfzeile werden die Bilder zunächst numeriert (\textit{micrograph ID}), sodass später eine Zuordnung möglich ist.
Anschließend wird die KTF der einzelnen Bilder analysiert und in der Kopfzeile als KTF-Objekt gespeichert.
Dafür wir ein KTF-Fit durch das 1D Powerspektrum des Bildes gefittet, wobei z.B. der Defokus durch die Fitparameter bestimmt wird.
Dabei wurde das Programm \textit{CTF-GUI} verwendet.
Dieses gibt neben den Fitparametern auch die dazugehörigen Bilder aus, welche bei der Qualitätskontrolle eine Rolle spielen.
Stimmt die Qualität nicht, kann das Bild sofort aussortiert werden.

Für die Extrahierung der Partikel im nächsten Schritt wird sowohl ein gaußförmiger Highpass-, als auch ein gaußförmiger Lowpassfilter auf eine Kopie der Bilder angewendet.
Dieser bewirkt, dass die besonders kleinen und großen Frequenzen (große und kleine Objekte) unterdrückt werden und die Strukturen der Proteine besser zu erkennen sind.

\FloatBarrier