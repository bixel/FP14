\section{Verbesserung der Auflösung} % (fold)
\label{sec:verbesserung_der_aufloesung}

Um nach einem erfolgreichen Iterationsschritt eine verbesserte Referenz zu erhalten ist es von Vorteil, die zuletzt erzeugte Rekonstruktion zu bearbeiten.
Dies kann durch Maskierung, Filterung, Symmetrisierung oder KTF-Korrektur möglich gemacht werden.
\\
\\
\textbf{Maskierung:} Bei einer 3D Rekonstruktion entstehen Fragmente (Rauschen) außerhalb des Dichtevolumens des Partikels. 
Diese können beim nächsten Iterationsschritt zu fehlerhaften Zuordnungen der Projektionsparameter führen.
Ein Entfernen des Rauschens ist durch das Maskieren des Partikels mit einer binären Maske möglich.
Wird das Volumen mit der Maske multipliziert bleibt nur der Teil übrig, der von der Maske umschlossen ist.
\\
\\
\textbf{Filterung:} Stimmen die Partikelprojektionsrichtungen nicht genau mit der Realität überein, kommt es zu Überlappungen der Dichten.
Besonders bei den großen Frequenzen (kleinen Strukturen) entsteht so ein Rauschen im Volumen.
Daher ist es von Vorteil diese aus der Dichte mithilfe eines Tiefpassfilters herauszufiltern, um das Rauschen zu unterdrücken.
\\
\\
\textbf{Symmetrisierung:} Viele Strukturen von Proteinen weisen Symmetrien auf.
Dabei treten folglich genau dieselben strukturellen Anordnungen auf, welche auch in der Dichte übereinstimmen sollten.
Es ist in solchen Fällen nicht nur möglich während des Rekonstruktionsprozesses die Partikel mit der Referenz zu vergleichen, sondern auch innerhalb der Rekonstruktion die symmetrischen Bereiche. 
Besitzt ein Partikel beispielsweise eine $C5$ Symmetrie und werden die symmetrischen Bereiche miteinander verglichen, entspricht dies der fünffachen Partikelanzahl.
Sind symmetriebrechende Abschnitte der Struktur vorhanden, ist es nötig diese zuvor mit einer extra angepassten Maske auszuschneiden.
Eine Symmetrisierung von nicht symmetrischen Bereichen führt zu einer Zerstörung der Struktur.
\\
\\
\textbf{KTF-Korrektur:} Die KTF beinhaltet alle Strukturinformationen des Partikels, doch sind besonders die hohen Frequenzen nicht einfach auszulesen.
Wird nicht viel Wert auf eine Hochauflösung gelegt ist es zum Beispiel möglich die KTF-Informationen ab der ersten Nullstelle abzuschneiden, sodass nur die großen Strukturen sichtbar werden.
Da jedoch für die Hochauflösung die großen Frequenzen relevant sind, ist es unter anderem möglich das Vorzeichen der negativen Amplituden zu ändern (\textit{phase flip}).
Dadurch werden die KTFs, welche durch verschiedene Defokus Werte erzeugt worden sind, einheitlich und eine Ausrichtung der Partikel wird erleichtert
Zudem können gesondert struktureigene Frequenzen verstärkt werden (Siehe \textbf{PDB Modell}).
Eine weitere Methode ist die Modifikation des B-Faktors (Vergleich Kapitel \ref{sec:elektron_spektroskopie}).
Durch Anhebung der Intensität aller Frequenzen wird auch die Information der großen Frequenzen verstärkt, jedoch verschlechtert sich das Signal-zu-Rausch Verhältnis, da bei der Verstärkung nicht zwischen Information und Rauschen unterschieden werden kann (\cite{aufloesung1},\cite{aufloesung2}).
\\
\\
\textbf{PDB Modell:}
Die PDB (\textit{Protein Data Bank}) beinhaltet 3D-Strukturdaten der verschiedensten Makromoleküle (\cite{pdb2}).
Dabei handelt es sich unter anderem um Untereinheiten größerer Proteine, deren Architektur meist mithilfe der Röntgenkristallographie entschlüsselt worden ist.

PDBs (Strukturen, welche aus der Datenbank entnommen wurden) helfen bei der Strukturaufklärung größerer Proteine.
Handelt es sich um Strukturen desselben Proteins, sollten diese bei hoher Auflösung nahezu perfekt in die Dichte der Rekonstruktion passen.
Meist genügt es jedoch, die homologe Struktur einer artverwandten Tierart zu betrachten, da diese sich häufig nur in einzelnen Aminosäuresequenzen unterscheidet.

Ist die Auflösung so hoch, dass bereits Strukturen in der Dichte erkennbar sind, ist es möglich die Röntgenstruktur mit dieser zu vergleichen.
Stimmen die Dichte und die Struktur zu großen Teilen überein, können die Frequenzen der Struktur extrahiert werden und bei einer KTF Korrektur der Partikel besonders verstärkt werden.

\FloatBarrier