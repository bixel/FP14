\begin{figure}
	\includegraphics[width = 14cm, height = 5cm]{Abbildungen/analyse.png}
	\caption[Vorgehensweise zur ersten 3D Rekonstruktion]{Vorgehensweise, um eine erste 3D Rekonstruktion zu erreichen.}
	\label{vorgehensweise}
\end{figure}


Ein Problem bei der Rekonstruktion eines 3D Modells mithilfe eines Transmissionselektronenmikroskops ist, dass sich nur 2D Abbildungen der 3D Elektronendichte erzeugen lassen.
Es ist daher eine Rückprojektion nötig, um das Objekt darstellen zu können.
Eine etablierte Methode für Rücktransformationen ist die Tomographie, doch ist diese bei biologischen Proben nicht vorteilhaft, da die relativ hohe Elektronendosis das Material zerstört und nicht alle Raumwinkel abgedeckt werden können.
Alternativ wird daher auf die Einzelpartikelanalyse zurückgegriffen.
Für diese ist es notwendig, dass die Probe ungeordnet vorliegt, sodass viele 2D Projektionen bei verschiedenen Projektionsparametern der Probe auftreten.
Diese sind neben den drei Euler-Winkeln $\Psi$, $\Phi$ und $\Theta$ auch die Verschiebung in x- und y-Richtung, da die ausgewählten Partikel nicht immer im Schwerpunkt des Bildes liegen.
Während bei der Tomographie diese bekannt sind, müssen bei dieser Methode diese erst iterativ berechnet werden.
Die Methode arbeitet nach dem in der Abbildung \ref{vorgehensweise} dargestellten Schema (\cite{einzelpartikel}).

Da die Anzahl der verwendeten Partikel mehrere Tausend umfasst, muss auf die Rechenleistung von Computern zurückgegriffen werden.
Eine Liste der verwendeten Systeme, Programme und Programmiersprachen ist in Tabelle \ref{EDV} angegeben.

Um eine effizientere Methode der Verarbeitung zu gewährleisten ist es nötig das Bildformat der Bilddateien von TIFF (\textit{tagged image file format}) auf HDF (\textit{hierarchical data format}) zu ändern.
Diese Art von Dateien besitzt eine Kopfzeile (\textit{header}), in der Bildinformationen (Projektionsparameter, Verschiebungen, Bildzugehörigkeit etc.) gespeichert werden können.
Zudem gibt es die Möglichkeit die Speicherung in einer Datenbank (BDB, \textit{Berkeley-Datenbank}) vorzunehmen.
In dieser sind Bild und Kopfzeile getrennt voneinander gespeichert, sodass diese getrennt geladen werden können um die Zugriffszeit zu minimieren.

In diesem Kapitel wird besonders auf die Datenauswertung der bei der Kryo-TEM entstandenen Daten eingegangen.
Bei der negativen Kontrastierung sind die Schritte analog. 
Der schematische Rekonstruktionsprozess kann wie folgt unterteilt werden:

\begin{enumerate}
	\item \textbf{Digitale Bildbearbeitung:} 
	Bestimmung des Defokus aus der KTF-Information; Aussortieren von Bildern (\textit{micrograph}); Filterung der Bilder
	\item \textbf{Extraktion der Partikel:} 
	Auswahl der Partikel aus den Bildern (\textit{particle picking}); Speichern der einzelnen Partikel in einem Datensatz (\textit{stack})
	\item \textbf{Ausrichtung und Klassifizierung:} 
	Zentrierung und Drehung der Partikel \\(\textit{alignment}); Zuordnung der Partikel in verschiedene Klassen; Aussortierung schlechter Einzelbilder
	\item \textbf{3D Rekonstruktion:} Iterative Bestimmung der Projektionsparameter; Rekonstruktion 
	\item \textbf{Verbesserung der Auflösung:} Filterung; Maskierung; Symmetrisierung; KTF Korrektur
	\item \textbf{PDB Modelle:} Bewertung der Auflösung; Bestimmung der Architektur
\end{enumerate}

\begin{table}
	\begin{tabular}[h!]{l l l}
		Name & Verwendung & Quelle \\
		\hline
		Eman/Eman2 & Prozessierung & http://blake.bcm.edu \\
		Sparx & Prozessierung & http://sparx-em.org \\
		CTF-GUI & KTF \& Defokus Bestimmung & MPI-Dortmund (R.Efremov,\\
		& & C.Gatsogiannis)\\
		GUI & Klassifizierung (K-Means) & MPI-Dortmund (C.Gatogiannis)\\ 
		USFC Chimera & 3D Modellierung & http://www.cgl.ucsf.edu/chimera \\
		Python & Programmierung & https://www.python.org \\
		CLB Cluster & Datenverarbeitung & MPI Dortmund \\
		\hline
		\hline
	\end{tabular}
	\caption[Verwendete EDV]{Verwendete Software, Programmiersprachen und Systeme.}
	\label{EDV}
\end{table}



\FloatBarrier