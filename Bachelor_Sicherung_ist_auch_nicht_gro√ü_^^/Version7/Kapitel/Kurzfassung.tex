\section*{Kurzfassung}
In der vorliegenden Arbeit ist die Struktur des Hämocyanins des Gemeinen Tintenfisches (\textit{Sepia Officinalis}) untersucht worden.
Dabei ist auf die Methode der Kryotransmissionselektronenmikroskopie zurückgegriffen worden.
Die notwendige Theorie wird erläutert.
Es wird auf die Probenpräparation mithilfe der negativen Kontrastierung und der Kryopräparation eingegangen, jedoch nicht auf die Aufreinigung des Proteins.
Da bisher nicht bekannt gewesen ist, ob das Protein eine Symmetrie aufweist, wird ein besonderes Augenmerk auf die Rekonstruktion von Proben mit nicht bekannter Symmetrie gelegt.
Dabei ist eine Einzelpartikelanalyse verwendet worden, wobei die Programme EMAN/EMAN2 und die Programmbibliothek SPARX benutzt worden sind.

Schlussendlich konnte eine Rekonstruktion bei einer Auflösung von ca. \unit[9.5]{\AA} erzeugt werden, mit deren Hilfe die Architektur bestimmt worden ist.
Innerhalb dieser Architektur konnten Symmetrien gefunden werden.

\section*{Abstract}
This bachelor thesis deals with the structure of the hemocyanin of the common cuttlefish (\textit{Sepia Officinalis}).
To solve the protein's structure cryoelectronmicroscopy has been used.
The necessary theory will be explained.
Furthermore, this thesis is about negative staining and cryopreparation, but not about the cleanup of the used protein.
As there is no known symmetry in this hemocyanin's structure, the focus is on the reconstruction of samples with unknown symmetry.
With the help of the programs EMAN/EMAN2 and the library SPARX a single particle analysis has been realized. 
Finally, a reconstruction with a resolution of approx. \unit[9.5]{\AA} has been received.
Consequently, the topology could be solved and symmetries have been found.