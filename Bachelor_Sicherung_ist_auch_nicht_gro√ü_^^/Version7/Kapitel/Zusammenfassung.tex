\chapter{Zusammenfassung und Ausblick}

In dieser Arbeit ist die Quartärstruktur des Hämocyanins des \textit{Sepia Officinalis} untersucht worden.
Dazu sind 2D Projektionen der Probe mithilfe der Kryotransmissionselektronenmikroskopie erstellt worden. 
Die zum Verständnis nötige Theorie wurde zudem erklärt.
Weiterhin ist auf die Einzelpartikelanalyse eingegangen worden, mit deren Hilfe durch digitale Prozessierung eine 3D Rekonstruktion erzeugt werden konnte.
Eine Auflösung von etwa \unit[9.3]{\AA} wurde erreicht, sodass bereits Details der Sekundärstruktur erkennbar sind.
Trennt man die Rekonstruktion in den inneren Kragen und die äußere Wand auf, hat die äußere Wand sogar eine Auflösung von ca. \unit[8]{\AA}, während der innere Kragen nur eine Auflösung von ca. \unit[10]{\AA} besitzt.
Damit ist diese Rekonstruktion besser aufgelöst als die zum Vergleich herangezogene aus den Veröffentlichungen (Name) (\cite{bla}) und (Name) (\cite{blubb}).\\


Des Weiteren sind die Ergebnisse mit denen aus den Veröffentlichungen von (Namen) (\cite{bla}, \cite{blubb}) verglichen worden.
Das Ergebnis aus der Veröffentlichung \cite{bla} konnte nicht bestätigt werden. Es hat sich herausgestellt, dass nicht das gesamte Protein $C5$ symmetrisch ist, sondern nur die äußere Wand.
Der innere Bereich hingegen besitzt insgesamt keine Symmetrie. Er lässt sich aber in einen $D1$ symmetrischen und einen unsymmetrischen Bereich unterteilen (Abbildung \ref{sym}).
Weiterhin konnte bestätigt werden, dass auch das Hämocyanin des \textit{Sepia Officinalis} die Struktur aus (Names) Veröffentlichung (\cite{blubb}) aufweist (Abbildung \ref{done}).
Um die Funktionsweise des Sauerstofftransports zu verstehen, war die Auflösung hinreichend.
\\
\\
Die Auswertung wurde bisher mit nur ca. 8000 Partikeln durchgeführt.
Um eine verbesserte Auflösung besonders im inneren Kragen zu erreichen, wäre es nötig eine Auswertung mit mehr Partikeln durchzuführen.
Da der innere Bereich einen $D1$ symmetrischen Bereich enthält, ist zu vermuten, dass auch die Wand des Proteins eine $D1$ Symmetrie besitzt.
Eine Berücksichtigung dieser bei der Rekonstruktion, könnte zusätzlich zu einer Verbesserung der Auflösung führen.
Ausgehend von einer höheren Auflösung kann auf die Funktionsweise zurückgeschlossen werden.