\section{Rekonstruktion} % (fold)
\label{sec:rekonstruktion}

Um eine erste 3D Rekonstruktion als Startreferenz zu erhalten, werden die erzeugten Klassensummen genutzt.
Deren Signal-zu-Rausch Verhältnis ist im Vergleich zu den Einzelpartikeln durch die Summation verbessert worden.
Mithilfe des SPARX-Programms \textit{sxviper.py} (Bisher unveröffentlicht) wird so ein erstes Modell erzeugt.
Als Inertialmodell wird hierbei eine zylindrische Struktur verwendet.

Eine iterative Findung der Projektionsparameter führt das SPARX-Programm \textit{sxali3d.py} durch.
Dafür überführt das Programm die 3D Referenz bei festen Winkelabschritten in 2D Projektionen und vergleicht diese mit den Partikeln (\textit{multi reference alignment}).
Die Partikel werden nach der am besten passenden Projektion ausgerichtet und die Projektionsrichtungen in die Kopfzeile geschrieben.
Aus den bestimmten Parametern wird erneut eine 3D Rekonstruktion mithilfe der Partikel erstellt und als Referenz an den nächsten Iterationsschritt übergeben.
Dabei wird die Größe der Winkelabschritte stetig verringert, sodass mehr Projektionsrichtungen möglich sind.

Um die Auflösung einer Rekonstruktion zu bestimmen ist das FSK$_{0.5}$-Kriterium (Fourier Schalen Korrelation, engl.:\textit{Fourier shell correlation}) angewendet worden.
Dazu wird der Datensatz der verwendeten Partikel in zwei gleich große Teile aufgespalten und von jedem eine Rekonstruktion erstellt.
Diese werden mithilfe einer Kreuzkorrelation miteinander verglichen und die Übereinstimmung in Abhängigkeit der Frequenz dargestellt.
Die Frequenz, bei der noch \unit[50]{\%} der Rekonstruktionen übereinstimmen, stellt die Auflösungsgrenze dar.

\FloatBarrier