\section{Probenpräparation für die negative Kontrastierung} % (fold)
\label{sec:probenpr_aparation_mit_negativer_kontrastierung}

Um Proben mit einem Transmissionselektronenmikroskop untersuchen zu können, müssen zunächst ein paar Vorbereitungen getroffen werden.
Die Probe muss strukturerhaltend auf dem Probenträger fixiert werden, damit diese sich nicht bewegen kann.
Dies hat zudem der Vorteil, dass die Probe nicht in das Vakuum des Mikroskops evaporieren kann.
Zudem muss ein Kontrast vorhanden sein, um Strukturen der Probe erkennen zu können.

Die Methode der negativen Kontrastierung ist eine Möglichkeit zum erfüllen dieser Anforderungen.
Dabei wird die Probe zusammen mit einer Schwermetalllösung (\unit[0.75]{\%} Uranylformiat) auf einem Probenträger aufgetragen.
Dieser ist ein (größe hier einfügen) großes, rundes Kupfergitter (G2400C, Plano GmbH).
Die Probe wird zur Verarbeitung in einem Puffer auf die benötigte Konzentration verdünnt (TRIS \unit[50]{mM} pH 7.4, NaCl \unit[150]{mM}, MgCl$_2$ \unit[5]{mM}, CaCl$_2$ \unit[5]{mM}).
Nach der Benetzung mit einem Polymerfilm wird das Gitter mit Kohlenstoff bedampft, auf dem die Probe später haften wird.
Das Gitter wird in ein Plasma eingebracht, das aus beschleunigten geladenen und ungeladenen Atomen und Molekülen besteht.
Durch die Wechselwirkung entstehen Ionen auf der Oberfläche und diese wird polarisiert.
Dies ist notwendig, da die Schwermetall- und Probenlösung hydrophil (polar) ist und die Probe ansonsten nicht haftet.
Nach dem Auftragen der Probe wird diese gewaschen, mit der Schwermetalllösung benetzt und anschließend luftgetrocknet.

Die negative Kontrastierung ist eine gute Möglichkeit zur Kontrastverstärkung in einem Transmissionselektronenmikroskop bei Proben, die primär aus leichten Elementen aufgebaut sind.
Wie in Kapitel \ref{sec:elektron_spektroskopie} beschrieben streuen größere Atomkerne im Vergleich vermehrt elastisch, während kleinere vermehrt inelastisch streuen.
Biologische Proben bestehen aus leichten Elementen (Kohlenstoff, Stickstoff, Sauerstoff etc.), während Schwermetalle schwere Elemente beinhalten.
Der auftretende Intensitätsunterschied führt in dem erzeugten Bild zu einem Amplitudenkontrast.
Da an der Lösung und nicht an der Probe gestreut worden ist, wird diese Methode als negative Kontrastierung bezeichnet.

Die Nachteile der negativen Kontrastierung wirken sich negativ auf die Auflösung aus.
Durch das Trocknen der Probe kann es zu Deformationen der nativen Konformation und Artefakten der Kontrastlösung kommen.
Dies führt besonders bei großen Molekülen zu deutlichen Fehlern in der Rekonstruktion.
Auch gibt es Probleme bei inneren Strukturen. 
Die Schwermetalle kommen nicht in die Proteine und an diesen Stellen ist der Kontrast kaum bis gar nicht vorhanden.

\FloatBarrier