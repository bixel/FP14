\section{Rekonstruktion} % (fold)
\label{sec:rekonstruktion}

Um eine erste 3D Rekonstruktion als Startreferenz zu erhalten, werden die erzeugten Klassensummen genutzt.
Deren Signal-zu-Rausch Verhältnis ist im Vergleich zu den Einzelpartikeln durch die Summation verbessert worden.
Mithilfe des Sparx-Programms \textit{sxviper.py} (Bisher unveröffentlicht) ist so ein erstes Modell erzeugt worden.
Als Inertialmodell wurde hierbei eine zylindrische Struktur verwendet.

Eine iterative Findung der Projektionsparamter führt das Sparx-Programms \textit{sxali3d.py} durch.
Dafür wird die 3D Referenz bei festen Winkelabschnitten in 2D Projektionen überführt und mit den Partikeln verglichen (\textit{multi reference alignment}).
Die Partikel werden nach der am besten passenden Projektion ausgerichtet und die Projektionsrichtungen in die Kopfzeile geschrieben.
Aus den bestimmten Parametern wird erneut eine 3D Rekonstruktion aus den Partikeln erstellt und als Referenz an den nächsten Iterationsschritt übergeben.
Dabei werden die Größe der Winkelabschnitte stetig verringert, sodass mehr Projektionsrichtungen möglich sind.

Um die Auflösung einer Rekonstruktion zu bestimmen ist das FSK$_{0.5}$-Kriterium angewendet worden.
Dazu wird der Datensatz der verwendeten Partikel in zwei gleich große Teile aufgespalten und von jedem eine Rekonstruktion erstellt.
Diese werden Mithilfe einer Kreuzkorrelation miteinander Verglichen und die Übereinstimmung in Abhängigkeit der Frequenz dargestellt.
Als noch Aufgelöst wird die Frequenz bezeichnet, bei der die Übereinstimmen noch \unit[50]{\%} beträgt.

\FloatBarrier