\section{Extraktion der Partikel} % (fold)
\label{sec:extraktion_der_partikel}

(Wrapfigure: Oben micrograph, da unter 3 einzelapartikel)

Aus den bearbeiteten Bildern müssen nun die einzelnen Partikel extrahiert werden.
Dafür ist das EMAN2 Programm \textit{e2boxer.py} verwendet worden.
Es wird dafür jeder Partikel auf jedem Bild einzeln ausgewählt und die Koordinaten der Partikelmittelpunkte in einer Textdatei gespeichert.
Anschließend werden die Koordianten auf den ungefilterten Datensatz angewendet und von dort extrahiert.
Dabei ist eine quadratische Boxgröße mit einer Seitenlänge von \unit[512]{px} bei einer Pixelgröße von \unit[0.83]{\AA} gewählt worden.
Diese ist nur knapp größer als das Protein und zudem ist die Boxgröße von \unit[256]{px} eine von EMAN2 empfohlene Boxgröße um eine erste Rekosntruktion mit einem verkleinerten Datensatz durchzuführen.
Zudem werden die extrahierten Partikel in einem HDF Datensatz (\textit{stack}) gespeichert.
Abschließend wird der Datensatz normalisiert, invertiert und KTF-Korrigiert.

\FloatBarrier