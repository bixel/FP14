\section{Transmissionselektronenmikroskopie} % (fold)
\label{sec:elektron_spektroskopie}

\begin{wrapfigure}{r}{7cm}
	\centering
	\includegraphics[width = 7cm]{Abbildungen/TEM.png}
	\caption[Schematische Aufbau eines Transmissionselektronenmikroskop]{Schematische Aufbau eines Transmissionselektronenmikroskop. (Quelle Benutzerhandbuch, JEOL Ldt.)}
	\label{TEM}
\end{wrapfigure}

Das Prinzip der Bildentstehung bei einem Transmissionselektronenmikroskop ist dem eines Lichtmikroskops sehr ähnlich:
Ein Elektronenstrahl erzeugt eine Projektion der Probe im Realraum.
Im Vergleich zum Lichtmikroskop werden jedoch keine optischen, sondern elektrische Linsen verwendet.
Ein schematischer Aufbau der verwendeten Mikroskope ist in Abbildung \ref{TEM} dargestellt.
Die dazugehörigen technischen Daten können der Tabelle (Tabellen ref hier einfügen) entnommen werden.

Als Elektronenquelle wird je nach Mikroskop entweder ein LaB$_6$-Kris\-tall (Lanthanhexaborid) oder eine Feldemissionsquelle genutzt.
Bei der Feldemission ($\Delta$E = \unit[0.6-1.2]{eV}) ist ein schmaleres Energiespektrum der Elektronen zu beobachten als beim LaB$_6$-Kris\-tall ($\Delta$E = \unit[1.5-3]{eV}), wodurch ein höheres Aufl"osungsverm"ogen ermöglicht wird.
Nach dem Beschleunigen der Elektronen werden diese mithilfe von elektronischen Linsen auf die Probe und anschließend auf die Kamera fokusiert.
Diese Linsen bestehen aus Spulen, die Magnetfelder erzeugen.
Die bewegten Elektronen werden von diesen durch die Lorentz-Kraft abgelenkt ohne deren Energie zu beeinflussen (Nur die senkrechte Komponente der Kraft wirkt auf die Elektronen).
Um Wechselwirkungen der Elektronen mit anderer Materie als der Probe zu verhindern, ist das gesamte Spulensystem evakuiert.
Abschließend wird das Transmissionsbild im Objektsystem vergrößert und auf einem Fluoreszensschirm abgebildet oder direkt mit einer Kamera oder einem Photofilm detektiert.
Abhängig von der Intensität erzeugt dieser ein Bild mit Graustufen.
Je dunkler der Bildpunkt ist, umso größer ist die Intensität.

Damit auf dem Bild Strukturen erkennbar werden, muss ein ausgeprägter Amplitudenkontrast zwischen Probe und Lösung vorhanden sein.
Eine Schwierigkeit tritt dabei bereits dadurch auf, dass die Elektronendosis bei biologischen Proben nicht zu groß gewählt werden darf um Schäden zu vermeiden.
Der Amplitudenkontrast kann sowohl durch Teilchen- als auch durch Welleneigenschaften der Elektronen entstehen. 
Bei der Betrachtung als Teilchen ergiebt sich der Kontrast durch Streueffekte.
Dafür dürfen die Ordnungszahlen der Elemente in der Probe und der Lösung nicht in derselben Größenordnung sein.
Dies ist auf die Wechselwirkung der Elektronen an Atomkernen zurückzuführen.
An großen Kernen werden diese vermehrt elastisch gestreut, während an kleinen Kernen bevorzugt inelastische Streuung auftritt.
Erstere führt zu einer Richtungsänderung der Elektronen.
Mithilfe einer Blende werden diese aus dem Elektronenstrahl entfernt.

Die inelastisch gestreuten Elektronen tragen jedoch nur zum Rauschen im Bild bei und tragen keine Information.
Der Amplitudenkontrast kommt dementsprechend durch die geringere Intensität bei Elementen größerer Ordnungszahl zustande.

Befinden sich nun die Elemente der Probe und der Lösung in derselben Größenordnung, werden an beiden etwa gleich viele Elektronen elastisch gestreut und es ist kein Kontrast zu erkennen.
Wird das Elektron jedoch als Welle betrachtet kann eine andere Art von Kontrast hergeleitet werden.
Wenn elektromagnetische Wellen mit Materie wechselwirken, kommt es zu einer Phasenverschiebung.
Dies führt wiederum zu Interferenzerscheinungen zwischen den gebeugten und den ungebeugten Elektronen und wird als Phasenkontrast bezeichnet.
Die Verschiebung wird durch die Gleichung \eqref{Phase}(Quelle) beschrieben.

\begin{equation}
	\gamma(k,\Delta z) = \frac{\pi}{2}\left(\lambda^3 C_s k^4 - 2\lambda \Delta z k^2\right). \label{Phase}
\end{equation}

Dabei ist $\Delta z$ der Unterfokus, $C_s$ die mikroskopspezifische sph"arischen Aberationskonstante, $\lambda$ die Wellenl"ange der Elektronen und $k$ der reziproke Abstand vom Bildmittelpunkt im Frequenzraum.
Biologische Proben führen nur zu einem schwachen Phasenkontrast, weshalb der Unterfokus relativ hoch gewählt werden muss.
Zur Rückrechnung auf den Amplitudenkontrast wird die Kontrast Transfer Funktion (KTF, Gleichung \eqref{KTF}, Quelle) verwendet, die sowohl Anteile von Phasen-, als auch Amplitudenkontrast beinhaltet.
Sie ist die Fouriertransformierte der Punktspreizfunktion, die angibt, wie ein punktförmiges Objekt durch das System dargestellt wird.

\begin{equation}
	\text{KTF}(\gamma(k,\Delta z)) = \sqrt{1-A^2} \sin[\gamma(k,\Delta z)] - A \cos[\gamma(k,\Delta z)]. \label{KTF}
\end{equation}

Ein Beispiel für eine KTF ist in Abbildung \ref{KTF_pic} dargestellt.
Die Funktion alterniert zwischen $+1$ und $-1$, wobei die Nullstellen vom Unterfokus abhängen.
An den Nullstellen gibt es keinen Kontrast und das Bild enthält keine Informationen bei den entsprechenden Frequenzen.
Es ist daher wichtig während der Bildaufnahme verschiedene Defokuswerte zu wählen, um möglichst alle Frequenzen im gesamten Fourierraum abzudecken.
Die Funktion wird durch eine einhüllende Funktion gedämpft und ist auf Störungen (Drift, Energieaufspaltung im Strahl usw.) zurückzuführen.
Beschrieben wird diese dabei durch den Exponentialkoeffizienten (B-Faktor) der approximierten Gaußfunktion.

\begin{figure}[h!]
\includegraphics[width = 15cm]{Abbildungen/KTF.png}
\caption[Beispiel der Kontrast Transfer Funktion]{Beispiel einer Kontrast Transfer Funktion. Links ist der theoretische, rechts ist der reale Verlauf dargestellt. Der reale Verlauf wird von einer  dämpfenden Funktion eingehüllt. (Quelle Behrmann,
2012)}
\end{figure}

Für das Auflösungsvermögen eines Mikroskops gilt: Zwei Gegenstände können nicht voneinander getrennt dargestellt werden, wenn das Objekt kleiner ist, als die Wellenlänge der eingesetzten Strahlung.
Bei Elektronen, die mit einer Beschleunigungsspannung von \unit[300]{kV} beschleunigt werden, liegt diese bei $\lambda = \unit[0.02]{\mathring{A}}$.
Die wichtigsten begrenzenden Faktoren für die Auflösungsgrenze sind das Abbe-Kriterium, Abbildungsfehler und die Pixelgröße des Photochips.
Das Abbe-Kriterium ergiebt sich aus der Gleichung \eqref{Abbe} (Quelle).

\begin{equation}
	d = 0.61 \frac{\lambda}{n \sin(\alpha)} \label{Abbe}
\end{equation}

Es ist abhängig von der Brechzahl $n$ des Mediums zwischen Objektiv und Probe und dem halben Öffnungswinkel des Objektivs (Quelle).
Abbildungsfehler, wie die spährische Aberation, führen zu Anteilen in der Kontrast Transfer Funktion (Gleichung \eqref{KTF}).
Bei höheren Frequenzen führt der geringer werdende Abstand zwischen zwei Nulldurchgängen dazu, dass das Signal nicht mehr fehlerfrei ausgewertet werden kann.
Auch ein driften der Probe während der Aufnahme führt zu einer Abschwächung des Signals.
Zudem spielt die Pixelgröße $A_{px}$ des Photochips eine Rolle, das Abtasttheorem (Quelle) besagt, dass die maximale Auflösung dem doppelten Wert der Pixelgröße entspricht.
Eine Fouriertransformation überträgt die Bildinformation in den Frequenzraum, in dem die Intensität bei verschiedenen Frequenzen $f_s$ (spartial frequency).
Die Frequenz der maximalen Auflösung wird als Nyquist-Frequenz bezeichnet.
Es wird meist die Einheit der absoluten Frequenz $f_a$ verwendet, die eine Verbindung zwischen der Pixelgröße und der Frequenz durch Gleichung \eqref{freq} erstellt.

\begin{equation}
	f_a = A_{px} / f_s \label{freq}
\end{equation}

Es ist zu erkennen, dass dieser Wert nicht kleiner als $0.5$ werden sollte, da dies der Nyquist Frequenz entspricht und eine höhere Auflösung nicht möglich ist.

\FloatBarrier