\chapter{Messungen}
\label{c:Messungen}

Die Messanordnung ist in Abbildung \ref{f:Aufbau} dargestellt und die Messwerte sind in Tabelle \ref{t:Messwerte} aufgef\"uhrt. Die h\"ochste Geschwindigkeit betr\"agt \unit[53]{km/h}. Messwerte mit Einheit bitte immer in der
Form '$\mathrm{\backslash unit[Zahl]\{Einheit\}}$' angeben.

Bei der grafischen Darstellung von Daten
     sollte man auf gut unterscheidbare Farben, ausreichende
     Strichdicken f"ur Linien, Symbole und Umrandungen sowie auf die
     Gr"o"se der Achsenbeschriftungen achten, siehe Abbildung \ref{f:plot}. Man bedenke, dass
     Abbildungen vielfach beim Einbinden in die Arbeit  verkleinert
     werden. Au"serdem wird man viele Abbildungen auch in der
     Pr"asentation zum Bachelor-Vortrag wiederverwenden wollen. Bei der
   Projektion mit einem Beamer wird dann eine zu geringe Strichdicke
   praktisch unsichtbar


\begin{figure}[!h]
 \begin{center}
   \includegraphics[width=0.5\textwidth]{Abbildungen/Aufbau.pdf}
   \caption[Messanordnung]{Messanordnung. Hier kann die Abbildung noch erl\"autert werden, ohne dass der Text im Abbildungsverzeichnis auftaucht.}
   \label{f:Aufbau}
 \end{center}
\end{figure}

\begin{figure}[!h]
 \begin{center}
   \includegraphics[width=0.5\textwidth]{Abbildungen/Strichdicken.pdf}
   \caption[Graphik]{Beispielgraphik mit gut unterscheidbaren Farben, ausreichenden
     Strichdicken f"ur Linien, Symbole und Umrandungen sowie angepasster
     Gr"o"se der Achsenbeschriftungen. }\label{f:plot}
   \label{f:Aufbau}
 \end{center}
\end{figure}


\begin{table}[!h]
\caption{aufgenommene Messwerte}
\label{t:Messwerte}
\vspace{2pt}
\begin{center}
\begin{tabular}{|l c|}
\hline
%{\small
Kennzeichen & Geschwindigkeit\\
& [km/h] \\
\hline
\hline
DO-XX XXXX & 51\\
EN-XX  XXX & 42\\
DO-X  XXXX & 50\\
UN-XX XXXX & 53\\
\hline
\end{tabular}
\end{center}
\end{table} 