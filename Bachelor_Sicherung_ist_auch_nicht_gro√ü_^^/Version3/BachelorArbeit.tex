\documentclass[11pt,a4paper,twoside]{report}
\usepackage{german}
\usepackage{graphicx}
\usepackage{Tex/nicehead}
\usepackage{epsfig}
\usepackage{amssymb}
\usepackage{amsmath}
\usepackage{tabularx}
\usepackage{calc}
\usepackage[vflt]{floatflt}
\usepackage{units}
\usepackage{upgreek}
\usepackage[pdfborder={0 0 0}, hypertexnames=false]{hyperref}
\usepackage[official]{eurosym}
\usepackage{fontspec}
\usepackage{wrapfig}
\usepackage[]{placeins}

\newcommand{\mytil}{{\raise.17ex\hbox{$\scriptstyle\mathtt{\sim}$}}} 

% Seitenstil
\pagestyle{emheadings}

% Breite des Textblocks und der Raender
\setlength{\evensidemargin}{0mm}
\setlength{\oddsidemargin}{13mm}
\setlength{\textwidth}{145mm}

\setcounter{secnumdepth}{3}   % Tiefe der Kapitelnummerierung
\setcounter{tocdepth}{3}      % Tiefe der Kapitelnummerierung im Inhaltsverzeichnis

\setcounter{totalnumber}{3}
\renewcommand{\floatpagefraction}{0.99}


\mathcode`\,="013B
\begin{document}
\pagenumbering{Roman}

% Anmerkung: Die Seitenraender wurden asymmetrisch gewaehlt,
%            damit genug Platz fuer eine Klemmbindung da ist.
%            %
%            Die Seitenraender koennen in der Datei Tex/global.tex
%            veraendert werden.

% >>> Titelseite <<<

\newcommand{\thetitle}{Titel der Bachelorarbeit}

\thispagestyle{empty}
\begin{center}
\Huge\textbf{\thetitle}
\vfill
\Large
Bachelorarbeit \\ zur Erlangung des akademischen Grades \\ Bachelor of Science \\
\vspace{20pt}
\normalsize
vorgelegt von \\[5pt]
{\Large Markus Stabrin} \\[5pt]
geboren in Hamburg \\
\vspace{20pt}
Angefertigt am Max-Planck-Institut f\"ur molekulare Physiologie in Dortmund \\ Geschrieben am Lehrstuhl f\"ur Experimentelle Physik I \\ Fakult\"at Physik \\
Technische Universit\"at Dortmund \\ 2014
\end{center}
\newpage

% >>> Gutachterseite <<<

\thispagestyle{empty}
\vspace*{\fill}
\begin{tabbing}
1. Gutachter : \=\kill
1. Gutachter : \>Prof. Dr. Metin Tolan \\[11pt]
2. Gutachter : \>Prof. Dr. Stefan Raunser \\[11pt]
\end{tabbing}
\vspace{11pt}
Datum des Einreichens der Arbeit: 09.\,07.\,2014
\newpage

% >>> Kurzfassung/Abstract <<<

\thispagestyle{empty}
\section*{Kurzfassung}
In der vorliegenden Arbeit ist die Struktur des Hämocyanins des Gemeinen Tintenfisches (\textit{Sepia Officinalis}) untersucht worden.
Dabei ist auf die Methode der Kryotransmissionselektronenmikroskopie zurückgegriffen worden.
Die notwendige Theorie wird erläutert.
Es wird auf die Probenpräparation mithilfe der negativen Kontrastierung und der Kryopräparation eingegangen, jedoch nicht auf die Aufreinigung des Proteins.
Da bisher nicht bekannt gewesen ist, ob das Protein eine Symmetrie aufweist, wird ein besonderes Augenmerk auf die Rekonstruktion von Proben mit nicht bekannter Symmetrie gelegt.
Dabei ist eine Einzelpartikelanalyse verwendet worden, wobei die Programme EMAN/EMAN2 und die Programmbibliothek SPARX benutzt worden sind.

Schlussendlich konnte eine Rekonstruktion bei einer Auflösung von ca. \unit[9.5]{\AA} erzeugt werden, mit deren Hilfe die Architektur bestimmt worden ist.
Innerhalb dieser Architektur konnten Symmetrien gefunden werden.

\section*{Abstract}
This bachelor thesis deals with the structure of the hemocyanin of the common cuttlefish (\textit{Sepia Officinalis}).
To solve the protein's structure cryoelectronmicroscopy has been used.
The necessary theory will be explained.
Furthermore, this thesis is about negative staining and cryopreparation, but not about the cleanup of the used protein.
As there is no known symmetry in this hemocyanin's structure, the focus is on the reconstruction of samples with unknown symmetry.
With the help of the programs EMAN/EMAN2 and the library SPARX a single particle analysis has been realized. 
Finally, a reconstruction with a resolution of approx. \unit[9.5]{\AA} has been received.
Consequently, the topology could be solved and symmetries have been found.
\newpage

% >>> Hauptteil <<<

\addcontentsline{toc}{chapter}{Inhaltsverzeichnis}
\tableofcontents\newpage
\addcontentsline{toc}{chapter}{Abbildungsverzeichnis}
\listoffigures\newpage
\addcontentsline{toc}{chapter}{Tabellenverzeichnis}
\listoftables\newpage

\setcounter{page}{0}
\pagenumbering{arabic}

\chapter{Einleitung}
\label{einleitung}

Hier folgt eine kurze Einleitung in die Thematik. 

Hier folgt eine kurze Einleitung in die Thematik.

Die Einleitung muss kurz sein, damit die vorgegebene Gesamtl"ange der
Arbeit von 25 Seiten nicht "uberschritten wird. Die Beschr"ankung der
Seitenzahl sollte man ernst nehmen, da "uberschreitung zu Abz"ugen in
der Note f"uhren kann. Selbstverst"andlich kann jede(r) f"ur sich eine
Version der Arbeit mit beliebig vielen und langen Anh"angen und
Methodenkapiteln erstellen. Das ist dann aber Privatvergn"ugen; die
einzureichende und zu beurteilende Arbeit muss der
L"angenbeschr"akung gen"ugen. Um der L"angenbeschr"ankung zu gen"ugen,
darf auch nicht an der Schriftgr"o"se, dem Zeilenabstand oder dem
Satzspiegel (bedruckte Fl"ache der Seite) manipuliert werden.

\FloatBarrier
\chapter{Theoretische Grundlagen und Messvorbereitung}
\section{H"amocyanin}

(Abbildung)

Mit einer Größe von \unit[3-8]{MDa} gehören Hämocyanine zu den größten bekannten Proteinen der Natur (Habe im Evolution of Hämocyanin structure gesucht, aber nichts über die Größe direkt gefunden).
Diese transportieren sowohl den Sauerstoff von marin lebenden wirbellosen Tieren als auch einiger Spinnenarten.
Sie liegen ohne an Blutzellen gebunden zu sein direkt in der Hämolymphe dieser Tierarten vor.
In den aktiven Zentren bindet der Sauerstoff an Kupfer und oxidiert diesen von Cu(I) zu Cu(II).
Dieses emittiert, durch Kupfer-Peroxo-Komplexbildung mit dem Sauerstoff, Lichtquanten im sichtbaren Wellenlängenbereich von \unit[480-420]{nm}. (http://www.chemieunterricht.de/dc2/komplexe/farbe.html)(http://archimed.uni-mainz.de/pub/2001/0109/diss.pdf)
Damit sind Lebewesen mit Hämocyaninen Blaublüter.

Hämocyanine besitzen mit einem Radius von ca. \unit[35]{nm} und einer Höhe von ca. \unit[15]{nm} eine hohlzylindrische Struktur (Quelle Paper von Julian).
Sie bestehen aus 10 strukturell ähnlichen Untereinheiten, und sind somit Decamere (Oligomere bestehend aus 10 Untereinheiten) (Quelle Paper Julian). 
Jede Untereinheit besteht weiterhin aus sieben bis acht funktionellen Einheiten (FE), die artspezifisch sind.
Das Hämocyanin des \textit{Sepia Officinalis} (\textit{gemeiner Tintenfisch}), das in dieser Arbeit untersucht wird, besteht beispielsweise aus acht dieser FEs (benannt N' term-a-b-c-d-d'-e-f-g-C' term) (Quelle Paper von Julian).
Jede FE ist ca. \unit[50]{kDa} schwer, wodurch das gesamte Protein ein Gewicht von ca. \unit[4]{MDa} besitzt.

Die FEs FE-a,FE-b,FE-c,FE-d,FE-e und FE-f sind bei allen bekannten Hämocyaninen vorhanden.
Dabei ähnelt sich deren Architektur und es liegt nahe, dass sich diese FEs vor der Aufspaltung der einzelnen Arten entwickelt haben.
Daher ist anzunehmen, dass diese in der Struktur dieselben Positionen besetzen und somit dieselbe Rolle beim Bilden der Quartärstruktur ihres Proteins spielen, obwohl dies noch nicht direkt gezeigt werden konnte (Habe die Paper gesucht aber nicht gefunden, waren das die, die du mir noch schicken wolltest?).
FE-d' hingegen entwickelte sich vor etwa 740 Millionen Jahren durch eine Duplizierung der FE-d.

Geometrisch betrachtet besteht der Hohlzylinder aus zwei kovalent miteinander verbundenen Bereichen, der äußeren Wand und dem inneren Kragen.
Die Wand ist charakteristisch für alle Hämocyanine und setzt sich aus den oben beschriebenen FEs a,b,c,d,e und f zusammen.
Der Kragen hingegen ist bei jeder Hämocyaninart anders aufgebaut und besteht beim \textit{Sepia Officinalis} aus FE-d' und FE-g.

Da die Funktionsweise von Proteinen durch ihre Quartärstruktur festgelegt ist, ist es notwendig eine Strukturanalyse durchzuführen.
Eine Möglichkeit wäre die hochauflösende Kristallographie, doch lassen sich besonders große Proteine nur sehr schwer kristallisieren.
Zudem liegen diese nicht mehr in ihrer nativen Konformation vor, wie es in Lösung der Fall ist.
Daher bietet sich die Kryotransmissionselektronenmikroskopie an, welche eine Hochauflösung der Struktur ermöglicht.

\FloatBarrier
\section{Transmissionselektronenmikroskopie} % (fold)
\label{sec:elektron_spektroskopie}

Das Prinzip der Bildentstehung bei einem Transmissionselektronenmikroskop ist dem eines Lichtmikroskops sehr ähnlich:
Ein Elektronenstrahl erzeugt eine Projektion der Probe im Realraum.
Im Vergleich zum Lichtmikroskop werden jedoch keine optischen, sondern elektrische Linsen verwendet.
Ein schematischer Aufbau der verwendeten Mikroskope ist in Abbildung \ref{TEM} dargestellt.
Die dazugehörigen technischen Daten können der Tabelle (\ref{TEM_tab}) entnommen werden.

Als Elektronenquelle wird je nach Mikroskop entweder ein LaB$_6$-Kris\-tall (Lanthanhexaborid) oder eine Feldemissionsquelle genutzt.
Bei der Feldemission ($\Delta$E = \unit[0.6-1.2]{eV}) ist ein schmaleres Energiespektrum der Elektronen zu beobachten als beim LaB$_6$-Kris\-tall ($\Delta$E = \unit[1.5-3]{eV}), wodurch ein höheres Aufl"osungsverm"ogen ermöglicht wird.
Nach dem Beschleunigen der Elektronen werden diese mithilfe von elektronischen Linsen auf die Probe und anschließend auf die Kamera fokusiert.
Diese Linsen bestehen aus Spulen, die Magnetfelder erzeugen.
Die bewegten Elektronen werden von diesen durch die Lorentz-Kraft abgelenkt ohne deren Energie zu beeinflussen (Nur die senkrechte Komponente der Kraft wirkt auf die Elektronen).
Um Wechselwirkungen der Elektronen mit anderer Materie als der Probe zu verhindern, ist das gesamte Spulensystem evakuiert.
Abschließend wird das Transmissionsbild im Objektsystem vergrößert und auf einem Fluoreszensschirm abgebildet oder direkt mit einer Kamera oder einem Photofilm detektiert.
Abhängig von der Intensität erzeugt dieser ein Bild mit Graustufen.
Je dunkler der Bildpunkt ist, umso größer ist die Intensität.

\begin{wrapfigure}{r}{7cm}
	\centering
	\includegraphics[width = 7cm]{Abbildungen/TEM.png}
	\caption[Schematische Aufbau eines Transmissionselektronenmikroskop]{Schematische Aufbau eines Transmissionselektronenmikroskop. (Entnommen aus Benutzerhandbuch JEOL Ldt.)}
	\label{TEM}
\end{wrapfigure}

Damit auf dem Bild Strukturen erkennbar werden, muss ein ausgeprägter Amplitudenkontrast zwischen Probe und Lösung vorhanden sein.
Eine Schwierigkeit tritt dabei bereits dadurch auf, dass die Elektronendosis bei biologischen Proben nicht zu groß gewählt werden darf um Schäden zu vermeiden.
Der Amplitudenkontrast kann sowohl durch Teilchen- als auch durch Welleneigenschaften der Elektronen entstehen. 
Bei der Betrachtung als Teilchen ergiebt sich der Kontrast durch Streueffekte.
Dafür dürfen die Ordnungszahlen der Elemente in der Probe und der Lösung nicht in derselben Größenordnung sein.
Dies ist auf die Wechselwirkung der Elektronen an Atomkernen zurückzuführen, da an großen Kernen diese vermehrt elastisch gestreut werden, während an kleinen Kernen bevorzugt inelastische Streuung auftritt.
Ersteres führt zu einer Richtungsänderung der Elektronen.
Mithilfe einer Blende werden diese aus dem Elektronenstrahl entfernt.

Die inelastisch gestreuten Elektronen tragen jedoch nur zum Rauschen im Bild bei und tragen keine Information.
Der Amplitudenkontrast kommt dementsprechend durch die geringere Intensität bei Elementen größerer Ordnungszahl zustande.

Befinden sich nun die Elemente der Probe und der Lösung in derselben Größenordnung, werden an beiden etwa gleich viele Elektronen elastisch gestreut und es ist kein Kontrast zu erkennen.
Wird das Elektron jedoch als Welle betrachtet kann eine andere Art von Kontrast hergeleitet werden.
Wenn elektromagnetische Wellen mit Materie wechselwirken, kommt es zu einer Phasenverschiebung.
Diese führt wiederum zu Interferenzerscheinungen zwischen den gebeugten und den ungebeugten Elektronen und wird als Phasenkontrast bezeichnet.
Die Verschiebung wird durch die Gleichung \eqref{Phase} (\cite{scherzer}) beschrieben.

\begin{equation}
	\gamma(k,\Delta z) = \frac{\pi}{2}\left(\lambda^3 C_s k^4 - 2\lambda \Delta z k^2\right). \label{Phase}
\end{equation}

Dabei ist $\Delta z$ der Unterfokus, $C_s$ die mikroskopspezifische sph"arischen Aberationskonstante, $\lambda$ die Wellenl"ange der Elektronen und $k$ der reziproke Abstand vom Bildmittelpunkt im Frequenzraum.
Biologische Proben führen nur zu einem schwachen Phasenkontrast, weshalb der Unterfokus relativ hoch gewählt werden muss.
Zur Rückrechnung auf den Amplitudenkontrast wird die Kontrast-Transfer-Funktion (KTF, Gleichung \eqref{KTF}, \cite{zhu}) verwendet, die sowohl Anteile von Phasen-, als auch Amplitudenkontrast beinhaltet.
Sie ist die Fouriertransformierte der Punktspreizfunktion, die angibt, wie ein punktförmiges Objekt durch das System dargestellt wird.

\begin{equation}
	\text{KTF}(\gamma(k,\Delta z)) = \sqrt{1-A^2} \sin[\gamma(k,\Delta z)] - A \cos[\gamma(k,\Delta z)]. \label{KTF}
\end{equation}

Ein Beispiel für eine KTF ist in Abbildung \ref{KTF_pic} dargestellt.
Die Funktion alterniert zwischen $+1$ und $-1$, wobei die Nullstellen vom Unterfokus abhängen.
An den Nullstellen gibt es keinen Kontrast und das Bild enthält keine Informationen bei den entsprechenden Frequenzen.
Daher ist es wichtig während der Bildaufnahme verschiedene Defokuswerte zu wählen, um möglichst alle Frequenzen im gesamten Fourierraum abzudecken.
Die Funktion wird durch eine einhüllende Funktion gedämpft und ist auf Störungen (Drift, Energieaufspaltung im Strahl usw.) zurückzuführen.
Beschrieben wird diese dabei durch den Exponentialkoeffizienten (B-Faktor) der approximierten Gaußfunktion.

\begin{figure}[h!]
\includegraphics[width = 14cm]{Abbildungen/KTF.png}
\caption[Beispiel einer Kontrast-Transfer-Funktion]{Beispiel einer Kontrast-Transfer-Funktion. Links ist der theoretische, rechts ist der reale Verlauf dargestellt. Der reale Verlauf wird von einer  dämpfenden Funktion eingehüllt. (Quelle Behrmann,
2012)}
\label{KTF_pic}
\end{figure}

Für das Auflösungsvermögen eines Mikroskops gilt: Zwei Gegenstände können nicht voneinander getrennt dargestellt werden, wenn das Objekt kleiner ist, als die Wellenlänge der eingesetzten Strahlung.
Bei Elektronen, die mit einer Beschleunigungsspannung von \unit[300]{kV} beschleunigt werden, liegt diese bei $\lambda$ = \unit[0.02]{\AA}.
Die wichtigsten begrenzenden Faktoren für die Auflösungsgrenze sind das Abbe-Kriterium, Abbildungsfehler und die Pixelgröße des Photochips.
Das Abbe-Kriterium ergiebt sich aus der Gleichung \eqref{Abbe} (\cite{alex}).

\begin{equation}
	d = 0.61 \frac{\lambda}{n \sin(\alpha)} \label{Abbe}
\end{equation}

Es ist abhängig von der Brechzahl $n$ des Mediums zwischen Objektiv und Probe und dem halben Öffnungswinkel des Objektivs (Quelle).
Abbildungsfehler, wie die spährische Aberation, führen zu Anteilen in der KTF (Gleichung \eqref{KTF}).
Bei höheren Frequenzen führt der geringer werdende Abstand zwischen zwei Nulldurchgängen dazu, dass das Signal nicht mehr fehlerfrei ausgewertet werden kann.
Auch ein driften der Probe während der Aufnahme führt zu einer Abschwächung des Signals.
Zudem spielt die Pixelgröße $A_{px}$ des Photochips eine Rolle.
Das Abtasttheorem (\cite{numerical_recipes}) besagt, dass die maximale Auflösung dem doppelten Wert der Pixelgröße entspricht.
Eine Fouriertransformation überträgt die Bildinformation in den Frequenzraum, in dem die Intensität bei verschiedenen Frequenzen $f_s$ (spartial frequency) dargestellt wird.
Die Frequenz der maximalen Auflösung wird als Nyquist-Frequenz bezeichnet.
Es wird meist die Einheit der absoluten Frequenz $f_a$ verwendet, die eine Verbindung zwischen der Pixelgröße und der Frequenz durch Gleichung \eqref{freq} erstellt.

\begin{equation}
	f_a = A_{px} / f_s \label{freq}
\end{equation}

Es ist zu erkennen, dass dieser Wert nicht kleiner als $0.5$ werden sollte, da dies der Nyquist Frequenz entspricht und eine höhere Auflösung nicht möglich ist.

\begin{table}

	\begin{tabular}[h!]{l l l}
			&	JEM1400	&	JEM3200-FSC\\
		\hline
		Elektronenquelle & LaB$_6$-Kristall & Feldemission (Schottky)\\
		Typ	&	TEM	&	Feldemission-Kryo-TEM\\
		Beschleunigungsspannung & \unit[40-120]{kV} & \unit[100-300]{kV}\\
		Vergrößerung & 200 - 1.200.000 & 100-1.200.000\\
		Probenkühlung & Stickstoff & Stickstoff, Helium\\
		Energiefilter & keiner & in-column-Energiefilter\\
		\hline
		\hline


	\end{tabular}
	\caption[Mikroskopdaten]{Kenngrößen der verwendeten Mikroskope.(Quelle Benutzerhandbuch, \cite{jeol})}
	\label{TEM_tab}
\end{table}

\FloatBarrier
\section{Probenpräparation für die negative Kontrastierung} % (fold)
\label{sec:probenpr_aparation_mit_negativer_kontrastierung}

Um Proben mit einem Transmissionselektronenmikroskop untersuchen zu können, werden zunächst ein paar Vorbereitungen getroffen.
Die Probe muss strukturerhaltend auf dem Probenträger fixiert werden, damit diese sich nicht bewegen kann.
Dies hat zudem den Vorteil, dass die Probe nicht in das Vakuum des Mikroskops evaporieren kann.
Weiterhin muss ein Kontrast vorhanden sein, um die Strukturen der Probe erkennen zu können.

Dafür wird die Probe zusammen mit einer Schwermetalllösung auf einen Probenträger aufgetragen.
Dieser ist ein \unit[7.3]{mm$^2$} großes, rundes Kupfergitter (G2400C, \cite{grid}).
Die Probe wird zur Verarbeitung in einem Puffer auf die benötigte Konzentration verdünnt.
Nach der Benetzung mit einem Polymerfilm wird das Gitter mit Kohlenstoff bedampft, auf dem die Probe später haften wird.
Das Gitter wird in ein Plasma eingebracht, das aus beschleunigten geladenen und ungeladenen Atomen und Molekülen besteht.
Durch die Wechselwirkung entstehen Ionen auf der Oberfläche und diese wird polarisiert.
Dies ist notwendig, da die Schwermetall- und Probenlösung hydrophil (polar) ist und die Probe ansonsten nicht haftet.
Nach dem Auftragen der Probe wird diese gewaschen, mit der Schwermetalllösung benetzt und anschließend luftgetrocknet (\cite{neg_stain}).

Die negative Kontrastierung ist eine gute Möglichkeit zur Kontrastverstärkung in einem Transmissionselektronenmikroskop bei Proben, die primär aus leichten Elementen aufgebaut sind.
Wie in Kapitel \ref{sec:elektron_spektroskopie} beschrieben streuen größere Atomkerne im Vergleich vermehrt elastisch, während kleinere vermehrt inelastisch streuen.
Biologische Proben bestehen aus leichten Elementen (Kohlenstoff, Stickstoff, Sauerstoff etc.), während Schwermetalle schwere Elemente (Uran) beinhalten.
Der auftretende Intensitätsunterschied führt in dem erzeugten Bild zu einem Amplitudenkontrast.
Da an der Lösung und nicht an der Probe gestreut worden ist, wird diese Methode als negative Kontrastierung bezeichnet.

Die Nachteile der negativen Kontrastierung wirken sich vor allem negativ auf die Auflösung aus.
Durch das Trocknen der Probe kann es zu Deformationen der nativen Konformation und Artefakten der Kontrastlösung kommen.
Dies führt besonders bei großen Molekülen zu deutlichen Fehlern in der Rekonstruktion.
Auch gibt es Probleme bei inneren Strukturen. 
Die Schwermetalle kommen nicht in die Proteine und an diesen Stellen ist der Kontrast kaum bis gar nicht vorhanden.

\FloatBarrier
\section{Kryopr"aparation}
\label{sec:probenpr_aparation_mit_kryopr_aparation}

Hämocyanine gehören zu den größten Proteinen der Natur und sind daher besonders anfällig für Rekonstruktionsfehler bei der negativen Kontrastierung.
Bei der Kryopräparation wird die Probe in amorphem, vitrifiziertem Eis schockgefroren.
Damit erfüllt diese Methode die Anforderungen aus Kapitel \ref{sec:probenpr_aparation_mit_negativer_kontrastierung}.
Zudem wird die Probe in Lösung gefroren und nicht getrocknet und somit verbleibt diesse in ihrer nativen Konformation.
Für den Vitrifizierungsvorgang ist das Gerät \textit{Cryo Plunge 3} verwendet worden.
Ein Gefrieren ohne Bildung von Eiskristallen ist mit flüssigem Stickstoff nicht möglich, da der Leidenfrost-Effekt ein langsames Gefrieren bedingt.
Daher findet das eigentliche Vitrifizieren in einem mit Ethan gefüllten Behälter statt, der mit Stickstoff gekühlt wird.

Biologische Proben bestehen wie Wasser aus leichten Elementen und streuen bevorzugt inelastisch.
Wie in Kapitel \ref{sec:elektron_spektroskopie} beschrieben, sind keine Strukturen bei einem Amplitudenkontrast etwa gleich großer Elemente zu erkennen.
Daher ist es besonders wichtig auf den Phasenkontrast nach Gleichung \eqref{KTF} zurückzugreifen.

Der Probenträger ist ein etwa (größe hier einfügen) großes Metallgitter (z.B. Quantifoil R2/1 Cu-300, Plano GmbH).).
Dieses ist bereits mit einem Kohlenstofffilm überzogen, in dem in regelmäßigen Abständen industriell Löcher geäzt worden sind (http://www.cp-download.de/plano11/Kapitel-1.pdf).
Die Oberfläche des Gitters wird, wie in Kapitel \ref{sec:probenpr_aparation_mit_negativer_kontrastierung} beschrieben, durch ein Plasma polarisiert.
Zum Auftragen der Probe wird das Gitter in eine Pinzette eingespannt und in das Gerät gehängt.
Durch eine sich seitlich befindende Öffnung wird nun die Probe auf das Gitter aufgetragen.
Das Gerät übernimmt den Ablösch- und Gefriervorgang vollautomatisch.
Vom Probenträger wird die Probenlösung mit Filterpapier fast vollkommen entfernt, sodass nur noch ein dünner Film übrig bleibt.
Anschließend wird das Gitter in den mit flüssigem Ethan gefüllten Behälter fallen gelassen.
Dabei ist es wichtig, dass die Luftfeuchtigkeit erhöht ist.
Dadurch wird ein Verdunsten während des Fallens verringert.
Abschließend kommt das Gitter direkt in die sich in flüssigem Stickstoff befindliche Probenaufbewahrungsbox und diese wird wiederum bis zur Verwendung in flüssigem Stickstoff gelagert.

\FloatBarrier
%\section{Einzelpartikelanalyse}
\label{einzelpartikelanalyse}
(Selbst erstellte Abbildung des Verlaufs)

Ein Problem bei der rekonstruktion eines 3D Modells mithilfe eines Transmissionselektronenmikroskops ist, dass sich nur 2D Abbildungen der 3D Elektronendichte erzeugen lassen.
Es ist daher eine Rückprojektion nötig, um das Objekt darstellen zu können.
Eine etablierte Methode für Rücktransformationen ist die Tomographie, doch ist diese bei biologischen Proben nicht vorteilhaft, da die relativ hohe Elektronendosis das Material zerstört und nicht alle Raumwinkel abgedeckt werden können.
Alternativ wird daher auf die Einzepartikelanalyse zurückgegriffen.
Für diese ist es notwendig, dass die Probe ungeordnet vorliegt, sodass viele 2D Projektionen bei verschiedenen Projektionsparametern der Probe auftreten.
Diese sind neben den drei Euler-Winkeln $\Psi$, $\Phi$ und $\Theta$ auch die Verschiebung in x- und y-Richtung, da die ausgewählten Partikel nicht immer im Schwerpunkt des Bildes liegen.
Während bei der Tomographie diese bekannt sind, müssen bei dieser Methode diese erst iterativ berechnet werden.
Die Methode Arbeitet nach dem in Abbildung \ref{Vorgehensweise} dargestellten Schema.

Von einer 3D Referenzrekonstruktion werden in festgelegten Winkelabständen eine definierte Anzahl an 2D Projektionen erstellt.
Jeder Partikel wird mit den Projektionen verglichen, indem die Projektionsparamter des Partikels variiert werden (\textit{multi reference alignment}). 
Anschließend wird er der am besten passenden Projektionsrichtung zugeordnet.
Wurde zu jedem Partikel eine Projektionsrichtung gefunden wird eine Rekonstruktion mit diesen Parametern erstellt.
Das erstellte Volumen kann anschließend noch symmetrisiert und gefiltert werden.
Abschließend wird das Volumen als Referenz für den nächsten Iterationsschritt übergeben.

Als Startreferenz dient zumeist ein der Geometrie der Probe angepasstes Inertialmodell.
Ist keine Information über die Geometrie vorhanden, wird ein Elipsoid verwendet, der mit gaußschem Rauschen gefüllt ist.
Da bisher keine Strukturen vorhanden sind, wird ohne Einschränkung der Parameter eine globale Suche mit großer Schrittweite durchgeführt.
Es folgt eine lokale Umgebungssuche mit kleiner Schrittweite.

Eine Verbesserung der Rekonstruktion lässt sich zudem durch eine Korrektur der Kontrast-Transfer-Funktion erreichen.
Dies ist zum Beispiel durch das Abschneiden der Information ab der ersten Nullstelle der KTF möglich. 
Jedoch ist gerade die Information der kleinen Frequenzen bei der Hochauflösung relevant.
Eine andere Möglichkeit ist das Umklappen der negativen Bereiche in den positiven Bereich (\textit{phase flipping}).

Zudem können die kleinen Frequenzen mithilfe des B-Faktors verstärkt werden (Vergleich Kapitel \ref{sec:elektron_spektroskopie}). 
Jedoch verschlechtert sich das Signal-zu-Rausch Verhältnis, da bei der Verstärkung nicht zwischen Information und Rauschen unterschieden werden kann.

\FloatBarrier
\section{Einzelpartikelanalyse}
\chapter{Auswertungsmethodik}
\chapter{Ergebnisse}
\chapter{Zusammenfassung und Ausblick}

Hier sollen die Ergebnisse zusammengefasst und weiterf\"uhrende Untersuchungen diskutiert werden. 


% >>> Anhang <<<

\begin{appendix}
%\input{Kapitel/Anhang}
\end{appendix}

% >>> Literaturverzeichnis <<<

\renewcommand{\bibname}{Quellenverzeichnis}
\addcontentsline{toc}{chapter}{\bibname}
\bibliographystyle{unsrt}
\bibliography{BachelorArbeit}

\newpage
\thispagestyle{empty}
\ \\

% >>> Erklaerung <<<

\thispagestyle{empty}
\begin{center}
\section*{Eidesstattliche Versicherung}
\end{center}
\vspace*{1cm}
\noindent
Ich versichere hiermit an Eides statt, dass ich die vorliegende Bachelorarbeit mit dem Titel ''{\thetitle}'' selbst\"andig und ohne unzul\"assige fremde Hilfe erbracht habe. Ich habe keine anderen als die angegebenen
Quellen und Hilfsmittel benutzt sowie w\"ortliche und sinngem\"a\ss e Zitate kenntlich gemacht.
Die Arbeit hat in gleicher oder \"ahnlicher Form noch keiner Pr\"ufungsbeh\"orde vorgelegen.
\vspace*{1cm}
\ \\
\ \\
\line(1,0){150} \hfill \line(1,0){150}\\
Ort, Datum \hfill Unterschrift \hspace*{3cm}
\vspace*{1.5cm}

\subsection*{Belehrung}
Wer vors\"atzlich gegen eine die T\"auschung \"uber Pr\"ufungsleistungen betreffende Regelung einer Hochschulpr\"ufungsordnung
verst\"o\ss t handelt ordnungswidrig. Die Ordnungswidrigkeit kann mit einer Geldbu\ss e von bis zu \unit[50.000,00]{\euro} geahndet werden. Zust\"andige Verwaltungsbeh\"orde f\"ur die Verfolgung und Ahndung von Ordnungswidrigkeiten ist
der Kanzler/die Kanzlerin der Technischen Universit\"at Dortmund. Im Falle eines mehrfachen oder sonstigen schwerwiegenden T\"auschungsversuches kann der Pr\"ufling zudem exmatrikuliert werden (\S\ 63 Abs. 5 Hochschulgesetz - HG - ).\\
\ \\
Die Abgabe einer falschen Versicherung an Eides statt wird mit Freiheitsstrafe bis zu 3 Jahren oder mit Geldstrafe bestraft.\\
\ \\
Die Technische Universit\"at Dortmund wird ggf. elektronische Vergleichswerkzeuge (wie z.B. die Software ''turnitin'') zur \"Uberpr\"ufung von Ordnungswidrigkeiten in Pr\"ufungsverfahren nutzen.\\
\ \\
Die oben stehende Belehrung habe ich zur Kenntnis genommen.
\vspace*{1cm}
\ \\
\ \\
\line(1,0){150} \hfill \line(1,0){150}\\
Ort, Datum \hfill Unterschrift \hspace*{3cm}
\vspace*{\fill}

\end{document} 