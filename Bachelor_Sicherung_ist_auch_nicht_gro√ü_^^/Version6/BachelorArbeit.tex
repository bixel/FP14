\documentclass[11pt,a4paper,twoside]{report}
\usepackage{german}
\usepackage{graphicx}
\usepackage{Tex/nicehead}
\usepackage{epsfig}
\usepackage{amssymb}
\usepackage{amsmath}
\usepackage{tabularx}
\usepackage{calc}
\usepackage[vflt]{floatflt}
\usepackage{units}
\usepackage{upgreek}
\usepackage[pdfborder={0 0 0}, hypertexnames=false]{hyperref}
\usepackage[official]{eurosym}
\usepackage{fontspec}
\usepackage{wrapfig}
\usepackage[]{placeins}

\newcommand{\mytil}{{\raise.17ex\hbox{$\scriptstyle\mathtt{\sim}$}}} 

% Seitenstil
\pagestyle{emheadings}

% Breite des Textblocks und der Raender
\setlength{\evensidemargin}{0mm}
\setlength{\oddsidemargin}{13mm}
\setlength{\textwidth}{145mm}

\setcounter{secnumdepth}{3}   % Tiefe der Kapitelnummerierung
\setcounter{tocdepth}{3}      % Tiefe der Kapitelnummerierung im Inhaltsverzeichnis

\setcounter{totalnumber}{3}
\renewcommand{\floatpagefraction}{0.99}


\mathcode`\,="013B
\begin{document}
\pagenumbering{Roman}

% Anmerkung: Die Seitenraender wurden asymmetrisch gewaehlt,
%            damit genug Platz fuer eine Klemmbindung da ist.
%            %
%            Die Seitenraender koennen in der Datei Tex/global.tex
%            veraendert werden.

% >>> Titelseite <<<

\newcommand{\thetitle}{Titel der Bachelorarbeit}

\thispagestyle{empty}
\begin{center}
\Huge\textbf{\thetitle}
\vfill
\Large
Bachelorarbeit \\ zur Erlangung des akademischen Grades \\ Bachelor of Science \\
\vspace{20pt}
\normalsize
vorgelegt von \\[5pt]
{\Large Markus Stabrin} \\[5pt]
geboren in Hamburg \\
\vspace{20pt}
Angefertigt am Max-Planck-Institut f\"ur molekulare Physiologie in Dortmund \\ Geschrieben am Lehrstuhl f\"ur Experimentelle Physik I \\ Fakult\"at Physik \\
Technische Universit\"at Dortmund \\ 2014
\end{center}
\newpage

% >>> Gutachterseite <<<

\thispagestyle{empty}
\vspace*{\fill}
\begin{tabbing}
1. Gutachter : \=\kill
1. Gutachter : \>Prof. Dr. Metin Tolan \\[11pt]
2. Gutachter : \>Prof. Dr. Stefan Raunser \\[11pt]
\end{tabbing}
\vspace{11pt}
Datum des Einreichens der Arbeit: 09.\,07.\,2014
\newpage

% >>> Kurzfassung/Abstract <<<

\thispagestyle{empty}
\section*{Kurzfassung}
In der vorliegenden Arbeit ist die Struktur des Hämocyanins des Gemeinen Tintenfisches (\textit{Sepia Officinalis}) untersucht worden.
Dabei ist auf die Methode der Kryotransmissionselektronenmikroskopie zurückgegriffen worden.
Die notwendige Theorie wird erläutert.
Es wird auf die Probenpräparation mithilfe der negativen Kontrastierung und der Kryopräparation eingegangen, jedoch nicht auf die Aufreinigung des Proteins.
Da bisher nicht bekannt gewesen ist, ob das Protein eine Symmetrie aufweist, wird ein besonderes Augenmerk auf die Rekonstruktion von Proben mit nicht bekannter Symmetrie gelegt.
Dabei ist eine Einzelpartikelanalyse verwendet worden, wobei die Programme EMAN/EMAN2 und die Programmbibliothek SPARX benutzt worden sind.

Schlussendlich konnte eine Rekonstruktion bei einer Auflösung von ca. \unit[9.5]{\AA} erzeugt werden, mit deren Hilfe die Architektur bestimmt worden ist.
Innerhalb dieser Architektur konnten Symmetrien gefunden werden.

\section*{Abstract}
This bachelor thesis deals with the structure of the hemocyanin of the common cuttlefish (\textit{Sepia Officinalis}).
To solve the protein's structure cryoelectronmicroscopy has been used.
The necessary theory will be explained.
Furthermore, this thesis is about negative staining and cryopreparation, but not about the cleanup of the used protein.
As there is no known symmetry in this hemocyanin's structure, the focus is on the reconstruction of samples with unknown symmetry.
With the help of the programs EMAN/EMAN2 and the library SPARX a single particle analysis has been realized. 
Finally, a reconstruction with a resolution of approx. \unit[9.5]{\AA} has been received.
Consequently, the topology could be solved and symmetries have been found.
\newpage

% >>> Hauptteil <<<

\addcontentsline{toc}{chapter}{Inhaltsverzeichnis}
\tableofcontents\newpage
\addcontentsline{toc}{chapter}{Abbildungsverzeichnis}
\listoffigures\newpage
\addcontentsline{toc}{chapter}{Tabellenverzeichnis}
\listoftables\newpage

\setcounter{page}{0}
\pagenumbering{arabic}

\chapter{Einleitung}
\label{einleitung}

Hier folgt eine kurze Einleitung in die Thematik. 

Hier folgt eine kurze Einleitung in die Thematik.

Die Einleitung muss kurz sein, damit die vorgegebene Gesamtl"ange der
Arbeit von 25 Seiten nicht "uberschritten wird. Die Beschr"ankung der
Seitenzahl sollte man ernst nehmen, da "uberschreitung zu Abz"ugen in
der Note f"uhren kann. Selbstverst"andlich kann jede(r) f"ur sich eine
Version der Arbeit mit beliebig vielen und langen Anh"angen und
Methodenkapiteln erstellen. Das ist dann aber Privatvergn"ugen; die
einzureichende und zu beurteilende Arbeit muss der
L"angenbeschr"akung gen"ugen. Um der L"angenbeschr"ankung zu gen"ugen,
darf auch nicht an der Schriftgr"o"se, dem Zeilenabstand oder dem
Satzspiegel (bedruckte Fl"ache der Seite) manipuliert werden.

\FloatBarrier
\chapter{Theoretische Grundlagen und Messvorbereitung}
\section{H"amocyanin}

(Abbildung)

Mit einer Größe von \unit[3-8]{MDa} gehören Hämocyanine zu den größten bekannten Proteinen der Natur (Habe im Evolution of Hämocyanin structure gesucht, aber nichts über die Größe direkt gefunden).
Diese transportieren sowohl den Sauerstoff von marin lebenden wirbellosen Tieren als auch einiger Spinnenarten.
Sie liegen ohne an Blutzellen gebunden zu sein direkt in der Hämolymphe dieser Tierarten vor.
In den aktiven Zentren bindet der Sauerstoff an Kupfer und oxidiert diesen von Cu(I) zu Cu(II).
Dieses emittiert, durch Kupfer-Peroxo-Komplexbildung mit dem Sauerstoff, Lichtquanten im sichtbaren Wellenlängenbereich von \unit[480-420]{nm}. (http://www.chemieunterricht.de/dc2/komplexe/farbe.html)(http://archimed.uni-mainz.de/pub/2001/0109/diss.pdf)
Damit sind Lebewesen mit Hämocyaninen Blaublüter.

Hämocyanine besitzen mit einem Radius von ca. \unit[35]{nm} und einer Höhe von ca. \unit[15]{nm} eine hohlzylindrische Struktur (Quelle Paper von Julian).
Sie bestehen aus 10 strukturell ähnlichen Untereinheiten, und sind somit Decamere (Oligomere bestehend aus 10 Untereinheiten) (Quelle Paper Julian). 
Jede Untereinheit besteht weiterhin aus sieben bis acht funktionellen Einheiten (FE), die artspezifisch sind.
Das Hämocyanin des \textit{Sepia Officinalis} (\textit{gemeiner Tintenfisch}), das in dieser Arbeit untersucht wird, besteht beispielsweise aus acht dieser FEs (benannt N' term-a-b-c-d-d'-e-f-g-C' term) (Quelle Paper von Julian).
Jede FE ist ca. \unit[50]{kDa} schwer, wodurch das gesamte Protein ein Gewicht von ca. \unit[4]{MDa} besitzt.

Die FEs FE-a,FE-b,FE-c,FE-d,FE-e und FE-f sind bei allen bekannten Hämocyaninen vorhanden.
Dabei ähnelt sich deren Architektur und es liegt nahe, dass sich diese FEs vor der Aufspaltung der einzelnen Arten entwickelt haben.
Daher ist anzunehmen, dass diese in der Struktur dieselben Positionen besetzen und somit dieselbe Rolle beim Bilden der Quartärstruktur ihres Proteins spielen, obwohl dies noch nicht direkt gezeigt werden konnte (Habe die Paper gesucht aber nicht gefunden, waren das die, die du mir noch schicken wolltest?).
FE-d' hingegen entwickelte sich vor etwa 740 Millionen Jahren durch eine Duplizierung der FE-d.

Geometrisch betrachtet besteht der Hohlzylinder aus zwei kovalent miteinander verbundenen Bereichen, der äußeren Wand und dem inneren Kragen.
Die Wand ist charakteristisch für alle Hämocyanine und setzt sich aus den oben beschriebenen FEs a,b,c,d,e und f zusammen.
Der Kragen hingegen ist bei jeder Hämocyaninart anders aufgebaut und besteht beim \textit{Sepia Officinalis} aus FE-d' und FE-g.

Da die Funktionsweise von Proteinen durch ihre Quartärstruktur festgelegt ist, ist es notwendig eine Strukturanalyse durchzuführen.
Eine Möglichkeit wäre die hochauflösende Kristallographie, doch lassen sich besonders große Proteine nur sehr schwer kristallisieren.
Zudem liegen diese nicht mehr in ihrer nativen Konformation vor, wie es in Lösung der Fall ist.
Daher bietet sich die Kryotransmissionselektronenmikroskopie an, welche eine Hochauflösung der Struktur ermöglicht.

\FloatBarrier
\section{Transmissionselektronenmikroskopie} % (fold)
\label{sec:elektron_spektroskopie}

Das Prinzip der Bildentstehung bei einem Transmissionselektronenmikroskop ist dem eines Lichtmikroskops sehr ähnlich:
Ein Elektronenstrahl erzeugt eine Projektion der Probe im Realraum.
Im Vergleich zum Lichtmikroskop werden jedoch keine optischen, sondern elektrische Linsen verwendet.
Ein schematischer Aufbau der verwendeten Mikroskope ist in Abbildung \ref{TEM} dargestellt.
Die dazugehörigen technischen Daten können der Tabelle (\ref{TEM_tab}) entnommen werden.

Als Elektronenquelle wird je nach Mikroskop entweder ein LaB$_6$-Kris\-tall (Lanthanhexaborid) oder eine Feldemissionsquelle genutzt.
Bei der Feldemission ($\Delta$E = \unit[0.6-1.2]{eV}) ist ein schmaleres Energiespektrum der Elektronen zu beobachten als beim LaB$_6$-Kris\-tall ($\Delta$E = \unit[1.5-3]{eV}), wodurch ein höheres Aufl"osungsverm"ogen ermöglicht wird.
Nach dem Beschleunigen der Elektronen werden diese mithilfe von elektronischen Linsen auf die Probe und anschließend auf die Kamera fokusiert.
Diese Linsen bestehen aus Spulen, die Magnetfelder erzeugen.
Die bewegten Elektronen werden von diesen durch die Lorentz-Kraft abgelenkt ohne deren Energie zu beeinflussen (Nur die senkrechte Komponente der Kraft wirkt auf die Elektronen).
Um Wechselwirkungen der Elektronen mit anderer Materie als der Probe zu verhindern, ist das gesamte Spulensystem evakuiert.
Abschließend wird das Transmissionsbild im Objektsystem vergrößert und auf einem Fluoreszensschirm abgebildet oder direkt mit einer Kamera oder einem Photofilm detektiert.
Abhängig von der Intensität erzeugt dieser ein Bild mit Graustufen.
Je dunkler der Bildpunkt ist, umso größer ist die Intensität.

\begin{wrapfigure}{r}{7cm}
	\centering
	\includegraphics[width = 7cm]{Abbildungen/TEM.png}
	\caption[Schematische Aufbau eines Transmissionselektronenmikroskop]{Schematische Aufbau eines Transmissionselektronenmikroskop. (Entnommen aus Benutzerhandbuch JEOL Ldt.)}
	\label{TEM}
\end{wrapfigure}

Damit auf dem Bild Strukturen erkennbar werden, muss ein ausgeprägter Amplitudenkontrast zwischen Probe und Lösung vorhanden sein.
Eine Schwierigkeit tritt dabei bereits dadurch auf, dass die Elektronendosis bei biologischen Proben nicht zu groß gewählt werden darf um Schäden zu vermeiden.
Der Amplitudenkontrast kann sowohl durch Teilchen- als auch durch Welleneigenschaften der Elektronen entstehen. 
Bei der Betrachtung als Teilchen ergiebt sich der Kontrast durch Streueffekte.
Dafür dürfen die Ordnungszahlen der Elemente in der Probe und der Lösung nicht in derselben Größenordnung sein.
Dies ist auf die Wechselwirkung der Elektronen an Atomkernen zurückzuführen, da an großen Kernen diese vermehrt elastisch gestreut werden, während an kleinen Kernen bevorzugt inelastische Streuung auftritt.
Ersteres führt zu einer Richtungsänderung der Elektronen.
Mithilfe einer Blende werden diese aus dem Elektronenstrahl entfernt.

Die inelastisch gestreuten Elektronen tragen jedoch nur zum Rauschen im Bild bei und tragen keine Information.
Der Amplitudenkontrast kommt dementsprechend durch die geringere Intensität bei Elementen größerer Ordnungszahl zustande.

Befinden sich nun die Elemente der Probe und der Lösung in derselben Größenordnung, werden an beiden etwa gleich viele Elektronen elastisch gestreut und es ist kein Kontrast zu erkennen.
Wird das Elektron jedoch als Welle betrachtet kann eine andere Art von Kontrast hergeleitet werden.
Wenn elektromagnetische Wellen mit Materie wechselwirken, kommt es zu einer Phasenverschiebung.
Diese führt wiederum zu Interferenzerscheinungen zwischen den gebeugten und den ungebeugten Elektronen und wird als Phasenkontrast bezeichnet.
Die Verschiebung wird durch die Gleichung \eqref{Phase} (\cite{scherzer}) beschrieben.

\begin{equation}
	\gamma(k,\Delta z) = \frac{\pi}{2}\left(\lambda^3 C_s k^4 - 2\lambda \Delta z k^2\right). \label{Phase}
\end{equation}

Dabei ist $\Delta z$ der Unterfokus, $C_s$ die mikroskopspezifische sph"arischen Aberationskonstante, $\lambda$ die Wellenl"ange der Elektronen und $k$ der reziproke Abstand vom Bildmittelpunkt im Frequenzraum.
Biologische Proben führen nur zu einem schwachen Phasenkontrast, weshalb der Unterfokus relativ hoch gewählt werden muss.
Zur Rückrechnung auf den Amplitudenkontrast wird die Kontrast-Transfer-Funktion (KTF, Gleichung \eqref{KTF}, \cite{zhu}) verwendet, die sowohl Anteile von Phasen-, als auch Amplitudenkontrast beinhaltet.
Sie ist die Fouriertransformierte der Punktspreizfunktion, die angibt, wie ein punktförmiges Objekt durch das System dargestellt wird.

\begin{equation}
	\text{KTF}(\gamma(k,\Delta z)) = \sqrt{1-A^2} \sin[\gamma(k,\Delta z)] - A \cos[\gamma(k,\Delta z)]. \label{KTF}
\end{equation}

Ein Beispiel für eine KTF ist in Abbildung \ref{KTF_pic} dargestellt.
Die Funktion alterniert zwischen $+1$ und $-1$, wobei die Nullstellen vom Unterfokus abhängen.
An den Nullstellen gibt es keinen Kontrast und das Bild enthält keine Informationen bei den entsprechenden Frequenzen.
Daher ist es wichtig während der Bildaufnahme verschiedene Defokuswerte zu wählen, um möglichst alle Frequenzen im gesamten Fourierraum abzudecken.
Die Funktion wird durch eine einhüllende Funktion gedämpft und ist auf Störungen (Drift, Energieaufspaltung im Strahl usw.) zurückzuführen.
Beschrieben wird diese dabei durch den Exponentialkoeffizienten (B-Faktor) der approximierten Gaußfunktion.

\begin{figure}[h!]
\includegraphics[width = 14cm]{Abbildungen/KTF.png}
\caption[Beispiel einer Kontrast-Transfer-Funktion]{Beispiel einer Kontrast-Transfer-Funktion. Links ist der theoretische, rechts ist der reale Verlauf dargestellt. Der reale Verlauf wird von einer  dämpfenden Funktion eingehüllt. (Quelle Behrmann,
2012)}
\label{KTF_pic}
\end{figure}

Für das Auflösungsvermögen eines Mikroskops gilt: Zwei Gegenstände können nicht voneinander getrennt dargestellt werden, wenn das Objekt kleiner ist, als die Wellenlänge der eingesetzten Strahlung.
Bei Elektronen, die mit einer Beschleunigungsspannung von \unit[300]{kV} beschleunigt werden, liegt diese bei $\lambda$ = \unit[0.02]{\AA}.
Die wichtigsten begrenzenden Faktoren für die Auflösungsgrenze sind das Abbe-Kriterium, Abbildungsfehler und die Pixelgröße des Photochips.
Das Abbe-Kriterium ergiebt sich aus der Gleichung \eqref{Abbe} (\cite{alex}).

\begin{equation}
	d = 0.61 \frac{\lambda}{n \sin(\alpha)} \label{Abbe}
\end{equation}

Es ist abhängig von der Brechzahl $n$ des Mediums zwischen Objektiv und Probe und dem halben Öffnungswinkel des Objektivs (Quelle).
Abbildungsfehler, wie die spährische Aberation, führen zu Anteilen in der KTF (Gleichung \eqref{KTF}).
Bei höheren Frequenzen führt der geringer werdende Abstand zwischen zwei Nulldurchgängen dazu, dass das Signal nicht mehr fehlerfrei ausgewertet werden kann.
Auch ein driften der Probe während der Aufnahme führt zu einer Abschwächung des Signals.
Zudem spielt die Pixelgröße $A_{px}$ des Photochips eine Rolle.
Das Abtasttheorem (\cite{numerical_recipes}) besagt, dass die maximale Auflösung dem doppelten Wert der Pixelgröße entspricht.
Eine Fouriertransformation überträgt die Bildinformation in den Frequenzraum, in dem die Intensität bei verschiedenen Frequenzen $f_s$ (spartial frequency) dargestellt wird.
Die Frequenz der maximalen Auflösung wird als Nyquist-Frequenz bezeichnet.
Es wird meist die Einheit der absoluten Frequenz $f_a$ verwendet, die eine Verbindung zwischen der Pixelgröße und der Frequenz durch Gleichung \eqref{freq} erstellt.

\begin{equation}
	f_a = A_{px} / f_s \label{freq}
\end{equation}

Es ist zu erkennen, dass dieser Wert nicht kleiner als $0.5$ werden sollte, da dies der Nyquist Frequenz entspricht und eine höhere Auflösung nicht möglich ist.

\begin{table}

	\begin{tabular}[h!]{l l l}
			&	JEM1400	&	JEM3200-FSC\\
		\hline
		Elektronenquelle & LaB$_6$-Kristall & Feldemission (Schottky)\\
		Typ	&	TEM	&	Feldemission-Kryo-TEM\\
		Beschleunigungsspannung & \unit[40-120]{kV} & \unit[100-300]{kV}\\
		Vergrößerung & 200 - 1.200.000 & 100-1.200.000\\
		Probenkühlung & Stickstoff & Stickstoff, Helium\\
		Energiefilter & keiner & in-column-Energiefilter\\
		\hline
		\hline


	\end{tabular}
	\caption[Mikroskopdaten]{Kenngrößen der verwendeten Mikroskope.(Quelle Benutzerhandbuch, \cite{jeol})}
	\label{TEM_tab}
\end{table}

\FloatBarrier
\section{Probenpräparation für die negative Kontrastierung} % (fold)
\label{sec:probenpr_aparation_mit_negativer_kontrastierung}

Um Proben mit einem Transmissionselektronenmikroskop untersuchen zu können, werden zunächst ein paar Vorbereitungen getroffen.
Die Probe muss strukturerhaltend auf dem Probenträger fixiert werden, damit diese sich nicht bewegen kann.
Dies hat zudem den Vorteil, dass die Probe nicht in das Vakuum des Mikroskops evaporieren kann.
Weiterhin muss ein Kontrast vorhanden sein, um die Strukturen der Probe erkennen zu können.

Dafür wird die Probe zusammen mit einer Schwermetalllösung auf einen Probenträger aufgetragen.
Dieser ist ein \unit[7.3]{mm$^2$} großes, rundes Kupfergitter (G2400C, \cite{grid}).
Die Probe wird zur Verarbeitung in einem Puffer auf die benötigte Konzentration verdünnt.
Nach der Benetzung mit einem Polymerfilm wird das Gitter mit Kohlenstoff bedampft, auf dem die Probe später haften wird.
Das Gitter wird in ein Plasma eingebracht, das aus beschleunigten geladenen und ungeladenen Atomen und Molekülen besteht.
Durch die Wechselwirkung entstehen Ionen auf der Oberfläche und diese wird polarisiert.
Dies ist notwendig, da die Schwermetall- und Probenlösung hydrophil (polar) ist und die Probe ansonsten nicht haftet.
Nach dem Auftragen der Probe wird diese gewaschen, mit der Schwermetalllösung benetzt und anschließend luftgetrocknet (\cite{neg_stain}).

Die negative Kontrastierung ist eine gute Möglichkeit zur Kontrastverstärkung in einem Transmissionselektronenmikroskop bei Proben, die primär aus leichten Elementen aufgebaut sind.
Wie in Kapitel \ref{sec:elektron_spektroskopie} beschrieben streuen größere Atomkerne im Vergleich vermehrt elastisch, während kleinere vermehrt inelastisch streuen.
Biologische Proben bestehen aus leichten Elementen (Kohlenstoff, Stickstoff, Sauerstoff etc.), während Schwermetalle schwere Elemente (Uran) beinhalten.
Der auftretende Intensitätsunterschied führt in dem erzeugten Bild zu einem Amplitudenkontrast.
Da an der Lösung und nicht an der Probe gestreut worden ist, wird diese Methode als negative Kontrastierung bezeichnet.

Die Nachteile der negativen Kontrastierung wirken sich vor allem negativ auf die Auflösung aus.
Durch das Trocknen der Probe kann es zu Deformationen der nativen Konformation und Artefakten der Kontrastlösung kommen.
Dies führt besonders bei großen Molekülen zu deutlichen Fehlern in der Rekonstruktion.
Auch gibt es Probleme bei inneren Strukturen. 
Die Schwermetalle kommen nicht in die Proteine und an diesen Stellen ist der Kontrast kaum bis gar nicht vorhanden.

\FloatBarrier
\section{Kryopr"aparation}
\label{sec:probenpr_aparation_mit_kryopr_aparation}

Hämocyanine gehören zu den größten Proteinen der Natur und sind daher besonders anfällig für Rekonstruktionsfehler bei der negativen Kontrastierung.
Bei der Kryopräparation wird die Probe in amorphem, vitrifiziertem Eis schockgefroren.
Damit erfüllt diese Methode die Anforderungen aus Kapitel \ref{sec:probenpr_aparation_mit_negativer_kontrastierung}.
Zudem wird die Probe in Lösung gefroren und nicht getrocknet und somit verbleibt diesse in ihrer nativen Konformation.
Für den Vitrifizierungsvorgang ist das Gerät \textit{Cryo Plunge 3} verwendet worden.
Ein Gefrieren ohne Bildung von Eiskristallen ist mit flüssigem Stickstoff nicht möglich, da der Leidenfrost-Effekt ein langsames Gefrieren bedingt.
Daher findet das eigentliche Vitrifizieren in einem mit Ethan gefüllten Behälter statt, der mit Stickstoff gekühlt wird.

Biologische Proben bestehen wie Wasser aus leichten Elementen und streuen bevorzugt inelastisch.
Wie in Kapitel \ref{sec:elektron_spektroskopie} beschrieben, sind keine Strukturen bei einem Amplitudenkontrast etwa gleich großer Elemente zu erkennen.
Daher ist es besonders wichtig auf den Phasenkontrast nach Gleichung \eqref{KTF} zurückzugreifen.

Der Probenträger ist ein etwa (größe hier einfügen) großes Metallgitter (z.B. Quantifoil R2/1 Cu-300, Plano GmbH).).
Dieses ist bereits mit einem Kohlenstofffilm überzogen, in dem in regelmäßigen Abständen industriell Löcher geäzt worden sind (http://www.cp-download.de/plano11/Kapitel-1.pdf).
Die Oberfläche des Gitters wird, wie in Kapitel \ref{sec:probenpr_aparation_mit_negativer_kontrastierung} beschrieben, durch ein Plasma polarisiert.
Zum Auftragen der Probe wird das Gitter in eine Pinzette eingespannt und in das Gerät gehängt.
Durch eine sich seitlich befindende Öffnung wird nun die Probe auf das Gitter aufgetragen.
Das Gerät übernimmt den Ablösch- und Gefriervorgang vollautomatisch.
Vom Probenträger wird die Probenlösung mit Filterpapier fast vollkommen entfernt, sodass nur noch ein dünner Film übrig bleibt.
Anschließend wird das Gitter in den mit flüssigem Ethan gefüllten Behälter fallen gelassen.
Dabei ist es wichtig, dass die Luftfeuchtigkeit erhöht ist.
Dadurch wird ein Verdunsten während des Fallens verringert.
Abschließend kommt das Gitter direkt in die sich in flüssigem Stickstoff befindliche Probenaufbewahrungsbox und diese wird wiederum bis zur Verwendung in flüssigem Stickstoff gelagert.

\FloatBarrier
\chapter{Computerbasierte Einzelpartikelanalyse} \label{EPA}
\begin{figure}
	\includegraphics[width = 14cm, height = 5cm]{Abbildungen/analyse.png}
	\caption[Vorgehensweise zur ersten 3D Rekonstruktion]{Vorgehensweise, um eine erste 3D Rekonstruktion zu erreichen.}
	\label{vorgehensweise}
\end{figure}


Ein Problem bei der Rekonstruktion eines 3D Modells mithilfe eines Transmissionselektronenmikroskops ist, dass sich nur 2D Abbildungen der 3D Elektronendichte erzeugen lassen.
Es ist daher eine Rückprojektion nötig, um das Objekt darstellen zu können.
Eine etablierte Methode für Rücktransformationen ist die Tomographie, doch ist diese bei biologischen Proben nicht vorteilhaft, da die relativ hohe Elektronendosis das Material zerstört und nicht alle Raumwinkel abgedeckt werden können.
Alternativ wird daher auf die Einzelpartikelanalyse zurückgegriffen.
Für diese ist es notwendig, dass die Probe ungeordnet vorliegt, sodass viele 2D Projektionen bei verschiedenen Projektionsparametern der Probe auftreten.
Diese sind neben den drei Euler-Winkeln $\Psi$, $\Phi$ und $\Theta$ auch die Verschiebung in x- und y-Richtung, da die ausgewählten Partikel nicht immer im Schwerpunkt des Bildes liegen.
Während bei der Tomographie diese bekannt sind, müssen bei dieser Methode diese erst iterativ berechnet werden.
Die Methode arbeitet nach dem in der Abbildung \ref{vorgehensweise} dargestellten Schema (\cite{einzelpartikel}).

Da die Anzahl der verwendeten Partikel mehrere Tausend umfasst, muss auf die Rechenleistung von Computern zurückgegriffen werden.
Eine Liste der verwendeten Systeme, Programme und Programmiersprachen ist in Tabelle \ref{EDV} angegeben.

Um eine effizientere Methode der Verarbeitung zu gewährleisten ist es nötig das Bildformat der Bilddateien von TIFF (\textit{tagged image file format}) auf HDF (\textit{hierarchical data format}) zu ändern.
Diese Art von Dateien besitzt eine Kopfzeile (\textit{header}), in der Bildinformationen (Projektionsparameter, Verschiebungen, Bildzugehörigkeit etc.) gespeichert werden können.
Zudem gibt es die Möglichkeit die Speicherung in einer Datenbank (BDB, \textit{Berkeley-Datenbank}) vorzunehmen.
In dieser sind Bild und Kopfzeile getrennt voneinander gespeichert, sodass diese getrennt geladen werden können um die Zugriffszeit zu minimieren.

In diesem Kapitel wird besonders auf die Datenauswertung der bei der Kryo-TEM entstandenen Daten eingegangen.
Bei der negativen Kontrastierung sind die Schritte analog. 
Der schematische Rekonstruktionsprozess kann wie folgt unterteilt werden:

\begin{enumerate}
	\item \textbf{Digitale Bildbearbeitung:} 
	Bestimmung des Defokus aus der KTF-Information; Aussortieren von Bildern (\textit{micrograph}); Filterung der Bilder
	\item \textbf{Extraktion der Partikel:} 
	Auswahl der Partikel aus den Bildern (\textit{particle picking}); Speichern der einzelnen Partikel in einem Datensatz (\textit{stack})
	\item \textbf{Ausrichtung und Klassifizierung:} 
	Zentrierung und Drehung der Partikel \\(\textit{alignment}); Zuordnung der Partikel in verschiedene Klassen; Aussortierung schlechter Einzelbilder
	\item \textbf{3D Rekonstruktion:} Iterative Bestimmung der Projektionsparameter; Rekonstruktion 
	\item \textbf{Verbesserung der Auflösung:} Filterung; Maskierung; Symmetrisierung; KTF Korrektur
	\item \textbf{PDB Modelle:} Bewertung der Auflösung; Bestimmung der Architektur
\end{enumerate}

\begin{table}
	\begin{tabular}[h!]{l l l}
		Name & Verwendung & Quelle \\
		\hline
		Eman/Eman2 & Prozessierung & http://blake.bcm.edu \\
		Sparx & Prozessierung & http://sparx-em.org \\
		CTF-GUI & KTF \& Defokus Bestimmung & MPI-Dortmund (R.Efremov,\\
		& & C.Gatsogiannis)\\
		GUI & Klassifizierung (K-Means) & MPI-Dortmund (C.Gatogiannis)\\ 
		USFC Chimera & 3D Modellierung & http://www.cgl.ucsf.edu/chimera \\
		Python & Programmierung & https://www.python.org \\
		CLB Cluster & Datenverarbeitung & MPI Dortmund \\
		\hline
		\hline
	\end{tabular}
	\caption[Verwendete EDV]{Verwendete Software, Programmiersprachen und Systeme.}
	\label{EDV}
\end{table}



\FloatBarrier
\section{Digitale Bildbearbeitung} % (fold)
\label{sec:bildbearbeitung}

(Wrapfigure: vorher/nachher untereinander)

Bei der Kryo-TEM sind von derselben Position 15 Bilder in \unit[200]{ms} Abschnitten gemacht worden.
Um das Signal-zu-Rausch Verhältnis zu verbessern, werden diese zunächst miteinander verglichen und aufsummiert.

Die Bilder des Mikroskops sollten in das HDF Format umformatiert werden, um effizienter bearbeitet werden zu können.
In der Kopfzeile werden die Bilder zunächst numeriert (\textit{micrograph ID}), sodass später eine Zuordnung möglich ist.
Anschließend wird die KTF der einzelnen Bilder analysiert und in der Kopfzeile als KTF-Objekt gespeichert.
Dafür wir ein KTF-Fit durch das 1D Powerspektrum des Bildes gefittet, wobei z.B. der Defokus durch die Fitparameter bestimmt wird.
Dabei wurde das Programm \textit{CTF-GUI} verwendet.
Dieses gibt neben den Fitparametern auch die dazugehörigen Bilder aus, welche bei der Qualitätskontrolle eine Rolle spielen.
Stimmt die Qualität nicht, kann das Bild sofort aussortiert werden.

Für die Extrahierung der Partikel im nächsten Schritt wird sowohl ein gaußförmiger Highpass-, als auch ein gaußförmiger Lowpassfilter auf eine Kopie der Bilder angewendet.
Dieser bewirkt, dass die besonders kleinen und großen Frequenzen (große und kleine Objekte) unterdrückt werden und die Strukturen der Proteine besser zu erkennen sind.

\FloatBarrier
\section{Extraktion der Partikel} % (fold)
\label{sec:extraktion_der_partikel}

Aus den bearbeiteten Bildern müssen nun die einzelnen Partikel extrahiert werden.
Dafür ist das EMAN2 Programm \textit{e2boxer.py} verwendet worden.
Es wird dafür jeder Partikel auf jedem Bild einzeln ausgewählt und die Koordinaten der Partikelmittelpunkte in einer .box Datei gespeichert.
Anschließend werden die Koordianten auf den ungefilterten Datensatz angewendet und von dort extrahiert.
Dem EMAN-Programm \textit{batchboxer} wird die Boxgröße in Pixeln übergeben, wobei sich die gewählten Koordinaten im Mittelpunkt der quadratischen Box befinden.
Dabei sollte darauf geachtet werden, dass die Größe von EMAN empfohlen wird, da sich ansonsten die Rechenzeit erheblich verlängert (http://blake.bcm.edu/emanwiki/EMAN2/BoxSize).
Weiterhin werden die extrahierten Partikel in einem HDF Datensatz (\textit{stack}) gespeichert.
Abschließend wird der Datensatz normalisiert, invertiert und KTF-Korrigiert.

\FloatBarrier
\section{Ausrichtung und Klassifizierung} % (fold)
\label{sec:klassifizierung}

Klassifizieren bedeutet, dass Gruppen gleicher Projektionsrichtungen gefunden werden.
Durch Aufsummieren der Bilder zu einer Klassensumme wird das Signal-zu-Rausch Verhältnis verbessert, wodurch die geringe Elektronendosis ausgeglichen werden kann.
Eine Klassensumme entspricht dabei einer Projektionsrichtung.
Können manche Partikel keinen Klassen zugeordnet werden, werden diese aus dem Datensatz entfernt, da sie zum Rauschen beitragen.
Der Prozess ist mithilfe des Programms \textit{GUI} automatisiert worden.
\\
\\
\textbf{1. Ausrichtung (Referenzfrei):}
Zunächst werden alle Bilder der Partikel zu einem Bild aufsummiert und als Referenz benutzt.
Das Sparxmodul \textit{sxali2d.py} vergleicht zunächst alle Partikel mit dieser und richtet diese durch Rotation an diesem aus.
Anschließend werden die Partikel noch zueinander verschoben.
Mathematisch wird dazu eine Kreuzkorrelation angewandt (Quelle).
Nach jedem Iterationsschritt wird eine Überlagerung der neu gefundenen Partikel als Referenz genutzt.
\\
\\
\textbf{2. K-Means Klassifizierung:}
Eine mögliche Klassifizierungsmethode ist der K-Means Algorithmus (SPARX:\textit{sxk\_means.py}).
Dieser erstellt eine festgelegt Anzahl $K$ an Klassen, indem er die Bilder in einem Hyperraum in Clustern zusammenfasst.
Dazu werden zunächst $K$ zufällige Schwerpunkte in die Hyperebene gesetzt und durch Verschieben ebendieser die Cluster gefunden.
Dieser Algorithmus ist zwar einfach und schnell, jedoch nicht reproduzierbar, da das Ergebnis stark von den Anfangsbedingungen und der Qualität der Partikel abhängig ist (Quelle Paper).
\\ 
\\
\textbf{3. Ausrichtung (\textit{multi reference alignment}):}
Im Gegensatz zur referenzfreien Ausrichtung stehen nun die erzeugten Klassen als Referenz zur Verfügung.
Die Partikel werden der am besten passenden Klassensumme zugeordnet und nach dieser Ausgerichtet.
Anschließend werden die zugeordneten Partikel zu neuen Klassensummen aufsummiert.
\\ 
\\
\textbf{4. ISAC:}
Der ISAC (\textit{Iterative Stable Alignment and Clustering}) Algorithmus ist eine gute Möglichkeit zur Überprüfung der Klassenergebnisse (SPARX:\textit{sxisac.py}).
In diesem Fall wird nicht die Anzahl der Klassen, sondern die der Partikel pro Klasse vorgegeben.
Nach jedem Iterationsschritt werden die Klassen auf Stabilität und Reproduzierbarkeit im Hinblick auf Alignierung und Klassifizierung geprüft (Quelle Paper).

\FloatBarrier
\section{Rekonstruktion} % (fold)
\label{sec:rekonstruktion}

Um eine erste 3D Rekonstruktion als Startreferenz zu erhalten, werden die erzeugten Klassensummen genutzt.
Deren Signal-zu-Rausch Verhältnis ist im Vergleich zu den Einzelpartikeln durch die Summation verbessert worden.
Mithilfe des Sparx-Programms \textit{sxviper.py} (Bisher unveröffentlicht) wird so ein erstes Modell erzeugt.
Als Inertialmodell wird hierbei eine zylindrische Struktur verwendet.

Eine iterative Findung der Projektionsparamter führt das Sparx-Programms \textit{sxali3d.py} durch.
Dafür überführt das Programm die 3D Referenz bei festen Winkelabschnitten in 2D Projektionen und vergleicht diese mit den Partikeln (\textit{multi reference alignment}).
Die Partikel werden nach der am besten passenden Projektion ausgerichtet und die Projektionsrichtungen in die Kopfzeile geschrieben.
Aus den bestimmten Parametern wird erneut eine 3D Rekonstruktion aus den Partikeln erstellt und als Referenz an den nächsten Iterationsschritt übergeben.
Dabei werden die Größe der Winkelabschnitte stetig verringert, sodass mehr Projektionsrichtungen möglich sind.

Um die Auflösung einer Rekonstruktion zu bestimmen ist das FSK$_{0.5}$-Kriterium angewendet worden.
Dazu wird der Datensatz der verwendeten Partikel in zwei gleich große Teile aufgespalten und von jedem eine Rekonstruktion erstellt.
Diese werden Mithilfe einer Kreuzkorrelation miteinander Verglichen und die Übereinstimmung in Abhängigkeit der Frequenz dargestellt.
Als noch Aufgelöst wird die Frequenz bezeichnet, bei der die Übereinstimmen noch \unit[50]{\%} beträgt.

\FloatBarrier
\section{Verbesserung der Auflösung} % (fold)
\label{sec:verbesserung_der_aufloesung}

Um nach einem erfolgreichen Iterationsschritt eine verbesserte Referenz zu erhalten ist es von Vorteil, die zuletzt erzeugte Rekonstruktion zu bearbeiten.
Dies kann durch Maskierung, Filterung, Symmetriesierung oder KTF-Korrektur möglich gemacht werden.
\\
\\
\textbf{Maskierung:} Bei einer 3D Rekonstruktion entstehen Fragmente (Rauschen) außerhalb des Dichtevolumens des Partikels. 
Diese können beim nächsten Iterationsschritt zu fehlerhaften Zuordnungen der Projektionsparamter führen.
Ein Entfernen des Rauschens ist durch das Maskieren des Partikels mit einer binären Maske möglich.
Wird das Volumen mit der Maske multipliziert bleibt nur der Teil übrig, der von der Maske umschlossen ist.
\\
\\
\textbf{Filterung:} Stimmen die Partikelprojektionsrichtungen nicht genau mit der Realität überein, kommt es zu Überlappungen der Dichten.
Besonders bei den großen Frequenzen (kleinen Strukturen) entsteht so ein Rauschen im Volumen.
Daher ist es von Vorteil die diese aus der Dichte mithilfe eines Lowpassfilters herauszufiltern, um das Rauschen zu unterdrücken.
\\
\\
\textbf{Symmetrisierung:} Viele Strukturen von Proteinen weisen Symmetrien auf.
Dabei treten folglich genau dieselben strukturellen Anordnungen auf, welche auch in der Dichte übereinstimmen sollten.
Es ist in solchen Fällen nicht nur möglich während des Rekonstruktionsprozesses die Partikel mit der Referenz zu vergleiche, sondern auch innerhalb der Rekonstruktion die symmetrischen Bereiche. 
Besitzt ein Partikel beispielsweise eine $C5$ Symmetrie und werden die symmetrischen Bereiche miteinander verglichen ist das etwa so, als wären 5 
mal so viele Partikel zur Rekonstruktion verwendet worden.
Sind symmetriebrechende Abschnitte der Struktur vorhanden, ist es nötig diese zuvor mit einer extra angepassten Maske auszuschneiden.
Eine Symmetrisierung von nicht symmetrische Bereiche führt nämlich zu einer Zerstörung der Struktur.
\\
\\
\textbf{KTF-Korrektur:} Die KTF beinhaltet alle Strukturinformationen des Partikels, doch sind besonders die hohen Frequenzen nicht einfach auszulesen.
Wird nicht viel Wert auf eine Hochauflösung gelegt ist es zum Beispiel möglich die KTF-Informationen ab der ersten Nullstelle abzuschneiden, sodass nur die große Strukturen sichtbar werden.
Da jedoch für die Hochaufösung die großen Frequenzen relevant sind, ist es unter anderem möglich das Vorzeichen der negativen Amplituden zu ändern (\textit{phase flip}).
Dadurch werden die KTFs, welche durch verschiedene Defokus Werte erzeugt worden sind, einheitlich und eine Ausrichtung der Partikel wird erleichtert (http://www.ncbi.nlm.nih.gov/pmc/articles/PMC3166661/).
Zudem können gesondert Struktureigene Frequenzen verstärkt werden (Siehe \textbf{PDB Modell}).
Eine weitere Methode ist die Modifikation des B-Faktors (Vergleich Kapitel \ref{sec:elektron_spektroskopie}).
Durch Anhebung der Intensität aller Frequenzen wird auch die Information der großen Frequenzen verstärkt, jedoch verschlechtert sich das Signal-zu-Rausch Verhältnis, da bei der Verstärkung nicht zwischen Information und Rauschen unterschieden werden kann.
\\
\\
\textbf{PDB Modell:}
Die PDB (\textit{Protein Data Bank}) beinhaltet 3D-Strukturdaten der verschiedensten Makromoleküle (Quelle http://www.rcsb.org/pdb/home/home.do).
Dabei handelt es sich unter anderem um Untereinheiten größerer Proteine, deren Architektur meist mithilfe der Röntgenkristallographie entschlüsselt worden ist.

PDBs (Strukturen, welche aus der Datenbank entnommen wurden) helfen bei der Strukturaufklärung größerer Proteine.
Handelt es sich um Strukturen desselben Proteins, sollten diese bei hoher Auflösung nahezu Perfekt in die Dichte der Rekonstruktion passen.
Meist genügt es jedoch, die homologe Struktur einer artverwandten Tierart zu betrachten, da diese sich häufig nur in einzelnen Aminosäuresequenzen unterscheidet.

Ist die Auflösung so hoch, dass bereits Strukturen in der Dichte erkennbar sind, ist es möglich die Röntgenstruktur mit dieser zu vergleichen.
Stimmen Dichte und Struktur quasi überein, können die Frequenzen der Struktur extrahiert werden und bei einer KTF Korrektur der Partikel besonders verstärkt werden.

\FloatBarrier
\chapter{Ergebnisse} 
\section{Negative Kontrastierung} % (fold)
\label{sec:negative_kontrastierung}

Bei der negativen Kontrastierung wurde nach der in Kapitel \ref{EPA} beschriebenen Einzelpartikelanalyse vorgegangen.
% Für die Bildgebung wurden am Mikroskop die in Tabelle \ref{neg_tab} dargestellten Daten eingestellt.
Die Metallgitter waren bereits vorbereitet, wobei zur Verdünnung der Probe folgender Puffer benutzt worden ist: TRIS \unit[50]{mM} pH 7.4, NaCl \unit[150]{mM}, MgCl$_2$ \unit[5]{mM}, CaCl$_2$ \unit[5]{mM}.
Als Schwermetalllösung ist \unit[0.75]{\%} Uranylformiat verwendet worden.
Eine Abbildung zur Auswertung ist in Abbildung \ref{stain_schema} dargestellt.

Zunächst werden aus den vom Mikroskop erzeugten Bildern die Partikel extrahiert, ausgerichtet und klassifiziert (Abbildung \ref{stain_schema}:a, b).
Dabei ist eine Boxgröße von \unit[140]{px} ausgewählt worden.
Eine erste Rekonstruktion aus den Klassensummen hat jedoch eine quaderartige Struktur ergeben, obwohl aus der Form der Partikel eine hohlzylindrische Struktur zu erwarten ist (Abbildung \ref{stain_schema}:c).
Dies ist durch Deformation der Partikel während des Trockenvorgangs zu erklären.
Durch Messen der Durchmesser der Seitenansichten und der Draufsichten ist dies verifiziert worden.
Es ist einer hohlzylindrischen Struktur zu erwarten, dass diese den gleichen Wert haben, jedoch sind die Durchmesser der Seitenansichten offensichtlich zu groß (Vergleich Abbildung \ref{stain_schema}).

\begin{figure}
	\includegraphics[width = 14cm, height = 5cm]{Abbildungen/stain2.png}
	\caption[Negativ: Weg zur ersten Rekonstruktion]{\textbf{a)} Ausrichtung der Partikel. \textbf{b)} Klassifizierte Partikel. \textbf{c)} Erste Rekonstruktion.}
	\label{stain_schema}
\end{figure}

% \begin{table}
% 	\begin{tabular}[h!]{l l l l l}
% 		Brightness & Defokus [cts] & Spot Size & Pixel Size [$\frac{\text{mm}}{\text{px}}$] & Voltage [kV]\\
% 		\hline
% 		2000 & [-1.5,-2.5] & 2 & 0.29 & 120\\
% 		\hline
% 		\hline
% 	\end{tabular}
% 	\caption[Einstellungen TEM negative Kontrastierung]{Einstellungen des Mikroskops zur Aufnahme der negativ Kontrastierten Probe. Für die Aufnahme wurde das JEOL JEM 1400 benutzt.}
% 	\label{neg_tab}
% \end{table}


\FloatBarrier
\section{Kryotransmissionselektronenmikroskopie} % (fold)
\label{sec:kryopraeparation}

Die Gitter mit der Probe sind bereits präpariert gewesen, wobei der Puffer wie folgt zusammengesetzt worden ist: 
(Puffer Rezeptur)

Nachdem einige Stellen mit der richtigen Eisdicke (Eisdicke jetzt hier) gefunden worden sind, war es möglich an ca. 800 Positionen Bilder aufzunehmen.
Während der Aufnahme ist darauf geachtet worden, dass das Power Spektrum nicht astigmatisch (elliptisch) oder verschwommen ist, um Fehler in der KTF zu minimieren.
Bei der Auswertung der Kryo-TEM Daten wurde nach der in Kapitel \ref{EPA} beschriebenen Einzelpartikelanalyse vorgegangen.
\\
\\
\textbf{Digitale Bildbearbeitung:}
Für jede der ca. 800 Positionen sind zunächst alle 15 Bilder miteinander verglichen, zueinander ausgerichtet und aufsummiert worden (Abbildung \ref{kryo_schema}:a).
Zudem werden die KTF-Parameter mithilfe der \textit{CTF-GUI} bestimmt und in die Kopfzeilen der Bilder geschrieben (Tabelle \ref{EDV}:CTF-GUI).
Abschließend werden die Bilder kopiert, um den Faktor zwei, zum schnelleren Verarbeiten verkleinert und sowohl Hoch-, als auch Tiefpass gefiltert, damit die Proteinstrukturen besser zu erkennen sind.
\\
\\
\begin{wrapfigure}{r}{0.2\textwidth}
	\centering
	\includegraphics[width =  0.2\textwidth]{Abbildungen/pdb2.png}
	\caption[Darstellung der Oktopus Röntgenkristallstruktur]{Verwendete PDB Struktur 1JS8 eines \textit{Oktopus}-Hämocyanins (\cite{pdb}).}
	\label{octo}
\end{wrapfigure}
\textbf{Extraktion der Partikel:}
Die zweifach verkleinerten Bilder sind mit einer Boxgröße von \unit[196]{px} ausgewählt worden, da diese nur geringfügig größer ist als das Protein selbst und der Mittelpunkt der Partikel leichter gefunden werden kann.
Für den Extraktionsprozess wird hingegen eine größere Boxgröße von \unit[256]{px} benutzt, da dies eine von EMAN empfohlene Größe ist.
Zudem werden Partikel, welche nicht ganz zentriert sind nicht abgeschnitten.
Die Koordinaten sind auf dem verkleinerten Datensatz ausgewählt worden, daher ist es nötig beim Wechsel auf den originalen Datensatz alle Koordinaten zu verdoppeln.
Dies führt dazu, dass sich auch die Boxgröße auf \unit[512]{px} vergrößert.
Anschließend werden die etwa 8000 Partikel invertiert, normalisiert und die Kopfzeile mit den aus den Bildern errechneten KTF-Informationen versehen.
Drei Partikel sind in der Abbildung \ref{kryo_schema}:b dargestellt.
\\
\\
\textbf{Ausrichtung und Klassifizierung:}
In die \textit{GUI} ist der Partikeldatensatz eingelesen worden (Tabelle \ref{EDV}:GUI).
Dem K-Means Algorithmus werden ersten Durchlauf zehn und im zweiten Durchlauf 40 Klassen übergeben.
Die Entwicklung einiger Klassen ist in Abbildung \ref{kryo_schema}:c dargestellt.
\\
\\
\textbf{3D Rekonstruktion:} 
Mithilfe der Klassensummen ist es gelungen, eine der Geometrie entsprechende Startreferenz zu erzeugen (Abbildung \ref{kryo_schema}:d).
Aus dieser wurden erste 2D Projektionen erzeugt und mit den extrahierten Partikeln verglichen.
Nach verschiedenen Herangehensweisen hat sich eine Rekonstruktion ergeben, welche insgesamt auf etwa \unit[9.3]{\AA} aufgelöst ist (Vergleich Kapitel \ref{sec:symmetrieproblem}).
\begin{figure}
	\includegraphics[width = 14cm, height = 5cm]{Abbildungen/kryo.png}
	\caption[Kryo: Weg zur ersten Rekonstruktion]{\textbf{a):} Aufsummieren der 15 Bilder zu einem. \textbf{b):} Ausrichtung der Partikel. \textbf{c):} Klassifizierte Partikel. \textbf{d):} Erste Rekonstruktion.}
	\label{kryo_schema}
\end{figure}
\\
\\
\textbf{PDB Modell:}
Zum Vergleichen der Dichte mit der Röntgenkristallstruktur ist die Struktur des homologe \textit{Oktopus}-Hämocyanins benutzt worden (Abbildung \ref{octo}, \cite{pdb}).
Zu Beginn werden die Strukturen mit der Dichte verglichen, um die am wenigsten aufgelösten Bereiche zu identifizieren. 
Hat sich die Auflösung soweit verbessert, dass feinere Strukturen in der Dichte sichtbar sind, ist es möglich, anhand der N'- und C'-Terme der einzelnen FEs einen strukturellen Zusammenhang und damit eine Topologie zu erkennen (Vergleich Kapitel \ref{sec:aufschluesselung_der_struktur}).

\FloatBarrier
\section{Trickreiche Symmetrieprobleme} % (fold)
\label{sec:symmetrieproblem}

\begin{figure}
	\includegraphics[width = 14cm, height = 6cm]{Abbildungen/sym_schem.png}
	\caption[Ablaufschema zur Symmetriebestimmung]{Ablaufschema zur Symmetriebestimmung. Mult: Verstärkung der Informationen, bin: grad der Verkleinerung des Datensatzes, C5: C5 Symmetrie.}
	\label{symmi}
\end{figure}

Um herauszufinden, welche Architektur das Hämocyanin des \textit{Sepia Officinalis} besitzt, ist nach dem in Abbildung \ref{symmi} dargestellten Schema vorgegangen worden.
Nach dem erzeugen der Startreferenz kann das Verfahren in vier Schritte unterteilt werden.
Dabei ist das Ziel der ersten drei Schritte eine gute Auflösung in allen Bereichen des Proteins zu erreichen.
Um dies zu realisieren werden die Partikel zunächst um den Faktor vier verkleinert um Rechenzeit zu sparen.
Dies führt zwar zu einer verschlechterten Auflösung, doch geht es zunächst um die Ausrichtung und Position der einzelnen FEs, wobei es sich um große Srukturen handelt.

Die originale reale Pixelgröße der Bilder beträgt \unit[0.83]{\AA/px}.
Wie in Kapitel \ref{sec:elektron_spektroskopie} kann damit eine maximale Auflösung von \unit[1.66]{\AA} erreicht werden.
Da die Größe des Bildes um den Faktor vier verkleinert wurde, vervierfacht sich damit auch die reale Pixelgöße auf \unit[3.32]{\AA/px} und somit ist die maximale Auflösung auf \unit[6.64]{\AA} beschränkt.

Der vierte Schritt dient dem Erkennen von Details und dazu wird auf einen zweifach verkleinerten Datensatz gewechselt.
Um die vorherigen Rekonstruktionen und Masken aus dem vierfach verkleinerten Datensatz nutzen zu können, muss deren Größe verdoppelt werden.
In diesem Schritt beträgt die reale Pixelgröße \unit[1.66]{\AA/px} und die Auflösung kann maximal \unit[3.32]{\AA} betragen.

Nach der Rekonstruktion kann das Volumen zudem noch mit einer sogenannten \textit{Benutzerfunktion} (engl.: \textit{Userfunction}) bearbeitet werden. In dieser werden vor allem Volumen ausgeschnitten, symmetrisiert und zusätzlich gefiltert um Rauschen zu unterdrücken.
\\
\\
\begin{wrapfigure}{r}{0.2\textwidth}
	\centering
	\includegraphics[width = 0.2\textwidth]{Abbildungen/inner_outer.png}
	\caption[Trennung des Proteins in zwei Bereiche]{Orange: Innerer Kragen; Blau: Äußere Wand}
	\label{i_o}
\end{wrapfigure}
\textbf{1. Schritt: Auflösung der äußeren Wand}\\
Zur Bestimmung der Architektur ist das Protein zunächst in 2 Bereiche unterteilt worden.
Einerseits in die äußere Wand und andererseits in den inneren Kragen (Abbildung \ref{i_o}).
Wie in Kapitel \ref{haemo} beschrieben, wird vermutet, dass die äußere Wand bei allen Hämocyaninen gleich aufgebaut ist.
Es ist davon auszugehen, dass es dadurch zur Bildung derselben räumlichen Anordnung kommen sollte und es wurde bereits bestätigt, dass in diesem Fall eine $C5$ Symmetrie vorliegt (Quelle).
Zur Überprüfung der $C5$ Symmetrie in der Wand wird dem Programm zunächst eine $C5$ Symmetrie für die Rekonstruktion übergeben.
Wenn keine $C5$ Symmetrie vorhanden wäre, würde die Symmetriesierung zu einer Zerstörung der Struktur führen, wie es im inneren Kragen der Fall gewesen ist.
Dies bestätigt die Vermutung aus Kapitel \ref{einleitung}, dass das Protein nicht $C5$ symmetrisch ist.
Im Vergleich dazu hat sich die Auflösung der Wand sichtbar verbessert, wodurch eine $C5$ Symmetrie in diesem Bereich bestätigt werden konnte.

Zudem wird eine noch bessere Auflösung durch eine KTF Korrektur und einer Anpassung des B-Faktors erreicht.
Während der gesamten Rekonstruktion ist eine alles umfassende Maske benutzt worden um nur das Rauschen außerhalb der Dichte zu unterdrücken, jedoch keine inneren Strukturen abzuschneiden.
\\
\\
\textbf{2. Schritt: Auflösung des inneren Kragens}\\
Nach der relativ guten Auflösung der äußeren Wand ist es nun von besonderem interesse, wie der innere Kragen zusammengesetzt ist.
Falls eine Symmetrie im inneren Kragen existiert, ist diese bisher nicht bekannt gewesen.
Daher wird während der nächsten Rekonstruktion dem Programm keine Symmetrieeigenschaft ($C1$) übergeben, sondern das erzeugte Volumen wird erst im Anschluss bearbeitet.
So werden für den nächsten Iterationsschritt bessere 2D Referenzprojektionen erhalten ohne auf den Rekonstruktionsprozess direkten Einfluss zu nehmen.

In den ersten Iterationen werden der Partikeln erste Projektionsrichtungen zugeordnet.
Sind diese vorhanden wird dazu übergegangen den äußeren Bereich nachträglich $C5$ zu symmetrisieren und die Inforamtion des innere Kragens durch eine Multiplikation zu besser auslesbaren niedrigeren Frequenzen zu verschieben.
Somit achtet das Programm beim nächsten Iterationsschritt bei der Rekonstruktion vermehrt auf diese Bereiche.
Dazu wird in der \textit{Benutzerfunktion} das Volumen mithilfe von Masken in die äußere Wand und den inneren Kragen aufgeteilt.

Zudem kann noch einmal überprüft werden, ob tasächlich eine $C5$ Symmetrie in der Wand vorliegt.
Da die Rekonstruktion ohne Symmetrie abläuft, wäre zu erwarten, dass sich bei falscher Symmetrisierung in der nächsten Iteration die Auflösung verschlechtert.
Diese hat sich jedoch verbessert, wodurch die $C5$ Symmetrie abermals bestätigt wurde.

Beim Teilen der Dichte in einen inneren und einen äußeren Bereich ist es zudem wichtig, welche Dichte ausgeschnitten wird.
Wird der äußere symmetrische Bereich ausgeschnitten, so verbleibt der unsymmetrische innere Kranz unverändert.
Da die Information zudem noch multiplikativ vergößert wird, wird auch das Rauschen im Rest des Bildes verstärkt.
Schneidet man hingegen den inneren Kragen aus und symmetrisiert den Rest, wird nicht nur die Auflösung der Wand verbessert, sondern auch das unsymmetrische Rauschen um die Dichte herum abgeschwächt.
Abschließend wird wie bei Schritt 1 die Auflösung durch KTF Korrektur und Anpassung des B-Faktors verbessert.
\\
\\
\textbf{3. Schritt: Auflösung eines unsymmetrischen Bereiches}\\
Im Laufe von Schritt 2 hat sich die Auflösung des inneren Bereiches soweit verbessert, dass die grobe Architektur sichtbar geworden ist.
Bei Betrachtung des Kragens fällt auf, dass vier Dichteeinheiten nicht so gut aufgelöst sind wie der Rest.
Zudem ist durch Rotation der Dichte zu erkennen, dass, bis auf den weniger gut aufgelösten Bereich, der innere Kragen eine $D1$ Symmetrie besitzt (Abbildung \ref{sym}).
Um auch im unsymmetrischen Bereich auf eine bessere Auflösung zu bekommen, wird eine ähnliche Methode wie in Schritt 2 angewendet.
Zunächst wird der Rekonstruktion keine Symmetrie übergeben, wobei auch die äußere Wand im Anschluss nicht symmetrisiert wird.
Dies könnte nämlich Auswirkungen auf die Referenzprojektionen des inneren Bereiches im nächsten Schritt haben.

In der \textit{Benutzerfunktion} wird nun anstatt des inneren, der unsymmetrische Bereich ausgeschnitten und dessen Information verstärkt.
Dabei gab es jedoch das Problem, dass sich der Bereich schnell verbessert hat. 
Die benutzte Maske passte daher nach dem ersten Iterationsschritt nicht mehr und um das Problem zu lösen ist nach jedem Iterationsschritt die Maske neu angepasst worden.
Nachdem nach einigen Iterationsschritten keine Anpassung der Maske mehr nötig war, konnte mit dem Rest der Iterationsschritte fortgefahren werden.

Nachdem sich die Auflösung des unsymmetrischen Bereiches verbessert hat wird eine erneute Iteration mit kleinen Winkelschritten durchgeführt.
Dabei wird wiederum das Volumen nicht nachträglich verändert, um die Projektionsrichtungen nicht zu beeinflussen.
Bevor das Volumen KTF korrigiert und eine Anpassung des B-Faktors vorgenommen wird, wird eine Iteration mit symmetriesierung der äußeren Wand vorgenommen.
\\
\begin{wrapfigure}{r}{0.2\textwidth}
	\centering
	\includegraphics[width = 0.2\textwidth]{Abbildungen/symmetrie.png}
	\caption[Trennenung des Proteins in Bereiche mit verschiedenen Symmetrien]{Blau: Äußere Wand mit $C5$ Symmetrie; Rot: Innerer Kragen mit $D1$ Symmetrie; Hellblau: Innerer Bereich ohne Symmetrie.}
	\label{sym}
\end{wrapfigure}
\textbf{4. Schritt: Höhere Auflösung}\\
Mit einem vierfach verkleinerten Datensatz ist der Auflösung eine relativ niedrige Grenze gesetzt worden.
Nachdem die grobe Struktur gelöst ist, können nun die Details der Dichte verbessert werden.
Dazu wird ein zweifach verkleinerter Datensatz erzeugt, da der originale Datensatz die Rechenzeiten erheblich verlängern würde, ohne das das Ergebnis anfangs deutlich besser wird.

Um die zuvor erreichten Ergebnisse verwenden zu können, muss die Größe der Referenzrekonstruktioen der letzten Iteration verdoppelt werden.
Da die Architektur bereits gelöst ist, können die Frequenzen den PDB Modellen entnommen werden.
Damit ist gleich sofort in der ersten Iteration eine KTF Korrektur möglich.
Abschließend ist noch der B-Faktor angepasst worden, sodass eine verbesserte Auflösung von insgesamt \unit[9.5]{\AA} erreicht werden konnte.
\FloatBarrier 
\section{Aufschlüsselung der Struktur} % (fold)
\label{sec:aufschluesselung_der_struktur}
\textbf{Äußere Wand:}\\
Eine Analyse der äußeren Wand mithilfe der homologen Kristallstrukturen lässt auf die in Abbildung \ref{aussen} dargestellte Architektur schließen.
Die Strukturen sind, wie in der Abbildung \ref{aussen}:Unten dargestellt, vollständig von der Dichte umschlossen.
Wären die FEs anders als angenommen angeordnet, würde dies zu größeren Lücken oder herausstehenden Strukturabschnitten führen.
Dies bestätigt, dass die Wand des \textit{Sepia}-Hämocyanins aus den in Kapitel \ref{haemo} beschriebenen Sequenzen aufgebaut ist.
\\
\\
\textbf{Innerer Kragen:}\\
Mithilfe der in Abbildung \ref{seq}:Unten dargestellten Sequenz des \textit{Sepia}-Hämocyanins und der bereits herausgefunden Architektur der Wand, konnten die FEs des inneren Kragen identifiziert werden.
Um dies zu erreichen, sind die N'- und C'-Terme der einzelnen FEs untersucht worden.
Der N'-Term stellt den Startpunkt einer Aminosäuresequenz dar.
Dieser muss mit dem Ende (C'-Term) der vorherigen FE zu verbinden sein.
\begin{wrapfigure}{r}{0.38\textwidth}
	\centering
	\includegraphics[width = 5cm]{Abbildungen/aussen.png}
	\caption[Architektur der äußeren Wand]{\textbf{Oben:} Dichtevolumen des Hämocyanins. Farblich hervorgehoben ist einer der fünf strukturell gleich aufgebauten Bereiche der äußeren Wand, bestehend aus zwei Untereinheiten. Rot - FE-a; Hellgrün - FE-b; Orange - FE-c; Hellblau - FE-d; Dunkelblau - FE-e; Dunkelgrün - FE-f. \textbf{Unten:} Dichte des Hämocyanins inklusive der homologen Kristallstruktur (\cite{pdb}).}
	\label{aussen}
	\centering
	\includegraphics[width = 5cm]{Abbildungen/innen2.png}
	\caption[Architektur des inneren Kragens]{Lila - FE-g, welche sich zu Dimeren zusammenfindet. Rot - Fe-d', welche die Lücken zwischen den FE-g Dimeren ausfüllt. \textbf{Oben:} Aufgeklappte Seitenansicht der inneren Struktur. \textbf{Unten:} Draufsicht von der oberen und der unteren Seite des Proteins.}
	\label{innen}
\end{wrapfigure}
\FloatBarrier
Diese Verbindungen (\textit{linker}) haben eine definierte Länge, weshalb nicht alle Kombinationen von FE  in jedem Protein möglich sind.
Die Anordnung der FE-g und FE-d' sind in Abbildung \ref{innen} dargestellt.
Es ist zu erkennen, dass sich jeweils zwei FE-g innerhalb der Struktur zu einem Dimer zusammenschließen, während die FE-d' in den Lücken dazwischen liegen.
\\
\\
\textbf{Konformation der Untereinheiten}\\
\begin{figure}[h!]
\includegraphics[width = 14cm]{Abbildungen/Konformer2.png}
\caption[Architektur des \textit{Sepia}-Hämocyanins]{Architektur des \textit{Sepia}-Hämocyanins. \textbf{Oben links}: Schematische Anordnung der funktionellen Einheiten. \textbf{Oben rechts}: Darstellung der fünf verschiedenen Konformationen der Untereinheiten inklusive der homologen Kristallstrukturen (\cite{pdb}). \textbf{Unten:} Drauf- und Seitenansichten des Hämocyanins bei farblicher Markierung der verschiedenen Konformationen. Rot - Untereinheit b; Blau - Untereinheit c; Grün - Untereinheit e; Gelb - Untereinheit d; Lila - Untereinheit a.}
\label{done}
\end{figure}
Bei der Untersuchung des inneren Kragen ist zunächst versucht worden, auf Basis der postulierten Struktur des \textit{Oktopus}-Hämocyanins eine Architektur zu bestimmen (Abbildung \ref{seq}:Oben).
Eine Übereinstimmung der Kristallstrukturen in der rekonstruierten Dichte konnte jedoch nicht eindeutig festgestellt werden (Siehe \textbf{Innerer Kragen}.
Daher wurde der in Kapitel \ref{einleitung} beschriebene Ansatz der Überkreuzung zweier Untereinheiten überprüft (Abbildung \ref{done}):Oben links).
Die FEs passen bei dieser Möglichkeit des Strukturaufbaus offensichtlich besser zueinander, da die Kristallstrukturen durch die Verbindungsglieder verbunden werden könnten.
Mit dieser neuen Struktur zeigt sich, dass die zehn Untereinheiten in fünf verschiedenen Konformationen vorliegen (Abbildung \ref{done}:Oben rechts).
Dabei kommt Typ a viermal, Typ b einmal, Typ c dreimal, Typ d einmal und Typ e einmal vor.
Die Verteilung der verschiedenen Konformationen ist in Abbildung \ref{done}:Unten farblich dargestellt, wobei jede Farbe einem Typ entspricht.
Es ist zu erkennen, dass sich das Protein in zwei Bereiche unterteilt, einen geordneten und einen ungeordneten.
Dies ist durch den unsymmetrischen Bereich im inneren Kragen zu erklären, welcher sich genau im Zentrum der ungeordneten Konformationen befindet.
\chapter{Zusammenfassung und Ausblick}

Hier sollen die Ergebnisse zusammengefasst und weiterf\"uhrende Untersuchungen diskutiert werden. 

% >>> Anhang <<<

\begin{appendix}
%\input{Kapitel/Anhang}
\end{appendix}

% >>> Literaturverzeichnis <<<
\renewcommand{\bibname}{Quellenverzeichnis}
\addcontentsline{toc}{chapter}{\bibname}
\bibliographystyle{unsrt}
\bibliography{BachelorArbeit}

\newpage
\thispagestyle{empty}
\ \\

% >>> Erklaerung <<<

\thispagestyle{empty}
\begin{center}
\section*{Eidesstattliche Versicherung}
\end{center}
\vspace*{1cm}
\noindent
Ich versichere hiermit an Eides statt, dass ich die vorliegende Bachelorarbeit mit dem Titel ''{\thetitle}'' selbst\"andig und ohne unzul\"assige fremde Hilfe erbracht habe. Ich habe keine anderen als die angegebenen
Quellen und Hilfsmittel benutzt sowie w\"ortliche und sinngem\"a\ss e Zitate kenntlich gemacht.
Die Arbeit hat in gleicher oder \"ahnlicher Form noch keiner Pr\"ufungsbeh\"orde vorgelegen.
\vspace*{1cm}
\ \\
\ \\
\line(1,0){150} \hfill \line(1,0){150}\\
Ort, Datum \hfill Unterschrift \hspace*{3cm}
\vspace*{1.5cm}

\subsection*{Belehrung}
Wer vors\"atzlich gegen eine die T\"auschung \"uber Pr\"ufungsleistungen betreffende Regelung einer Hochschulpr\"ufungsordnung
verst\"o\ss t handelt ordnungswidrig. Die Ordnungswidrigkeit kann mit einer Geldbu\ss e von bis zu \unit[50.000,00]{\euro} geahndet werden. Zust\"andige Verwaltungsbeh\"orde f\"ur die Verfolgung und Ahndung von Ordnungswidrigkeiten ist
der Kanzler/die Kanzlerin der Technischen Universit\"at Dortmund. Im Falle eines mehrfachen oder sonstigen schwerwiegenden T\"auschungsversuches kann der Pr\"ufling zudem exmatrikuliert werden (\S\ 63 Abs. 5 Hochschulgesetz - HG - ).\\
\ \\
Die Abgabe einer falschen Versicherung an Eides statt wird mit Freiheitsstrafe bis zu 3 Jahren oder mit Geldstrafe bestraft.\\
\ \\
Die Technische Universit\"at Dortmund wird ggf. elektronische Vergleichswerkzeuge (wie z.B. die Software ''turnitin'') zur \"Uberpr\"ufung von Ordnungswidrigkeiten in Pr\"ufungsverfahren nutzen.\\
\ \\
Die oben stehende Belehrung habe ich zur Kenntnis genommen.
\vspace*{1cm}
\ \\
\ \\
\line(1,0){150} \hfill \line(1,0){150}\\
Ort, Datum \hfill Unterschrift \hspace*{3cm}
\vspace*{\fill}

\end{document} 