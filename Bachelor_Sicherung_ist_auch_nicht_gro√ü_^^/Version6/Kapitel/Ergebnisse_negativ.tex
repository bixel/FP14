\section{Negative Kontrastierung} % (fold)
\label{sec:negative_kontrastierung}

Bei der negativen Kontrastierung wurde nach der in Kapitel \ref{EPA} beschriebenen Einzelpartikelanalyse vorgegangen.
% Für die Bildgebung wurden am Mikroskop die in Tabelle \ref{neg_tab} dargestellten Daten eingestellt.
Die Metallgitter waren bereits vorbereitet, wobei zur Verdünnung der Probe folgender Puffer benutzt worden ist: TRIS \unit[50]{mM} pH 7.4, NaCl \unit[150]{mM}, MgCl$_2$ \unit[5]{mM}, CaCl$_2$ \unit[5]{mM}.
Als Schwermetalllösung ist \unit[0.75]{\%} Uranylformiat verwendet worden.
Eine Abbildung zur Prozessierung ist in Abbildung \ref{stain_schema} dargestellt.

Zunächst werden aus den vom Mikroskop erzeugten Bildern die Partikel extrahiert, ausgerichtet und klassifiziert (Abbildung \ref{stain_schema}: a
,b ).
Dabei ist eine Boxgröße von \unit[140]{px} ausgewählt worden.
Eine erste Rekonstruktion aus den Klassensummen hat jedoch eine quaderartige Struktur ergeben, obwohl aus der Form der Partikel eine hohlzylindrische Struktur zu erwarten ist.
Dies ist durch Deformation der Partikel während des Trockenvorgangs zu erklären.
Durch Messen der Durchmesser der Seitenansichten und der Draufsichten ist dies verifiziert worden.
Es ist einer hohlzylindrischen Struktur zu erwarten, dass diese den gleichen Wert haben, jedoch sind die Durchmesser der Seitensansichten offensichtlich zu groß (Vergleich Abbildung \ref{stain_schema}).

\begin{figure}
	\includegraphics[width = 14cm, height = 5cm]{Abbildungen/stain2.png}
	\caption[Negativ: Weg zur ersten Rekonstruktion]{\textbf{a):} Ausrichtung der Partikel. \textbf{b):} Klassifizierte Partikel. \textbf{c):} Erste Rekonstruktion.}
	\label{stain_schema}
\end{figure}

% \begin{table}
% 	\begin{tabular}[h!]{l l l l l}
% 		Brightness & Defokus [cts] & Spot Size & Pixel Size [$\frac{\text{mm}}{\text{px}}$] & Voltage [kV]\\
% 		\hline
% 		2000 & [-1.5,-2.5] & 2 & 0.29 & 120\\
% 		\hline
% 		\hline
% 	\end{tabular}
% 	\caption[Einstellungen TEM negative Kontrastierung]{Einstellungen des Mikroskops zur Aufnahme der negativ Kontrastierten Probe. Für die Aufnahme wurde das JEOL JEM 1400 benutzt.}
% 	\label{neg_tab}
% \end{table}


\FloatBarrier