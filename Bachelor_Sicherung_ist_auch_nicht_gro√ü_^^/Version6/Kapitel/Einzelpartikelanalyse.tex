\section{Einzelpartikelanalyse}
\label{einzelpartikelanalyse}
(Selbst erstellte Abbildung des Verlaufs)

Ein Problem bei der rekonstruktion eines 3D Modells mithilfe eines Transmissionselektronenmikroskops ist, dass sich nur 2D Abbildungen der 3D Elektronendichte erzeugen lassen.
Es ist daher eine Rückprojektion nötig, um das Objekt darstellen zu können.
Eine etablierte Methode für Rücktransformationen ist die Tomographie, doch ist diese bei biologischen Proben nicht vorteilhaft, da die relativ hohe Elektronendosis das Material zerstört und nicht alle Raumwinkel abgedeckt werden können.
Alternativ wird daher auf die Einzepartikelanalyse zurückgegriffen.
Für diese ist es notwendig, dass die Probe ungeordnet vorliegt, sodass viele 2D Projektionen bei verschiedenen Projektionsparametern der Probe auftreten.
Diese sind neben den drei Euler-Winkeln $\Psi$, $\Phi$ und $\Theta$ auch die Verschiebung in x- und y-Richtung, da die ausgewählten Partikel nicht immer im Schwerpunkt des Bildes liegen.
Während bei der Tomographie diese bekannt sind, müssen bei dieser Methode diese erst iterativ berechnet werden.
Die Methode Arbeitet nach dem in Abbildung \ref{Vorgehensweise} dargestellten Schema.

Von einer 3D Referenzrekonstruktion werden in festgelegten Winkelabständen eine definierte Anzahl an 2D Projektionen erstellt.
Jeder Partikel wird mit den Projektionen verglichen, indem die Projektionsparamter des Partikels variiert werden (\textit{multi reference alignment}). 
Anschließend wird er der am besten passenden Projektionsrichtung zugeordnet.
Wurde zu jedem Partikel eine Projektionsrichtung gefunden wird eine Rekonstruktion mit diesen Parametern erstellt.
Das erstellte Volumen kann anschließend noch symmetrisiert und gefiltert werden.
Abschließend wird das Volumen als Referenz für den nächsten Iterationsschritt übergeben.

Als Startreferenz dient zumeist ein der Geometrie der Probe angepasstes Inertialmodell.
Ist keine Information über die Geometrie vorhanden, wird ein Elipsoid verwendet, der mit gaußschem Rauschen gefüllt ist.
Da bisher keine Strukturen vorhanden sind, wird ohne Einschränkung der Parameter eine globale Suche mit großer Schrittweite durchgeführt.
Es folgt eine lokale Umgebungssuche mit kleiner Schrittweite.

Eine Verbesserung der Rekonstruktion lässt sich zudem durch eine Korrektur der Kontrast-Transfer-Funktion erreichen.
Dies ist zum Beispiel durch das Abschneiden der Information ab der ersten Nullstelle der KTF möglich. 
Jedoch ist gerade die Information der kleinen Frequenzen bei der Hochauflösung relevant.
Eine andere Möglichkeit ist das Umklappen der negativen Bereiche in den positiven Bereich (\textit{phase flipping}).
Zudem können gesondert Struktureigene Frequenzen verstärkt werden (\textit{pw adjustment}).

Zudem können die kleinen Frequenzen mithilfe des B-Faktors verstärkt werden (Vergleich Kapitel \ref{sec:elektron_spektroskopie}). 
Jedoch verschlechtert sich das Signal-zu-Rausch Verhältnis, da bei der Verstärkung nicht zwischen Information und Rauschen unterschieden werden kann.

\FloatBarrier