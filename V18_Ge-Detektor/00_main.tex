\documentclass[%
    a4paper,%
    twoside,%
    BCOR12mm,%
    DIV13,%
    %cleardoubleplain,%
    bibtotoc,%
    openany,%
]{scrartcl}
	\usepackage[ngerman]{babel}
    \usepackage[german]{isodate}
    \usepackage{fontspec}

    \KOMAoptions{draft=on}

    \usepackage{subcaption}

	\usepackage{siunitx}
	\sisetup{
	    locale=DE,
	    separate-uncertainty=true,
        per-mode=symbol,
        range-phrase=%
             \ifmmode\mathbin{-}
             \else%
             { bis }%
             \fi,%
        math-micro=\text{µ},
        text-micro=µ
	}
    \DeclareSIUnit\clight{\ensuremath{c}}

    \usepackage{amsmath}
    \usepackage{unicode-math}

    \usepackage{xfrac}

    \usepackage{booktabs}

    \usepackage{hyperref}
    \hypersetup{
        colorlinks,
        citecolor=black,
        filecolor=black,
        linkcolor=black,
        urlcolor=black
    }

    \usepackage{indentfirst}
	% Texteinzug vor Absatz
	\parindent 12pt
    % Schönere Captions
    \usepackage[font=footnotesize,labelfont=bf]{caption}
    \setcapindent{10pt}

    \usepackage[olines]{scrpage2}
    \pagestyle{scrheadings}
    \clearscrheadfoot
    \ofoot[\pagemark]{\pagemark}

    \usepackage{cite}
    
    \usepackage[]{placeins}
    \usepackage{wrapfig}

    \title{%
        V18 - Der Reinst-Germanium-Detektor als Instrument der Gamma-Spektroskopie
    }
    \author{Kevin Heinicke and Markus Stabrin}
    \date{16. Juni 2014}
\begin{document}
    \maketitle%
    \tableofcontents

    \section{Theoretische Grundlagen}
\label{sec:theoretische_grundlagen}
Atome können in verschiedene Energieniveaus angeregt werden. Dabei werden
Elektronen aus ihrem Grundzustand in einen Zustand höherer Energie gebracht.
Die niedrigsten Niveaus werden dabei als S- und P-Schalten bezeichnet.  Das
Verhältnis der Besetzungszahlen $N_1$ und $N_2$ dieser Zustände ohne äußeres
Magnetfeld wird dabei durch die Boltzmann-Verteilung beschrieben:
\begin{equation}
\label{eq:boltzmann}
    \frac{N_2}{N_1} = \frac{g_2}{g_1}\cdot
        \mathrm{e}^\frac{W_1 - W_2}{k_\text{B}T}\,.
\end{equation}
Dabei bezeichnet $g_i$ ein jeweiliges statistisches Gewicht, $W_i$, die
jeweilige Energie, $T$ die Temperatur und $k_\text{B}$ die bekannte
Boltzmann-Konstante.

Diese Verteilung kann durch sogenanntes optisches Pumpen, worauf im Abschnitt
\ref{subsec:optisches_pumpen} näher eingegangen wird, beeinflusst werden.
Durch Einstrahlung von Licht der Wellenlänge
\begin{equation}
\label{eq:wellenlaenge}
    h\nu = W_2 - W_1
\end{equation}
können Übergänge vom niedrigeren Energieniveau $W_1$ in das höhere Niveau
$W_2$ angeregt werden. Außerdem können Elektronen spontan oder induziert in ein
niedrigeres Niveau fallen, wobei ein neues Quant der Energie
\ref{eq:wellenlaenge} abgestrahlt wird.

\subsection{Zeeman-Effekt und Hyperfeinstruktur}
\label{subsec:zeemanneffekt_und_hyperfeinstruktur}
Die Energiestruktur von Atomen ist unter anderem durch den Gesamtdrehimpuls
$\vec{L}$ und Spin $\vec{S}$ der Elektronenhülle, sowie den Kernspin $\vec{I}$
bestimmt. Dabei sind mit diesen Drehimpulsen die magnetischen Momente
\begin{align}
    \vec{\mu}_L &= -\mu_\text{B}\vec{L}\,,
        \label{eq:magn_moment_elektronen}\\
    \vec{\mu}_S &= -g_S\mu_\text{B}\vec{S}\,,
        \label{eq:magn_moment_espin}\\
    \text{und}\qquad\vec{\mu}_I &= -g_I\mu_\text{B}\vec{I}
        \label{eq:magn_moment_kernspin}
\end{align}
verknüpft, wobei $g_I$ und $g_S$ die Landé-Faktoren bezeichnen und
$\mu_\text{B}$ das Bohrsche Magneton.
Der Gesamtdrehimpuls der Elektronenhülle koppelt dabei an den Kernspin,
was zur sogenannten Hyperfeinstruktur führt.
Zusätzlich kann der gesamte Drehimpuls $\vec{F} = \vec{J} + \vec{I}$ an äußere
magnetische Felder koppeln, was zu einer weiteren Aufspaltung in $\num{2}F +
\num{1}$ Niveaus für jedes Niveau der Hyperfeinstruktur führt.
Dieser Effekt ist in Abbildung \ref{fig:aufspaltung} schematisch dargestellt.
\begin{figure}
    \centering
    \includegraphics[width=0.8\linewidth]{img/aufspaltung.pdf}
    \caption{
        Energieaufspaltung eines Alkali-Atoms mit Kernspin $I=\sfrac{3}{2}$
        \cite{V21}.
    }
    \label{fig:aufspaltung}
\end{figure}
Die Energie benachbarter Zeeman-Niveaus unterscheidet sich dabei jeweils um
\begin{equation}
\label{eq:zeeman_energie}
    U_\text{HF} = g_F \mu_\text{B} B\,,
\end{equation}
mit dem Landé-Faktor $g_F$ und einem äußeren Magnetfeld der Feldstärke $B$.
Diese Differenz liegt üblicherweise im Bereich von einigen
\si{\micro\electronvolt} und ist damit wesentlich kleiner als andere
Energieskalen der Atomstruktur.
Aus der vektoriellen Betrachtung der magnetischen Momente und deren
Richtungsquantelung folgt für den Landé-Faktor des Atoms
\begin{equation}
\label{eq:lande_faktor}
    g_F = g_J \frac{F(F+1) + J(J+1) - I(I+1)}{2F(F+1)}\,.
\end{equation}
Da $g_S$ bekannt ist, kann ebenso $g_J$ bestimmt werden zu
\begin{equation}
\label{eq:gj}
    g_J = \frac{\num{3.0023}\cdot J(J+1)
                + \num{1.0023}\cdot\left[S(S+1) - L(L+1)\right]}
               {2J(J+1)}
          \,.
\end{equation}
Mit Hilfe des hier vorgestellten Versuchs soll der oben genannte Landé-Faktor
$g_F$ bestimmt werden.

\subsection{Optisches Pumpen}
\label{subsec:optisches_pumpen}
% Da der Energiebereich der Zeeman-Aufspaltung westentlich kleiner als der des
% sichtbaren Lichtes ist, lassen sich Übergänge zwischen diesen Niveaus nicht
% direkt messen.
Übergänge zwischen zwei Energieniveaus eines Atoms können auf drei
verschiedenen Wegen von Statten gehen. Unter Absorption eines Photons mit einer
Energie, die der Lücke zweier Niveaus entspricht kann ein Elektron in das
jeweils höhere Niveau angeregt werden.  Ein bereits angeregtes Elektron kann
spontan unter Emission eines Photons in ein niedrigeres Niveau fallen.
Schließlich kann ein Übergang von einem höheren Niveau $W_2$ in ein niedrigeres
Niveau $W_1$ durch Einstrahlung eines Photons mit der Energie $W_2 - W_1$
stimuliert werden.  Dabei wird ein zweites Photon mit denselben Eigenschaften
des eingestrahlten Photons emittiert.

Für diese Übergänge existieren auf Grund der Quanteneigenschaften der
Elektronen bestimmte Auswahlregeln.  Unter Einstrahlung von rechtszirkular
polarisiertem D1-Licht ($\sigma^+$-Licht) werden beispielsweise lediglich
Übergänge angeregt, für die $\Delta M = \num{+1}$ gilt.  $M$ ist dabei die
Quantenzahl der Zeeman-Aufspaltung.  Unter dieser Voraussetzung lässt sich ein
Niveau $W_1$ entleeren, wenn es keine niedrigeren Niveaus gibt, aus denen $W_1$
unter Beachtung von $\Delta M = \num{+1}$ erreicht werden kann, jedoch höhere
Niveaus mit dieser Eigenschaft existieren, die folglich befüllt werden.

Genau diese Bedingungen herrschen bei den hier untersuchten Atomen.
Durch Einstrahlung von $\sigma^+$-Licht lässt sich eine Besetzungsinversion
herstellen.


    \FloatBarrier
    \bibliographystyle{unsrt}
    \bibliography{00_lit}
\end{document}
