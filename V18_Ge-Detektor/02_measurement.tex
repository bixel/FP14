\section{Messung}
\label{sec:messung}
Im Folgenden Abschnitt wird zunächst eine Energiekalibration der Apparatur durchgeführt. Anschließend werden mit der Aufnahme eines Cäsium-Strahlers verschiedene Detektoreigenschaften bestimmt.
Daraufhin wird die Aktivität einer ???-Quelle gemessen und schließlich einige Holzkohlebriketts auf ihre radioaktiven Bestandteile untersucht.

\subsection{Eichung und Effizienz des Ge-Detektors}
\label{subse:eichung}
Wie in Abschnitt \ref{sub:bestimmung_der_energie_und_der_aktivität_einer_gamma_quelle} erwähnt, liegen nach der Messung lediglich Informationen über einen Energiekanal vor, in dem ein Ereignis detektiert wurde.
Um das entsprechende Energiespektrum zu erhalten, wird den Kanälen $C$ mit einer lineare Kalibrationsfunktion $E(C)$ eine jeweilige Energie $E$ zugeordnet.
Mit der Steigung $m$ und einem Offset $b$ wird hierfür verwendet:%
%
\begin{align}
    \label{eqn:kalibration}
    E(C) = mC + b\,.
\end{align}

Zunächst wird das Spektrum eines $^{52}$Eu-Strahlers aufgenommen (Abb. \ref{fig:eu_uncalibrated}).
Um den Einfluss von Untergrundereignisse zu verringern, wird zudem eine Leermessung durchgeführt und die Einträge dieser Messung der einzelnen Kanäle von dem Spektrum des $^{52}$Eu-Strahlers abgezogen (Abb. \ref{fig:leermessung}).
Die unterschiedliche Messdauer wird mit einem Faktor $t_\text{leer} / t_\text{Eu}$ der Messdauer $t_\text{Eu}$ des $^{52}$Eu-Strahlers und $t_\text{leer}$ der Leermessung berücksichtigt.
Dabei werden die in Tabelle \ref{tab:maxima} aufgeführten Maxima gewählt und den jeweiligen Kanäle mit Hilfe von \eqref{eqn:kalibration} eine Energie zugeordnet. Der Fit der Kalibratinosfunktion ist in Abbildung \ref{fig:calibration} dargestellt.
Es ergeben sich die Koeffizienten
\begin{align*}
     m = \SI{345.22+-0.04}{keV/C} \qquad b = \SI{-1.82+-0.09}{keV} \,.
\end{align*}

\begin{table}[htp]
    \caption{maxima and stuff}
    \label{tab:maxima}
    \begin{tabular}{c}
        bla
    \end{tabular}
\end{table}
