\section{Messung}
\label{sec:messung}
Im Folgenden Abschnitt wird zunächst eine Energiekalibration der Apparatur
durchgeführt. Anschließend werden mit der Aufnahme eines Cäsium-Strahlers
verschiedene Detektoreigenschaften bestimmt.
Daraufhin wird die Aktivität einer $^{137}$Ba-Quelle gemessen und schließlich einige
Holzkohlebriketts auf ihre radioaktiven Bestandteile hin untersucht.

\subsection{Eichung und Effizienz des Ge-Detektors}
\label{subse:eichung}
Wie in Abschnitt \ref{sub:bestimmung_der_energie_und_der_aktivität_einer_gamma_quelle}
erwähnt, liegen nach einer Messung mit dem Ge-Detektor lediglich Informationen
über einen Energiekanal vor, in dem ein Ereignis detektiert wurde.
Um das entsprechende Spektrum in der Dimension \si{keV} der Energie zu erhalten,
wird den Kanälen $C$ mit einer lineare Kalibrationsfunktion $E(C)$ eine jeweilige
Energie $E$ zugeordnet.
Mit der Steigung $m$ und einem Offset $b$ wird hierfür verwendet:%
%
\begin{equation}
    \label{eqn:kalibration}
    E(C) = mC + b\,.
\end{equation}

Zunächst wird das Spektrum eines $^{52}$Eu-Strahlers aufgenommen (Abb.
\ref{fig:eu_uncalibrated}).
Um den Einfluss von Untergrundereignisse zu verringern, wird zudem eine
Leermessung durchgeführt und die Einträge dieser Messung der einzelnen Kanäle
von dem Spektrum des $^{52}$Eu-Strahlers abgezogen.
\begin{figure}[htb]
    \centering
    \begin{subfigure}{.49\linewidth}
        \includegraphics[width=1.0\linewidth]{img/02_maxima.pdf}
        \caption{
            Energiespektrum des $^{52}$Eu-Strahlers.
        }
        \label{fig:eu_uncalibrated}
    \end{subfigure}%
    \begin{subfigure}{.49\linewidth}
        \includegraphics[width=1.0\linewidth]{img/03_calibration.pdf}
        \caption{
            Fit der Kalibrationsfunktion
        }
        \label{fig:calibration}
    \end{subfigure}
    \caption{
        Energiespektrum des $^{52}$Eu-Strahlers und Kalibrationsfunktion.
        Die Zur Kalibration genutzen Maxima sind in \ref{fig:eu_uncalibrated}
        durch einen Pfeil markiert.
    }
\end{figure}
Die unterschiedliche Messdauer wird mit einem Faktor $t_\text{leer} / t_\text{Eu}$
der Messdauer $t_\text{Eu}$ des $^{52}$Eu-Strahlers und $t_\text{leer}$ der
Leermessung berücksichtigt.
Nach allen Messungen wird das $^{52}$Eu-Spektrum erneut vermessen,
um sicherzustellen, dass sich die Kanäle nicht verschoben haben, was in Abbildung
\ref{fig:control} verdeutlicht wird.
\begin{figure}
    \centering
    \includegraphics[width=0.7\linewidth]{img/01_comp.pdf}
    \caption{
        Vergleich der Kalibrationsspektren vor und nach der Messung.
    }
    \label{fig:control}
\end{figure}
Für die Kalibration werden die in Tabelle \ref{tab:maxima} aufgeführten
Maxima gewählt und den jeweiligen Kanäle mit Hilfe von \eqref{eqn:kalibration}
eine Energie zugeordnet. Der Fit der Kalibratinosfunktion ist in Abbildung
\ref{fig:calibration} dargestellt. Es ergeben sich die Koeffizienten
%
\begin{align*}
     m = \SI{345.22+-0.04}{eV/C} \qquad b = \SI{-1.82+-0.09}{keV} \,.
\end{align*}

Nach der Kalibration des Gerätes, kann die Effizienz $Q$ wie in Abschnitt
\ref{sub:bestimmung_der_energie_und_der_aktivität_einer_gamma_quelle}
beschrieben bestimmt werden. Mit Gleichungen \eqref{eqn:raumwinkel} ergibt
sich ein Raumwinkelanteil des Detektors von
\begin{align*}
    \frac{\Omega}{4\pi} = \SI{1.575+-0.017}{\percent}\,.
\end{align*}
Die Gesamtaktivität $A_\text{ges}$ des Stahlers am Versuchstag berechnet
sich mit Hilfe der bekannten Halbwertszeit $t_{1/2} = \SI{4943+-5}{d}$, der
Anfangsaktivität $A_0 = \SI{4130+-60}{\becquerel}$ und der Zerfallszeit $\Delta
t = \SI{5006}{d}$ zu
\begin{align*}
    A_\text{ges} = \SI{1500+-22}{\becquerel}\,.
\end{align*}
Für die Aktivität $A$, die im Detektor gemessen werden kann, gilt schließlich
\begin{align*}
    A &= \frac{\Omega}{4\pi}\cdot A_\text{ges}\,\\
    \Rightarrow \quad A &= \SI{23.6+-0.4}{\becquerel}\,.
\end{align*}
Unter Hinzunahme der bekannten Emissionswahrscheinlichkeiten kann nun wie
in Gleichung \eqref{eqn:effizienz} beschrieben, eine Effizienzfunktion $Q(E)$
für den Detektor aufgestellt werden.
Eine Funktion $Q(E)$ der Form
\begin{align*}
    Q(E) = ae^{-bE}
\end{align*}
wird anschließend an die so erhaltenen Wertepaare $(Q,E)$ angepasst. Die oben
aufgeführte Exponentialfunktion für $Q$ liefert im Gegensatz zu der im Skript
vorgeschlagenen Potenzfunktion kleinere Fehler des Fits und wird daher bevorzugt.
Der Fit ist in Abbildung \ref{fig:efficiency_fit} dargestellt und ergibt
\begin{align*}
    a &= \num{0.63+-0.26}\,, &\quad b &= \SI{1.8+-1.1}{eV^{-1}}\,.
\end{align*}

\begin{figure}[htb]
    \centering
    \includegraphics[width=0.7\linewidth]{img/05_efficiencies.pdf}
    \caption{
        Fit der Effizienzfunktion $Q$.
    }
    \label{fig:efficiency_fit}
\end{figure}

\begin{table}[htb]
    \centering
    \caption{
        Die für die Kalibration des Ge-Detektors verwendeten Maxima des $^{52}$Eu-Spektrums.
    }
    \label{tab:maxima}
    \begin{tabular}{%
        S[table-format=4.2]%
        S[table-format=2.1]%
        S[table-format=4.0]%
        S[table-format=5.0]%
        S[table-format=5.0]%
        S[table-format=2.1]%
    }
        \toprule
        {Energie [\si{keV}]} &
        {Emissionsw. [\si{\percent}]} &
        {Kanal} &
        {gemessen} &
        {erwartet} &
        {Effizienz [\si{\percent}]} \\
        \midrule
        121.78 & 28.6 & 358  & 12544 & 20006 & 62.7 \\
        244.7  &  7.6 & 714  & 5319  & 53162 & 10.0 \\
        344.3  & 26.5 & 1003 & 2117  & 18537 & 11.4 \\
        411.12 & 02.2 & 1196 & 1051  &  1539 & 68.3 \\
        443.96 & 03.1 & 1291 & 996   &  2168 & 45.9 \\
        778.9  & 12.9 & 2262 & 938   &  9024 & 10.4 \\
        964.08 & 14.6 & 2798 & 817   & 10213 &  8.0 \\
        1085.9 & 10.2 & 3150 & 429   &  7135 &  6.0 \\
        1112.1 & 13.6 & 3227 & 598   &  9513 &  6.3 \\
        1408.0 & 21.0 & 4084 & 372   & 14690 &  2.5 \\
        \bottomrule
    \end{tabular}
\end{table}

\subsection{Bestimmung einiger Detektoreigenschaften} % (fold)
\label{sub:detektoreigenschaften}
Mit Hilfe eines $^{137}$Cs-Strahlers werden im folgenden Abschnitt einige
Detektoreigenschaften bestimmt.
Das Spektrum des Stahlers ist in Abbildung \ref{fig:cs_spektrum} dargestellt.
\begin{figure}[htb]
    \centering
    \begin{subfigure}{0.49\linewidth}
        \centering
        \includegraphics[width=1.0\linewidth]{img/06_caesium.pdf}
        \caption{
            Spektrum des $^{137}$Cs-Strahlers.
        }
        \label{fig:cs_spektrum}
    \end{subfigure}
    \begin{subfigure}{0.49\linewidth}
        \centering
        \includegraphics[width=1.0\linewidth]{img/06_caesium_zoomed.pdf}
        \caption{
            Compton-Streuung im $^{137}$Cs-Spektrum.
        }
        \label{fig:cs_spektrum_zoom}
    \end{subfigure}
    \caption{
        Spektrum des $^{137}$Cs-Strahlers. In Abbildung \ref{fig:cs_spektrum}
        ist das gesamte Spektrum inklusive deutlich erkennbarem Photo-Peak
        dargestellt.
        Abbildung \ref{fig:cs_spektrum_zoom} stellt den Bereich der
        Compton-Streuung dar, wobei jeder Datenpunkt aufgetragen ist und
        im rot markierten Bereich eine Funktion der Gestalt \eqref{eqn:dsde}
        gefittet wird.
    }
\end{figure}

Zunächst wird der Photopeak des Spektrums genauer untersucht. Durch Fit einer
Gaußfunktion im Bereich des Peaks lassen sich die genaue Lage und der Inhalt
des Photopeaks bestimmen.
Für die Halb- und Zehntelwertsbreite $E_{1/2}$ und $E_{1/10}$ gilt mit Kenntnis
der Breite $\sigma$ der Gaußverteilung
\begin{align*}
    E_{1/2} = 2\sigma\sqrt{\ln 2}\,,\quad E_{1/10} = 2\sigma\sqrt{\ln 10}\,.
\end{align*}
Der Fit ist in Abbildung \ref{fig:cs_gauss} dargestellt, er liefert folgende
Werte für die Lage des Photopeaks $E_\gamma$, die Breiten $E_{1/2}$ und
$E_{1/10}$, sowie die Anzahl $N$ der Ereignisse im Peak:
\begin{align*}
    E_\gamma &= \SI{661.623+-0.006}{keV} \,,&\quad N &= \num{12250+-110}\,,\\
    E_{1/2} &= \SI{2.752+-0.010}{keV}\,,&\quad E_{1/10} &= \SI{5.016+-0.018}{keV}\,,\\
    E_{1/2}^\text{t} &= \num{2.35} \sqrt{\num{0.1} E_\gamma E_\text{EL}} = \SI{1.03+-0.00}{keV}\,.
\end{align*}
Der Wert $E_{1/2}^\text{t}$ bezeichnet den im Skript erwähnten Theoriewert
für die Halbwertsbreite.
Das Verhältnis der Halb- und Zehntelwertsbreiten beträgt
\begin{align*}
    \frac{E_{1/10}}{E_{1/2}} = \num{1.822+-0.010}\,,
\end{align*}
was mit dem erwarteten Verhältnis dieser Werte einer Gaußverteilung
übereinstimmt.
\begin{figure}
    \centering
    \includegraphics[width=0.7\linewidth]{img/06_caesium_fit.pdf}
    \caption{
        Spektrum des $^{137}$Cs-Strahlers im Bereich des Photo-Peaks. Durch
        die Datenpunkte wird eine Gaußfunktion gefittet, um die genaue Lage
        des Peaks zu bestimmen.
    }
    \label{fig:cs_gauss}
\end{figure}

Aschließend wird die Lage der Compton- und Rückstreukante bestimmt.
Auf Grund der Überlagerung der unterschiedlichen Wechselwirkungen im Detektor,
ist die Bestimmung dieser Kanten durch einen Fit schwierig.
Stattdessen werden die Maxima der Energieverteilungen in den entsprechenden
Bereichen mit der Position der Kanten identifiziert und der Fehler mit
der Breite eines Kanals geschätzt.
Dies liefert die Energie $E_\text{C}$ an der Compton-Kante, sowie $E_\text{R}$
an der Rückstreukante mit
\begin{align*}
    E_\text{C} &= \SI{463.5+-0.3}{keV}\,, &\quad E_\text{R} &= \SI{192.9+-0.3}{keV} \,,\\
    E_\text{C}^\text{t} &= \SI{477.302+-0.005}{keV}\,, &\quad E_\text{R}^\text{t} &= \SI{184.3206+-0.0005}{keV} \,.\\
\end{align*}
Die Theoretischen Werte $E^\text{t}$ lassen sich mit Hilfe der Formeln
\eqref{eqn:compton_peak} und \eqref{eqn:rueckstreu_peak} berechnen. Für
den Rückstreu-Peak wird ein Streuwinkel von $\phi = \SI{180}{\degree}$ angenommen.

Der Inhalt $I_\text{C}$ des Compton-Kontinuums wird durch numerische
Integration der im Skript vorgeschlagenen Funktion für
$\text{d}\sigma/\text{d}E$ von \SI{50}{keV} bis zum Compton-Peak bestimmt.
Die Paramter dieser Funktion werden dabei durch Fit an den in Abbildung
\ref{fig:cs_spektrum_zoom} markierten Bereich bestimmt. Damit ergibt sich
\begin{equation*}
    I_\text{C} = \num{47190+-220}\,.
\end{equation*}

Abschließend wird das Verhältnis der Anzahl an Ereignissen im Photo-Peak
und dem Compton-Kontinuum untersucht.
Mit Hilfe der Extinktionskoeffizienten $\mu$ für Compton- und Photo-Effekt
lassen sich unter Kenntnis der Detektorabmessungen $D$ die Absorptionswahrscheinlichkeiten
$W(D)=1-\exp(-\mu D)$ für beide Effekte vorhersagen. Die Koeffizienten $\mu$
werden in Abbildung \ref{german} abgelesen:
\begin{align*}
    \mu_\text{C} &= \num{0.42}\,, &\quad \mu_\text{P} &= \num{0.008}\,,\\
    \Rightarrow \quad W_\text{C} &\approx \SI{80}{\percent} &\quad W_\text{P} &\approx \SI{3}{\percent} \,.
\end{align*}
Das Compton-Kontinuum sollte somit im Vergleich zum Photo-Peak die etwa
\num{27}-fache Anzahl an Ereignissen beinhalten.
Die gemessenen Inhalte ergeben jedoch ein Verhältnis von etwa $4:1$.
% subsection bestimmung_einiger_detektoreigenschaften (end)

\subsection[Aktivität einer $^{133}$Ba-Quelle]{Aktivität einer $\mathbf{^{133}}$Ba-Quelle} % (fold)
\label{sub:ba_quelle}
In diesem Abschnitt wird das Spektrum einer $^{133}$Ba-Quelle vermessen.
Anschließend wird der Inhalt der einzelnen Peaks im Spektrum bestimmt,
woraus unter Berücksichtigung der Effizienzfunktion $Q$ und dem
Raumwinkelanteil $\Omega$ des Detektors auf die Aktivität $A_\text{ges}$
der Quelle geschlossen werden kann.
Das aufgenommene Spektrum ist in Abbildung \ref{fig:ba_spektrum} gezeigt.
Es werden die in Tabelle \ref{tab:barium} aufgeführten Peaks des Spektrums
mit den bekannten Gamma-Linien aus dem Skript identifiziert.
Mit Hilfe von Gleichung \eqref{eqn:effizienz} kann dann die Aktivität $A$
der Quelle als Parameter einer linearen Ausgleichsrechnung bestimmt werden.
Der Fit für die Bestimmung von $A_\text{ges}$ ist in Abbildung
\ref{fig:barium_aktivitaet} dargestellt und liefert
\begin{align*}
    A_\text{ges} = \SI{970+-190}{\becquerel}\,.
\end{align*}
\begin{figure}
    \centering
    \begin{subfigure}{0.49\linewidth}
        \includegraphics[width=1.\linewidth]{img/07_barium.pdf}
        \caption{
            Spektrum einer $^{133}$Ba-Probe.
        }
        \label{fig:ba_spektrum}
    \end{subfigure}
    \begin{subfigure}{0.49\linewidth}
        \includegraphics[width=1.\linewidth]{img/07_barium_activity.pdf}
        \caption{
            Fit zur Aktivitätsbestimmung der $^{133}$Ba-Probe.
        }
        \label{fig:barium_aktivitaet}
    \end{subfigure}
    \caption{
        Vermessung einer $^{133}$Ba-Quelle zur Aktivitätsbestimmung. Abbildung
        \ref{fig:ba_spektrum} beinhaltet das Spektrum der Qulle, während
        Abbildung \ref{fig:barium_aktivitaet} die Fit-Funktion zur Aktivitätsbestimmung
        darstellt.
    }
\end{figure}
\begin{table}
    \centering
    \caption{
        Detektierte Maxima mit jeweiligen Emissionswahrscheinlichkeiten des
        $^{133}$Ba-Spektrums.
    }
    \label{tab:barium}
    \begin{tabular}{%
        S[table-format=3.1]%
        S[table-format=5.0]%
        S[table-format=2.1]%
    }
        \toprule
        {Peak [\si{keV}]}  & {Ereignisse} & {Emissionsw.} \\
        \midrule
         81.0 & 11797 & 34.1 \\
        160.7 &   274 &  0.6 \\
        303.0 &  4099 & 18.3 \\
        356.1 & 11660 & 62.1 \\
        384.1 &  1567 &  8.9 \\
        \bottomrule
    \end{tabular}
\end{table}

\subsection{Aktivität von Holzkohlebrikkets} % (fold)
\label{sub:holzkohle}
Abschließend werden etwa \SI{500}{g} Holzkohlebrikkets auf Radioaktive
Isotope untersucht.
Die Probe wird dafür für eine Dauer von etwa \num{23} Stunden um den Detektor
herum verteilt.
Das Aufgenommene Spektrum ist in Abbildung \ref{fig:coal} dargestellt.
Tabelle \ref{tab:coal} führt einige Maxima des Energiespektrums auf und weist
diesen die Herkunft aus einem radioaktiven Nuklid zu.
Dabei wird auf die \texttt{LARA Database} \cite{lara} zurückgegriffen.
Obwohl das Spektrum mehr als die drei aufgeführten Peaks beinhaltet, stellt es
sich als äußerst schwierig heraus, für andere, als die aufgeführten Energien
eindeutige Quell-Isotope zu finden, da stets etliche Isotope mit ähnlichen
Gamma-Energien gefunden werden können.
\begin{figure}[htb]
    \centering
    \includegraphics[width=0.7\linewidth]{img/08_coal.pdf}
    \caption{
        Spektrum von etwa \SI{500}{g} Holzkohle-Brikkets, Marke Kaufland.
    }
    \label{fig:coal}
\end{figure}
\begin{table}[htb]
    \centering
    \caption{
        Einige Maxima des Spektrums einer Portion Holzkohle-Brikkets.
        Den jeweiligen Peaks wird anhand der $\gamma$-Energie ein Isotop
        zugeordnet, sofern das eindeutig möglich ist.
    }
    \label{tab:coal}
    \begin{tabular}{%
        S[table-format=4.1]%
        S[table-format=5.0]%
        l%
        l%
        S[table-format=2.2]%
    }
        \toprule
        {Peak [\si{keV}]}  & {Ereignisse} & Strahler   & Reihe      & {Emissionsw. [\si{\percent}]} \\
        \midrule
         609.2             &  330         & $^{214}$Bi & $^{238}$U  & 45.49 \\
         661.7             & 1862         & $^{137}$Cs &            & 84.99 \\
        1461.6             &  565         & $^{40}$K   &            & 10.55 \\
        \bottomrule
    \end{tabular}
\end{table}
Die Gesamtaktivität der Quelle beträgt etwa $A_\text{coal} = \SI{0.758+-0.004}{\becquerel}$,
wobei von einem Raumwinkelanteil der Probe von etwa \num{0.5} ausgangen wird.
Die Aktivität wird dabei durch einfache Summation über alle Ereignisse
bestimmt und der Fehler mit $\sqrt{N}$ geschätzt.
% Auf Basis der hier aufgeführten Peaks des Kohle-Spektrums wird die Aktivität
% der Probe abgeschätzt. Für die Peaks von $^{214}$Bi bei \SI{609}{keV},
% $^{137}$Cs bei \SI{661}{keV} und $^{40}$K bei \SI{1461}{keV} wird die Gesamte
% Aktivität $A_\text{i}$ des Isotops in der Probe mit
% \begin{equation*}
%     A_\text{i} \approx \frac{N_\text{i}}{p_\text{i} Q(E_\text{i})} \frac{1}{t_\text{coal}}
% \end{equation*}
% genähert. Dabei bezeichnet $N$ die Anzahl der Ereignisse im entsprechenden Peak,
% bestimmt durch Fit einer Gauß-Funktion, $p_\text{i}$ die Emissionswahrscheinlichkeit
% der entsprechenden Energie $E_\text{i}$ und $Q$ die Effizienz.
% Die Dauer der Messung wird mit $t_\text{coal}$ bezeichnet.
% Der durch die Probe abgedeckte Raumwinkelanteil wird mit \num{0.5} genähert.
% Insgesamt ergibt sich damit eine Aktivität der Holzkohlebrikkets von
% \begin{equation*}
%     A_\text{ges} = 2 \sum A_\text{i} \approx \SI{2.7+-1.5}{\becquerel}\,.
% \end{equation*}
% Der Fehler wird mit etwa \SI{50}{\percent} geschätzt, da keine Weiteren
% Bestandteile identifiziert werden können, diese jedoch einen großen Beitrag
% liefern.

\section{Diskussion}
\label{sec:diskussion}
Insgesamt kann in diesem Versuch die Präzision des Germanium-Detektors
hinsichtlich
der Positionen des Energiespektrums verschiedener Strahler verdeutlicht
werden.

Eine Schwierigkeit bei allen hier durchgeführten Messungen stellt allerdings
die bestimmung der Effizienzkurve $Q(E)$ dar.
Die hierfür genutzten Wertepaare unterliegen starken Schwankungen, was Fehler
der Größenordnung \SI{40}{\percent} Folge hat und weiter untersucht werden
sollte.
Die Lagen der Compton- und Rückstreukante im $^{137}$Cs-Spektrum
weichen um \SI{3}{\percent} beziehungsweise \SI{4.5}{\percent}
vom theoretischen Wert ab, was möglicherweise durch den nicht Berücksichtigten
Untergrund zurückzuführen ist.
Auch die Große Diskrepanz zwischen erwarteten Verhältnis der Inhalte von
Compton-Kon\-ti\-nu\-um und Photo-Peak kann ohne weiteres nicht erklärt werden.
Möglichwerweise sind beide Effekte stark korreliert, was weiter untersucht
werden müsste.

Abschließend stellt sich die Untersuchung der Holzkohle-Brikkets als sehr
interessant heraus. Gegen die Erwartungen fällt die naiv bestimmte Aktivität
von etwa \SI{0.8}{\becquerel} gering aus, wobei hieraus keine Rückschlüsse auf
etwaige Risiken für den jeweiligen Grillmeister geschlossen werden können,
da die Untersuchung lediglich Sensitivität auf Gamma-Emission liefert.
