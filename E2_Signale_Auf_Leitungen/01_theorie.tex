\section{Grundlagen} % (fold)
\label{sec:grundlagen}
Elektrische Leitungen stellen den grundlegenden Bestandteil moderner
Energie- und Telekommunikationsinfrastruktur dar.
In diesem Versuch werden daher verschiedene Konstanten zur klassifizierung
elektrischer Leitungen bestimmt. Zudem wird das Verhalten elektrischer
Signalimpulse auf Leitungen untersucht.

\subsection{Beschreibung einer elektrischer Leitungen} % (fold)
\label{sub:beschreibung}
Eine ideale elektrische Leitung lässt sich physikalisch durch ein Schaltbild
aus Induktivität $L$ und Kapazität $C$ beschreiben.
Zur Beschreibung einer realen Leitung müssen zusätzlich ein ohmscher Widerstand
$R$ und ein induktiver Widerstand $G$ betrachtet weden, wobei der Betrag des
ohmschen Widerstandes bei vielen metallischen Leitern den des induktiven
übersteigt.
Die entsprechenden Ersatzschaltbilder sind in Abbildung \ref{fig:schaltbild}
dargestellt.
\begin{figure}[h]
    \center
    \begin{subfigure}{0.39\linewidth}
        \center
        \begin{circuitikz}
            \ctikzset{label/align = smart}
            \draw (0,2) to[L=$L$] (3,2) -- (4,2);
            \draw (3,0) to[C, l_=$C$, *-*] (3,2);
            \draw (0,0) -- (4,0);
        \end{circuitikz}
    \end{subfigure}
    \begin{subfigure}{0.59\linewidth}
        \center
        \begin{circuitikz}
            \ctikzset{label/align = smart}
            \draw (0,2) -- (0.5,2) to[R=$R$] (2,2) to[L=$L$] (3.5,2) -- (6,2);
            \draw (4,0) to[R=$G$, *-*] (4,2);
            \draw (5,0) to[C, l_=$C$, *-*] (5,2);
            \draw (0,0) -- (6,0);
        \end{circuitikz}
    \end{subfigure}
    \caption{
        Ersatzschaltbild einer verlustfreien und verlustbehafteten Leitung.
    }
    \label{fig:schaltbild}
\end{figure}
Die Ausbreitung eines elektrischen Signals $U$ auf einem Leiter lässt sich
mathematisch durch eine gedämpfte harmonische Welle mit hin- und rücklaufenden
Anteilen in Ausbreitungsrichtung $z$, in Abhängigkeit der Zeit $t$ beschreiben:
\begin{equation*}
    U(z,t) = U_0\e^{-\gamma z}\e^{-\i\omega t}\,.
\end{equation*}
Hierbei geht mit dem Dämpfungsbelag $\alpha$ und dem Phasenbelag $\beta$
die Ausbreitungskonstante
$\gamma = \alpha + \i\beta = \sqrt{(R+\i\omega L)(G+\i\omega C)}$ ein.
% subsection beschreibung (end)

\subsection{Dispersion} % (fold)
\label{sub:dispersion}
Wie bei allen elektromagnetischen Wellen kann bei einem elektrischen Signal,
das durch Materie läuft, Disperson -- also eine frequenzabhängige
Phasengeschwindigkeit -- auftreten.
Dies ist in einer verlustbehafteten Leitung der Fall und die Form der Disperson
ist vom Aufbau und dem Wellenwiderstand $Z_0$ der Leitung abhängig.
Der Wellenwiderstand ist dabei für ein sinusförmiges Signal mit der Frequenz
$\omega$ definiert über:
\begin{equation*}
    Z_0 = \frac{U(\omega)}{I(\omega)}=\sqrt{\frac{R+\i\omega L}{G+\i\omega C}}\,.
\end{equation*}
% subsection dispersion (end)

\subsection{Beschreibung von Signalpulse} % (fold)
\label{sub:signalpulse}
Da in der Digitaltechnik häufig gepulste Singale übertragen werden, ist es
wichtig, das Verhalten von Signalpulsen auf Leitungen zu untersuchen.
Der hinlaufende Puls wird dabei an einer Quelle auf die zu untersuchende
Leitung gegeben, wobei er durch die Quellimpedanz $Z_\text{g}$ beeinflusst
wird.
Befindet sich am Ende der Leitung eine Last, wie zum Beispiel ein
Empfangsgerät, wird ein Teil des einlaufenden Pulses hier reflektiert, wobei
die Rücklaufende Welle wiederum durch die Lastimpedanz $Z_\text{L}$
beeinflusst wird und einen Phasensprung $\varphi_\Gamma$ vollziehen kann.
Die Spannung $U_\text{L}$, die die Last erreicht setzt sich anschließend aus
der Summe der ein- und auslaufenden Pulse $U_0$ und $U_\text{r}$ zusammen.
Der Quotient aus beiden Anteilen wird als Reflektionsfaktor $\Gamma$
bezeichnet:
\begin{equation*}
    \Gamma = \frac{U_\text{r}}{U_0} = \frac{Z_\text{L} - Z_0}{Z_\text{L} + Z_0} = \left|\Gamma\right|\e^{\i\varphi_\Gamma}\,.
\end{equation*}
Es wird deutlich, dass unter der Bedingung $Z_\text{L} = Z_0$ der reflektierte
Anteil der Welle verschwindet und der gesamte Puls an der Last anliegt.
In diesem Fall ist die Leitung angepasst.
% subsection signalpulse (end)

\subsection%
    [Bestimmung der Leitungsimpedanz $Z_0$]%
    {Bestimmung der Leitungsimpedanz $\mathbf{Z_0}$} % (fold)
\label{sub:impedanz}
Der Spannungsverlauf der reflektierten Spannung $U_0$ kann analytisch mit
Hilfe einer Laplacetransformation des reflektierten Signals bestimmt werden.
Für einen idealen Rechteckpuls als Eingangspuls $U_0$ ergibt sich bei einer
Induktivität $L$ als Abschlußimpedanz ein reflektiertes Signal
\begin{equation*}
    U_\text{r}(t) = - U_0 + 2U_0\e^{-t \frac{Z_0}{L}}\,.
\end{equation*}
Somit kann beispielsweise in dieser Schaltung durch Messung von $U_\text{r}{t}$
die Impedanz $Z_0$ bestimmt werden.
% subsection impedanz (end)

\subsection{Störstellen und Impulsfahrplan} % (fold)
\label{sub:impulsfahrplan}

% subsection impulsfahrplan (end)
% section grundlagen (end)
