\section{Grundlagen} % (fold)
\label{sec:grundlagen}
Elektrische Leitungen stellen den grundlegenden Bestandteil moderner
Energie- und Telekommunikationsinfrastruktur dar.
In diesem Versuch werden daher verschiedene Konstanten zur klassifizierung
elektrischer Leitungen bestimmt. Zudem wird das Verhalten elektrischer
Signalimpulse auf Leitungen untersucht.

\subsection{Beschreibung elektrischer Leitungen} % (fold)
\label{sub:beschreibung}
Eine ideale elektrische Leitung lässt sich physikalisch durch ein Schaltbild
aus Induktivität $L$ und Kapazität $C$ beschreiben.
Zur Beschreibung einer realen Leitung müssen zusätzlich ein ohmscher Widerstand
$R$ und ein induktiver Widerstand $G$ betrachtet weden, wobei der Betrag des
ohmschen Widerstandes bei vielen metallischen Leitern den des induktiven
überragt.
\begin{figure}[h]
    \center
    \begin{subfigure}{0.39\linewidth}
        \center
        \begin{circuitikz}
            \ctikzset{label/align = smart}
            \draw (0,2) to[L=$L$] (3,2) -- (4,2);
            \draw (3,0) to[C, l_=$C$, *-*] (3,2);
            \draw (0,0) -- (4,0);
        \end{circuitikz}
    \end{subfigure}
    \begin{subfigure}{0.59\linewidth}
        \center
        \begin{circuitikz}
            \ctikzset{label/align = smart}
            \draw (0,2) -- (0.5,2) to[R=$R$] (2,2) to[L=$L$] (3.5,2) -- (6,2);
            \draw (4,0) to[R=$G$, *-*] (4,2);
            \draw (5,0) to[C, l_=$C$, *-*] (5,2);
            \draw (0,0) -- (6,0);
        \end{circuitikz}
    \end{subfigure}
    \caption{
        Ersatzschaltbild einer verlustfreien und verlustbehafteten Leitung.
    }
\end{figure}
% subsection beschreibung (end)
% section grundlagen (end)
