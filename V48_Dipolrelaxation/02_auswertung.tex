\section{Auswertung}
\label{sec:auswertung}
Es werden in diesem Versuch zwei Messreihen mit unterschiedlicher Heizrate
$b$ durchgeführt. Zunächst müssen die Messwerte von Untergründen bereinigt
werden, auf die im Folgenden eingegangen wird.

Alle Fehler in dieser Auswertung werden -- wenn nicht anders beschrieben --
mittels Gaußscher Fehlerfortpflanzungen mit Hilfe des \texttt{python}-Paketes
\texttt{uncertainties} \cite{py-uncertainties} berechnet.

Zur Beschreibung des oben genannten Untergrundes wird eine lineare Funktion
$I_\text{bkg}(T)$ von den Daten abgezogen. Die Variablen der Funktion werden
durch Fit an einen Teil der Daten bestimmt, der in Abbildungen
\ref{fig:data-a} und \ref{fig:data-b} markiert ist.
Die Fitfunktion hat die Gestalt
\begin{equation*}
    I_\text{bkg}(T) = A\cdot T + B\,.
\end{equation*}
Das Resultat, sowie der Fit sind ebenfalls in oben genannter Abbildung
dargestellt und die entsprechenden Daten in den Tabellen \ref{tab:data1} und
\ref{tab:data2} aufgeführt.
\begin{figure}
    \centering
    \includegraphics[width=0.8\linewidth]{build/plots/cleaned_data-set1.pdf}
    \caption{Messdaten, roh und bereinigt. Der zu erwartende Peak ist
    ausdrücklich schlecht zu erkennen.}
    \label{fig:data-a}
\end{figure}
\begin{figure}
    \centering
    \includegraphics[width=0.8\linewidth]{build/plots/cleaned_data-set2.pdf}
    \caption{Messdaten zur zweiten Messreihe, roh und bereinigt. Hier ist
    der zu erwartende Peak gut zu erkennen.}
    \label{fig:data-b}
\end{figure}
\begin{table}
    \centering
    \caption{Messdaten der ersten Messreihe.}
    \label{tab:data1}
    \begin{tabular}{
    S[table-format=3.1]
    S[table-format=2.1]
    S[table-format=2.2]
    @{\hskip 3em}
    S[table-format=3.1]
    S[table-format=2.1]
    S[table-format=1.2]
    @{\hskip 3em}
    S[table-format=3.1]
    S[table-format=2.1]
    S[table-format=1.2]
}
\toprule
$T/\si{\kelvin}$&
$I/\si{\pico\ampere}$&
$I_\mathrm{cl}/\si{\pico\ampere}$&
$T/\si{\kelvin}$&
$I/\si{\pico\ampere}$&
$I_\mathrm{cl}/\si{\pico\ampere}$&
$T/\si{\kelvin}$&
$I/\si{\pico\ampere}$&
$I_\mathrm{cl}/\si{\pico\ampere}$\\
\midrule
231.8 & 18.0 & 12.37 & 277.3 & 11.0 & 4.19 & 303.0 & 14.0 & 3.04 \\
233.8 & 15.0 & 9.36 & 279.4 & 11.0 & 4.02 & 304.0 & 14.0 & 2.73 \\
236.2 & 12.0 & 6.34 & 281.4 & 12.0 & 4.85 & 304.8 & 14.0 & 2.43 \\
238.3 & 11.0 & 5.32 & 282.4 & 13.0 & 5.76 & 305.8 & 14.0 & 2.11 \\
240.6 & 10.0 & 4.31 & 283.4 & 13.0 & 5.66 & 306.8 & 14.0 & 1.74 \\
242.6 & 9.0 & 3.29 & 284.4 & 13.0 & 5.55 & 307.8 & 15.0 & 2.35 \\
244.4 & 9.0 & 3.27 & 285.3 & 14.0 & 6.46 & 308.6 & 15.0 & 1.98 \\
246.2 & 9.0 & 3.25 & 286.3 & 14.0 & 6.34 & 309.6 & 15.0 & 1.55 \\
248.0 & 8.0 & 2.22 & 287.3 & 14.0 & 6.21 & 310.8 & 15.0 & 1.04 \\
249.8 & 8.0 & 2.2 & 288.3 & 14.0 & 6.08 & 311.8 & 16.0 & 1.55 \\
251.8 & 8.0 & 2.16 & 289.4 & 14.0 & 5.93 & 312.6 & 16.0 & 1.09 \\
253.7 & 7.0 & 1.13 & 290.4 & 14.0 & 5.78 & 313.6 & 17.0 & 1.55 \\
255.6 & 7.0 & 1.09 & 291.5 & 15.0 & 6.61 & 314.6 & 17.0 & 0.97 \\
257.5 & 7.0 & 1.05 & 292.6 & 15.0 & 6.43 & 315.6 & 18.0 & 1.36 \\
259.4 & 6.0 & 0.0 & 293.8 & 15.0 & 6.24 & 316.5 & 18.0 & 0.78 \\
261.2 & 7.0 & 0.95 & 294.8 & 15.0 & 6.05 & 317.4 & 19.0 & 1.18 \\
263.0 & 7.0 & 0.89 & 295.8 & 15.0 & 5.83 & 318.4 & 20.0 & 1.46 \\
265.0 & 7.0 & 0.83 & 296.8 & 15.0 & 5.62 & 319.4 & 21.0 & 1.71 \\
267.0 & 7.0 & 0.75 & 297.8 & 14.0 & 4.4 & 320.4 & 21.0 & 0.91 \\
268.8 & 7.0 & 0.67 & 299.0 & 14.0 & 4.14 & 321.5 & 22.0 & 0.98 \\
270.8 & 8.0 & 1.58 & 300.0 & 14.0 & 3.88 & 322.6 & 24.0 & 1.99 \\
273.0 & 9.0 & 2.47 & 301.0 & 14.0 & 3.62 & 323.6 & 25.0 & 2.03 \\
275.2 & 10.0 & 3.34 & 302.0 & 14.0 & 3.34 &  &  &  \\
\bottomrule
\end{tabular}

\end{table}
\begin{table}
    \centering
    \caption{Messdaten der zweiten Messreihe.}
    \label{tab:data2}
    \begin{tabular}{
    S[table-format=3.1]
    S[table-format=2.1]
    S[table-format=2.2]
    @{\hskip 3em}
    S[table-format=3.1]
    S[table-format=2.1]
    S[table-format=1.2]
    @{\hskip 3em}
    S[table-format=3.1]
    S[table-format=2.1]
    S[table-format=1.2]
}
\toprule
$T/\si{\kelvin}$ & $I/\si{\pico\ampere}$ & $I_\mathrm{cl}/\si{\pico\ampere}$ & $T/\si{\kelvin}$ & $I/\si{\pico\ampere}$ & $I_\mathrm{cl}/\si{\pico\ampere}$ & $T/\si{\kelvin}$ & $I/\si{\pico\ampere}$ & $I_\mathrm{cl}/\si{\pico\ampere}$\\
\midrule
209.2 & 7.0 & 19.98 & 268.2 & 21.5 & 19.77 & 290.5 & 15.0 & 7.68 \\
213.6 & 6.5 & 18.36 & 268.8 & 21.5 & 19.59 & 292.0 & 15.5 & 7.83 \\
217.0 & 6.0 & 17.04 & 269.3 & 21.0 & 18.97 & 293.2 & 15.0 & 7.01 \\
220.6 & 6.0 & 16.11 & 270.0 & 20.0 & 17.82 & 294.5 & 15.0 & 6.68 \\
224.7 & 6.0 & 15.09 & 270.5 & 19.5 & 17.17 & 296.0 & 15.0 & 6.33 \\
228.7 & 5.5 & 13.59 & 271.0 & 19.0 & 16.54 & 296.8 & 15.0 & 6.11 \\
232.7 & 5.5 & 12.6 & 271.6 & 18.5 & 15.89 & 298.2 & 14.5 & 5.29 \\
236.6 & 5.0 & 11.12 & 272.2 & 18.0 & 15.25 & 299.3 & 14.5 & 4.99 \\
240.6 & 5.5 & 10.65 & 272.8 & 17.0 & 14.1 & 300.5 & 14.0 & 4.19 \\
244.6 & 5.0 & 9.15 & 273.3 & 17.0 & 13.97 & 301.8 & 14.0 & 3.86 \\
248.7 & 5.0 & 8.11 & 274.0 & 16.0 & 12.82 & 303.2 & 14.0 & 3.54 \\
253.0 & 5.5 & 7.56 & 275.0 & 15.0 & 11.57 & 304.3 & 13.5 & 2.74 \\
257.2 & 7.5 & 8.49 & 275.5 & 14.5 & 10.92 & 305.6 & 13.5 & 2.42 \\
261.3 & 13.0 & 12.96 & 276.2 & 14.5 & 10.77 & 307.2 & 13.5 & 2.04 \\
262.2 & 15.0 & 14.74 & 276.8 & 14.0 & 10.12 & 308.6 & 13.5 & 1.67 \\
263.2 & 17.5 & 16.99 & 277.4 & 14.0 & 9.95 & 311.2 & 13.5 & 1.04 \\
264.0 & 18.5 & 17.82 & 278.2 & 14.0 & 9.77 & 313.5 & 14.0 & 0.95 \\
264.5 & 19.5 & 18.67 & 279.0 & 14.0 & 9.57 & 316.0 & 14.0 & 0.35 \\
265.2 & 20.5 & 19.52 & 279.6 & 13.5 & 8.9 & 318.0 & 14.5 & 0.32 \\
265.8 & 21.5 & 20.37 & 284.4 & 14.0 & 8.2 & 319.8 & 15.0 & 0.4 \\
266.3 & 22.0 & 20.72 & 286.3 & 14.5 & 8.23 & 321.3 & 15.0 & 0.0 \\
267.0 & 22.0 & 20.57 & 287.5 & 15.0 & 8.43 & 322.8 & 15.5 & 0.13 \\
267.6 & 22.0 & 20.39 & 289.2 & 15.0 & 8.03 & 324.2 & 16.5 & 0.78 \\
\bottomrule
\end{tabular}

\end{table}
Bei der ersten Messreihe ist auffällig, dass nur ein einzelner, flacher Peak
zu erkennen ist. Dies macht die spätere Auswertung extrem schwierig.
Aus diesem Grund werden die Ergebnisse der erste Messreihe im weiteren
Verlauf nicht betrachtet.

\subsection{Approximative Ausgleichsrechnung}
\label{subsec:approx}
Zunächst wird die in Abschnitt \ref{sub:berechnung_der_aktivierungsenergie_w_}
aufgeführte Gleichung \ref{eqn:approx} zur Bestimmung der Aktivierungsenergie
$W$ benutzt. Dabei wird diese mit einem nicht-linearen Fit in einem Bereich
$T$ zwischen \SIrange{250}{267}{\kelvin} an die Daten angepasst.
Das Ergebnis des Fits ist in Abbildung \ref{fig:fit_approx_set2} dargestellt.
Der Fit liefert
\begin{equation*}
    \input{build/tex/W_approx_set2.tex}
\end{equation*}
\begin{figure}
    \centering
    \includegraphics[width=0.8\linewidth]{build/plots/fit_approx_set2.pdf}
    \caption{Ausgleichskurve zur approximativen Bestimmung der
    Aktivierungsenergie $W$. Die für den Fit benutzten Datenpunkte
    sind grün markiert.}
    \label{fig:fit_approx_set2}
\end{figure}

\subsection{Ausgleichsrechnung mit Integration}
\label{subsec:integration}
Als zweite Methode zur Bestimmung der Aktivierungsenergie $W$ wird ein
linearer Fit der Form
\begin{equation*}
    F(T) = A\cdot\frac{1}{T} + B
\end{equation*}
an die inversen Temperaturdaten $1/T$ durchgeführt.
Hierbei ist $F(T)$ wie in Gleichung \ref{eqn:integrate} definiert als
\begin{equation*}
    F(T) = \frac{\int_T^\ast i(T^\prime)\mathrm{d}T^\prime}{i(T)}\,.
\end{equation*}
Aus der Steigung $A$ lässt sich somit die Aktivierungsenergie $W$
berechnen durch
\begin{equation*}
    W = A\cdot k_\text{B}\,.
\end{equation*}
Außerdem wird schließlich die Relaxationszeit $\tau_0$ aus der Temperatur
bei maximalem Strom $T_\text{max}$ bestimmt:
\begin{equation*}
    \tau_0 = \frac{k_\text{B}T_\text{max}^2}{Wb}
             \exp\!\left[-\frac{W}{k_\text{B}T_\text{max}} \right]
\end{equation*}
Der Fit ist in Abbildung \ref{fig:integrate_fit} dargestellt, er liefert
\begin{equation*}
    \input{build/tex/W_integrated_set2.tex} \qquad\text{und}\qquad \input{build/tex/tau0_integrated_set2.tex}\,.
\end{equation*}
\begin{figure}
    \centering
    \includegraphics[width=0.9\linewidth]{build/plots/integrated-fit-set2.pdf}
    \caption{Fit an linearisierte Daten der zweiten Messung.}
    \label{fig:integrate_fit}
\end{figure}

\newpage
\section{Diskussion}
\label{sec:diskussion}
Wie oben bereits erwähnt, fehlt in dieser Auswertung die erste Messreihe.
Die Daten wären zur Validierung der mit dem zweiten Datensatz ermittelten Werte
für $W$ und $\tau_0$ nützlich gewesen und hätten die große Unsicherheit
auf den Wert $\tau_0$ möglicherweise reduzieren können.

Zudem wird der Untergrund der Messunge möglicherweise nicht ausreichend durch
eine Lineare Funktion beschrieben. Die Wahl der für den Untergrundfit gewählten
Datenpunkte scheint in gewissem Maße willkürlich, was die Aussagekraft der
hier ermittelten Werte weiter einschränkt.

Der Grundsätzliche Nachweis einer Dipolrelaxation im hier betrachten Modell
kann jedoch verifiziert werden.
